\documentclass[gradu]{tktltiki}
\usepackage{amsmath}
\begin{document}

\title{Referaatti artikkelista The "Boston" School Choice Mechanism}
\author{Jani Rahkola}
\date{\today}
\level{Referaatti}

\maketitle

\doublespacing

\faculty{Matemaattis-luonnontieteellinen}
\department{Tietojenkäsittelytieteen laitos}
\subject{Tietojenkäsittelytiede}

\pagenumbering{arabic}
\hyphenation{mo-no-to-ni-suu-den}

Fuhito Kojiman ja M. Utku Ünverin kuvaavat niin sanotun Bostonin
mekanismin toimintaa ja ominaisuuksia ensin sanallisesti. Sen jälkeen
he määrittelevät useita aksioomia ja karakterisoivat mekanismin sen
kahdessa eri tapauksessa. Toinen karakterisaatio koskee tapausta,
jossa toisen joukon alkioilla on jaettavana useita resursseja. Toinen
tapaus on vastaava kuin klassisessa pysyvän avioliiton ongelmassa,
jossa kummankin joukon alkioilla on tasan yksi resurssi.

Bostonin mekanismi on yleisesti käytössä oppilaitosten
valintajärjestelmissä. Niissä mekanismi toimii tapana sijoittaa
hakijat oppilaitosten tarjoamiin paikkoihin hakijoiden ja
oppilaitosten järjestelmään ennalta toimittamien preferenssilistojen
perusteella. Algoritmi etenee kierroksittain. Ensimmäisellä
kierroksella jokaiseen oppilaitokseen sijoitetaan oppilaitoksen listan
mukaisessa järjestyksessä ne hakijat jotka ovat listanneet
oppilaitoksen ensimmäiseksi vaihtoehdokseen. Toisella kierroksella
vielä avoinna olevat paikat täytetään oppilaitoksen toiseksi
listanneilla hakijoilla ja näin jatketaan, kunnes kaikki paikat ovat
täytetty tai kaikki hakijat ovat sijoitettu johonkin oppilaitokseen.

Bostonin mekanismissa on kuitenkin teoreettisia ja käytännön ongelmia.
Se ei tuota pysyvää sijoittelua, sillä hakija voi jäädä valitsematta
oppilaitokseen johon häntä alemmalla sijalla oleva toinen hakija
valitaan. Lisäksi mekanismin tuottamaan tulokseen on mahdollista
vaikuttaa muuttamalla preferenssilistaansa. Mekanismi ei siis ole
strategiankestävä.

Galen ja Shapleyn viivytetyn valinnan mekanismi (deferred acceptance
mechanism) takaa sekä pysyvän sijoittelun että strategiankestävyyden.
Lisäksi kokeellisessa ympäristössä on havaittu, että useammat hakijat
manipuloivat preferenssilistaansa Bostonin mekanismin ollessa käytössä
kuin viivytetyn valinna mekanismin tapauksessa. Bostonin mekanismissa
preferenssilistojen manipulointi johtaa useammin tulokseen, jossa
listaansa manipuloineet hakijat sijoittuvat paremmin
totuudenmukaisesti listansa täyttäneiden kustannuksella.

Bostonin mekanismilla on myös hyvät puolensa. Sen on osoitettu olevan
tietyissä tilanteissa hakijoiden kannalta parempi kuin viivytetyn
valinnan mekanismi. Yleisessäkin tilanteessa osa totuudenmukaisesti
preferenssilistansa täyttäneistä saattaa sijoittua paremmin Bostonin
mekanismissa. Hakijan mahdollisuutta vaikuttaa sijoitteluun
listaamalla tietty oppilaitos korkealle listassaan voidaan myös pitää
Bostonin mekanismin vahvuutena. Näin hakijoilla on suurempi
mahdollisuus vaikuttaa suotuisasti sijoitteluun. Tästä
mahdollisuudesta seuraavat kuitenkin myös useat Bostonin mekanismin
ongelmat.

Aksiomaattista lähestymistapaa varten luodaan sijoittelusta seuraava
malli. Olkoon \(I\) hakijoiden ja \(C\) oppilaitosten epätyhjä ja
äärellinen joukko. Jokainen oppilas on joko sijoitettu johonkin
oppilaitokseen tai vielä sijoittamatta. Merkitään sijoittamatta
olemista sijoituksella oppilaitokseen \(\emptyset\).

Jokaisella oppilaitoksella \(c \in C\) on \(q_c\) avoinna olevaa
paikkaa ja \(q_{\emptyset} = \infty\). Olkoon \(q = (q_c)_{c \in C}\)
kaikkien oppilaitosten avoimia paikkoja kuvaava vektori.

Jokaisella hakijalla \(i\) on preferenssirelaatio \(P_i\) joukossa \(C
\cup \{\emptyset\}\). Olkoon \(P = (P_i)_i \in I\)
preferenssiprofiili. Merkitään \(P_i(c) = l\) oppilaitoksen \(c\)
sijoitusta hakijan \(i\) listalla. Nyt kaikilla \(c, d \in C\) pätee
\(P_i(c) < P_i(d)\) jos ja vain jos \(cP_id\). Siis hakija \(i\)
toivoo pääsevänsä mieluummin oppilaitokseen \(c\) kuin \(d\). Olkoon
lisäksi \(cR_id\) jos ja vain jos \(P_i(c) \leq P_i(d)\).

Sijoittelu on funktio \(\mu : I \rightarrow C \cup \{\emptyset\}\)
missä \(\mu(i)\) on oppilaitos johon hakija \(i\) on sijoitettu.
Merkitään oppilaitokseen \(c\) sijoitettujen hakijoiden joukkoa
\(\mu_{c} = \{i \in I : \mu(i) = c\}\). Sijoittelun tulee
täyttää ehto \(|\mu_c| < q_c\) kaikilla \(c \in C \cup
\{\emptyset\}\). Siis oppilaitokseen ei voi olla sijoitettuna avoimia
paikkoja useampaa hakijaa.

\(I, C, P \text{ ja } q\) määrittelevät valintaongelman.
Valintamekanismi on algoritmi joka antaa sijoittelun jokaiseen
ongelmaan. Kiinnittämällä joukot \(I\) ja \(C\) voimme merkitä
ongelmaa preferenssiprofiilin ja avoimien paikkojen määrän avulla
\([P;q]\). Olkoon \(M[q]\) ongelman \([P;q]\) kaikkien sijoittelujen
joukko ja \(\gamma [P;q] \in M[q]\) mekanismin \(\gamma\) antama
sijoittelu ongelmalle \([P;q]\).

Ensimmäistä aksioomaa varten määritellään \\ \(I_c = \{i \in I :
\gamma [P;q](i) = c\) jollain \(P\) ja \(q\}\) olemaan niiden
hakijoiden joukko, jotka sijoitetaan oppilaitokseen \(c\) jollakin
preferenssiprofiililla \(P\) ja avoimien paikkojen määrällä \(q\)
käytettäessa mekanismia \(\gamma\). Nyt \emph{sijoittelu \(\mu\)
  noudattaa preferenssilistoja} mekanismissa \(\gamma\),
jos \[\text{kaikilla }c \in C \cup \{\emptyset\}, i \in I_c\text{ ja }
j \in \mu_c\text{ pätee }cP_{i}\mu(i) \Rightarrow |\mu_c| = q_c\text{
  ja }P_j(c) \leq P_i(c).\] Siis jos hakija \(i \in I_c\) sijoitetaan
\(c\):tä epämieluisempaan oppilaitokseen, on \(c\):n kaikki paikat
täytetty hakijoilla jotka sijoittivat \(c\):n vähintään yhtä korkealle
kuin \(i\). \emph{Mekanismi noudattaa preferenssilistoja}, jos se
löytää jokaiseen ongelmaan sijoittelun joka noudattaa
preferenssilistoja.

Sijoittelumekanismin strategiankestävyys voidaan esittää formaalisti
Maskin monotonisuuden avulla. Se vaatii, että mikäli hakija tulisi
sijoitetuksi oppilaitokseen \(c\), muuttuu sijoittelun tulos vain, jos
hakija nostaa listassaan \(c\):tä alempana olleen oppilaitoksen
\(c\):tä ylemmäs. Bostonin mekanismi ei kuitenkaan ole Maskin
monotoninen, sillä hakija voi muuttaa sijoittelun tulosta lyhentämällä
listaansa sen yläpäästä. Lieventämällä Maskin monotonisuuden ehtoja
saadaan kuitenkin aikaan edelleen mielenkiintoinen ominaisuus, jonka
Bostonin mekanismi toteuttaa.

Preferenssiprofiili \(P'\) on \emph{monotoninen transformaatio}
profiilista \(P\), jos \(\forall i \in I : \forall b \in C \cup
\{\emptyset\} : bP'_{i}c \Rightarrow bP_{i}c \text{ missä } c =
\mu(i)\). Olkoon lisäksi \[U_i(P,\mu) = \{j \in I : P_j(\mu (i)) \leq
P_i(\mu (i)) \text{ ja } P_j(\mu (i)) \leq P_j(\mu (j))\}\]
\[V_i(P,\mu) = \{j \in I : P_j(\mu (i)) < P_i(\mu (i)) \text{ ja } P_j(\mu
(i)) \leq P_j(\mu (j))\}\] Nyt \(P'\) on \emph{sijoitusta noudattava
  monotoninen transformaatio} profiilista \(P\) (\(P'\) s. n. m. t.
\(P\)), jos kaikilla \(i\) joilla \(\mu (i) \in C\) pätee
\(U_i(P',\mu) \subseteq U_i(P,\mu) \text{ ja } V_i(P',\mu) \subseteq
V_i(P,\mu)\). Nyt mekanismi \(\gamma\) on \emph{sijoitusta noudattaen
  Maskin monotoninen}, jos \[P'\text{:n ollessa s. n. m. t.
  profiilista }P \text{ pätee }\gamma [P';q] = \gamma [P;q].\] Nyt
siis sijoittelun tulos muuttuu vain kuten Maskin monotonisella
mekanismilla, tai jos hakijan \(i\) preferenssilistan muutos lisää
kilpailua pääsystä johonkin muuhun kouluun kuin \(\mu (i)\).

Mekanismi \(\gamma\) on \emph{resurssimonotoninen}, jos kaikilla
preferenssiprofiileilla \(P\) pätee \((\forall c \in C : q_c \geq
q'_c) \Rightarrow (\forall i \in I : \gamma [P;q](i) R_i \gamma
[P;q'](i))\). Siis jos avoimien paikkojen määrää lisätään, jokaisen
hakijan sijoitus mahdollisesti paranee.

Mekanismi \(\gamma\) on \emph{yksilöllisesti rationaalinen}, jos
kaikilla \(P \text{ ja } q \text{ sekä } i \in I\) pätee \(\gamma
[P;q](i) R_i \emptyset\). Siis jokainen hakija pitää saamaansa
sijoitusta parempana kuin sijoittumatta jäämistä.

Olkoon \(P\textsuperscript{\(\emptyset\)}\) preferenssiprofiili, jossa
hakija \(i\) ei halua tulla sijoitetuksi. Mekanismi \(\gamma\) on
\emph{populaatiomonotoninen}, jos kaikilla preferenssiprofiileilla
\(P\) pätee \\ \(\gamma [P\)\textsuperscript{\(\emptyset\)}\(; q](j)
R_j \gamma [P;q](j)\) kaikilla \(j \neq i\). Siis jos hakija jättäytyy
pois valinnasta, jokaisen jäljelle jääneen sijoitus mahdollisesti
paranee.

Olkoon \(q\textsuperscript{\(\emptyset\)}\)\(= q_{\gamma [P;q](i)} -
1\). Mekanismi \(\gamma\) on \emph{konsistentti}, jos \(\gamma
[P\)\textsuperscript{\(\emptyset\)}\(;q\)\textsuperscript{\(\emptyset\)}\(](j)
= \gamma [P;q](j)\) kaikilla \(j \neq i\). Siis jos sekä hakija että
avoin paikka johon hänet olisi sijoitettu poistetaan, sijoittelun
tulos ei jäljelle jääneiden hakijoiden osalta muutu.

Nyt Bostonin mekanismi voidaan karaterisoida seuraavasti. Mekanismi
\(\gamma\) on \emph{Bostonin mekanismi}, jos ja vain jos \(\gamma\)
noudattaa preferenssilistoja, on konsistentti, resurssimonotoninen ja
preferenssilistoja noudattaen Maskin monotoninen.

Monissa tapauksissa kummallakin osapuolella on vain yksi resurssi.
Näin on esimerkiksi jaettaessa toimistoja tai muita jakamattomia
resursseja hakijajoukon kesken. Kiinnitetään siis \(q_c = 1\) kaikilla
\(c \in C\). Nyt \(\gamma\) on Bostonin mekanismi, jos ja vain jos se
noudattaa preferenssilistoja, on yksilöllisesti rationaalinen,
populaatiomonotoninen ja preferenssilistoja noudattaen Maskin
monotoninen.
\end{document}
