\documentclass[gradu, twoside]{tktltiki}
\usepackage{amsmath}
\begin{document}

\title{Vakaan avioliiton ongelman ja sen muunnelmien aikavaativuuksia}
\author{Jani Rahkola}
\date{\today}
\level{}

\maketitle

\doublespacing

\faculty{Matemaattis-luonnontieteellinen}
\department{Tietojenkäsittelytieteen laitos}
\subject{Tietojenkäsittelytiede}

\begin{abstract}
Vakaan avioliiton ongelmassa miesten joukon $M$ ja naisten joukon $N$
välille pyritään löytämään kuvaus $\mu : M \rightarrow N$ noudattaen
miesten ja naisten mieltymyksiä kuvaavia järjestysrelaatioita $P_M =
\{P_a \subset N \times N | a \in M\}$ ja $P_N = \{P_b \subset M \times
M | b \in N\}$. Gale ja Shapley antavat ongelman esittelyn yhteydessä
algoritmin joka ratkaisee ongelman ajassa $O(n^2)$. Ng osoittaa, että
Galen ja Shapleyn algoritmi on suoritusajaltaan optimaalinen.

Käytännön kannalta on usein tarpeellista muokata ongelmanasettelua
niin, että järjestysrelaatioiden ei tarvitse kattaa koko vastakkaisen
sukupuolen joukkoa, tai niissä on luvallista olla tasatilanteita.
Molemmat näistä muunnoksista voidaan ratkaista ajassa $O(n^2)$, mutta
yhdessä ne tekevät ongelmasta NP-täydellisen.
\end{abstract}

\mytableofcontents

\section{Johdanto}

Gale ja Shapley määrittelevät vakaan avioliiton ongelman (stable
marriage problem) artikkelissaan College Admissions and the Stability
of Marriage \cite{galeshapley62} jossa he esittävät ratkaisun
yliopistojen valintaongelmaan. Vakaan avioliiton ongelma on
yksinkertaistettu tilanne, jossa joukosta naisia ja miehiä pyritään
muodostamaan pareja noudattaen sekä miesten että naisten mieltymyksiä.
Gale ja Shapley määrittelevät vakauden (stability) ja optimaalisuuden
käsitteen kuvaamaan hyviä ratkaisuja ongelmaan.

Ensimmäisessä luvussa annan tarkan määritelmän vakaan avioliiton
ongelmalle, ratkaisun vakaudelle ja sen optimaalisuudelle. Seuraavassa
luvussa esittelen Galen ja Shapleyn esittelemän algoritmin ratkaisun
löytämiseksi. Saman luvun lopussa käyn läpi analyysin algoritmin
suoritusajasta. Kolmannessa luvussa esittelen tuloksen joka osoittaa
Galen ja Shapleyn algoritmin olevan suoritusajaltaan optimaalinen.

Päätösluvussaan Gale ja Shapley toteavat luomansa vakaan avioliiton
ongelman olevan hyvin teoreettinen ja kaukana alkuperäisestä
yliopistojen valintaongelmasta. Ongelman ja sen ratkaisun
sovellettavuutta voidaan kuitenkin parantaa luopumalla joistakin
alkuperäisen määrittelyn oletuksista. Neljäs luku käsittelee
muunnelmaa, jossa naisilla ja miehillä on mahdollisuus ilmaista
olevansa yhtä mieltyneitä kahteen tai useampaan vastakkaisen
sukupuolen edustajaan. Tämä muunnelma voidaan ratkaista muokkaamalla
Galen ja Shapleyn algoritmia suoritusajan pysyessä muuttumattomana
\cite{manlove02}. Viidennessä luvussa käsittelen muunnelmaa jossa
miehillä ja naisilla on mahdollisuus ilmoittaa etteivät he missään
tapauksessa halua muodostaa paria joidenkin vastakkaisen sukupuolen
edustajien kanssa. Myös tämän ongelman ratkaisuun on suoritusajaltaan
muuttumaton muunnelma Galen ja Shapleyn algoritmista
\cite{gusfield89}.

Mainitut kaksi muunnelmaa mahdollistavat niitä koskevien tulosten
soveltamisen moniin käytännön ongelmiin \cite{manlove02}. Kuudennessa
luvussa käsittelen muunnelmien yhdistelmää, jonka ratkaiseminen on
ongelmana NP-täydellinen \cite{manlove02}.

\section{Vakaan avioliiton ongelma}

Määritellään vakaan avioliiton ongelma seuraavasti. Olkoon $M$ joukko
miehiä ja $N$ joukko naisia. Olkoon lisäksi $P_M = \{P_a \subset N
\times N | a \in M\}$ ja $P_N = \{P_b \subset M \times M | b \in N\}$
joukot täydellisiä järjestysrelaatioita jotka kertovat mihin
järjestykseen mies $a$ (vastaavasti nainen $b$) sijoittaa vastakkaisen
sukupuolen edustajat. Jos siis $x,y \in M$ ja $xP_by$ pitää nainen $b$
enemmän miehestä $x$ kuin $y$. Nyt nelikko $(M, N, P_M, P_N)$
määrittelee vakaan avioliiton ongelman.

Nyt injektiota $\mu : M \rightarrow N$ joka määrää jokaiselle
miehelle $a$ naisen $\mu(a)$ kutsutaan \emph{sijoitteluksi}.
Vakaan avioliiton ongelman ratkaiseminen tarkoittaa funktion $\mu$
määräämistä niin että se on vakaa. Sijoittelu on \emph{epävakaa}, jos
joillakin $x, y \in M$ ja $a, b \in N$ pätee, että $\mu(x) = a$
ja $\mu(y) = b$ mutta $aP_yb$ ja $yP_ax$. Siis mies $y$ pitää
enemmän naisesta $a$ kuin paristaan $b$, ja nainen $a$ pitää
enemmän miehestä $y$ kuin paristaan $x$. Sijoittelua voitaisiin
siis parantaa $y$:n ja $a$:n kannalta määräämällä $\mu(y) = a$.

Toinen sijoittelun hyvyyttä kuvaava ominaisuus on optimaalisuus.
Sijoittelu $\mu$ on \emph{optimaalinen} jos kaikilla $x \in M$ ja $a
\in N$ pätee, että heidän tilanteensa on ainakin yhtä hyvä kuin
missään muussa vakaassa sijoittelussa. \textbf{Tälle formaalimpi
  esitys.}

\section{Ratkaisu ajassa $\boldsymbol{\Theta(n^2)}$}

Jokaiseen vakaan avioliiton ongelmaan on olemassa vakaa ratkaisu
\cite{galeshapley62}. Gale ja Shapley todistavat tämän antamalla
seuraavanlaisen algoritmin vakaan sijoittelun löytämiseen.
Ensimmäisellä kierroksella jokainen mies kosii naista josta hän pitää
eniten. Naiset valitsevat alustavasti mahdollisista kosijoistaan sen
josta he pitävät eniten. Seuraavilla kierroksilla valitsematta jääneet
miehet kosivat listallaan seuraavana olevia naisia, ja naiset
valitsevat kosijoistaan ja mahdollisesta alustavasti valitusta
miehestä jälleen sen josta he pitävät eniten. Näin jatketaan kunnes
jokaista naista on kosittu ainakin kerran, jolloin muodostuneet parit
voidaan vahvistaa.

Jos jokaista naista on kosittu ainakin kerran, ei yksikään mies ole
tullut enää torjutuksi. Jos joku mies olisi tullut torjutuksi, olisi
joku nainen saanut useamman kuin yhden kosinnan, jolloin joku naisista
olisi jäänyt ilman kosintaa. Siispä menetelmä on määrännyt parin $(a,
\mu(a))$ kaikille $a \in M$ kun jokaista naista on kosittu ainakin
kerran.

Menetelmän antama sijoittelu on vakaa \cite{galeshapley62}. Oletetaan,
että saatu sijoittelu on epävakaa ja on siis olemassa $x \in M$ ja $a
\in N$ joilla $\mu(x) \neq a$, mutta $aP_x\mu(x)$ ja
$xP_a\mu^{-1}(a)$. Koska $x$ pitää enemmän naisesta $a$ kuin
paristaan, on hänen täytynyt kosia naista $a$ ennen pariaan. Koska $x$
ja $a$ eivät kuitenkaan ole pari, on naisen $a$ täytynyt torjua
kosinta. Nainen $a$ pitää siis enemmän paristaan kuin miehestä $x$.
Tämä on kuitenkin ristiriita. Siis saatu sijoittelu on vakaa.

Galen ja Shapley algoritmi ei vaadi, että naisten ja miesten joukot
ovat yhtämahtavat. Mikäli miesten joukon mahtavuus on $a$, naisten $b$
ja $a < b$, päättyy algoritmi kun $a$ kappaletta naisia on saanut
kosinnan. Vastaavasti jos $b < a$, päättyy algoritmi kun jokainen mies
on tullut alustavasti valituksi tai jokaisen naisen torjumaksi.
\cite{galeshapley62}

\subsection{Yläraja $\boldsymbol{O(n^2)}$}

Gale ja Shapley kirjoittavat algoritmin päättyvän korkeintaan $n^2 -
2n + 2$ kierroksen jälkeen \cite{galeshapley62}. Asia esitetään
kuitenkin vain sivuhuomautuksena, eivätkä kirjoittajat ota kantaa
algoritmin varsinaiseen aikavaativuuteen. Kierrosten määrä ei anna
suoraan aihetta odottaa $O(n^2)$ aikavaativuutta, sillä yhden
kierroksen aikana voi tapahtua useita kosintoja. Ei siis ole selvää,
että yksi kierros on mahdollista suorittaa vakioajassa.

Gusfield ja Irving esittävät algoritmille muotoilun \cite{gusfield89}
jonka analysointi on alkuperäistä helpompaa. Siinä yhden kierroksen
aikana yksi mies kosii naista josta pitää eniten ja jota hän ei ole
vielä kosinut. Gusfiel ja Irving osoittavat että yksi kosinta voidaan
suorittaa vakioajassa \cite{gusfield89}. Nyt riittää tarkastella
kosintojen määrää joka tapauksessa $|M| = |N| = n$ on korkeintaan
$n^2$, sillä jokainen mies kosii jokaista naista korkeintaan kerran
\cite{gusfield89}. Tästä seuraa algoritmin suorituksen asymptoottinen
yläraja $O(n^2)$.

\subsection{Alaraja $\boldsymbol{\Omega(n^2)}$}

Ng osoittaa, että Galen ja Shapleyn algoritmi on asymptoottiselta
aikavaativuudeltaan optimaalinen \cite{cheng89}. Ngin päätuloksen
mukaan satunnaisesta mies-nainen -parista voidaan päätellä onko se
vakaa annetussa ongelmassa ajassa $\Omega(n^2)$. Tästä seuraa, että
vakaan avioliiton ongelman ratkaiseminen on asymptoottiselta
aikavaativuudeltaan $\Theta(n^2)$ \cite{cheng89}. Jos jokin algoritmi
löytäisi vakaan sijoittelun alle $\Theta(n^2)$ ajassa, päättelisi se
myös sijoitteluun kuuluvien parien vakauden alle $\Omega(n^2)$ ajassa
mikä olisi ristiriita. Toisin kuin Gusfieldin ja Irvinig analyysi
algoritmin ylärajasta, Ng olettaa, että naisten ja miesten joukot ovat
yhtämahtavat.

Ng analyysi tarkastelee sitä kuinka monta kertaa mikä tahansa vakaan
avioliiton ongelmaa ratkaiseva algorimi joutuu tarkastelemaan
syötteenään saatua ongelman kuvausta. Ng aloittaa kanonisesta
ongelmasta jossa tutkittava pari on vakaa. Sitten hän esittää
menetelmän jolla voidaan muodostaa tästä ongelmasta muunnelmia joissa
tutkittava pari ei ole vakaa. Nyt tunnistaakseen onko pari vakaa
syötteenä annetussa ongelmassa, tulee algoritmin kyetä erottamaan
kanoninen ongelma siitä luoduista muunnelmista. Ng osoittaa tähän
tarvittavan ainakin $\Omega(n^2)$ ongelman kuvauksen tarkastelua.
\cite{cheng89}

\section{Osittaiset järjestysrelaatiot}

Galen ja Shapleyn määrittelemässä vakaan avioliiton ongelmassa
molemmat sukupuolet määrittelevät kiinnostuksensa kaikista
vastakkaisen sukupuolen henkilöistä. Käytännön sovelluksissa tämä ei
kuitenkaan ole aina haluttua. Esimerkiksi oppilaitosten valinnoissa
hakijat eivät ole halukkaita opiskelemaan missä tahansa
oppilaitoksessa eivätkä oppilaitokset halua ottaa vastaan ketä tahansa
hakijaa.

Näin syntyy muunnelma vakaan avioliiton ongelmasta jossa sukupuolten
mieltymyksiä kuvaavien järjestysrelaatioiden ei tarvitse olla
täydellisiä. Ne eivät siis välttämättä anna järjestystä kaikille
vastakkaisen sukupuolen henkilöille. Lisäksi sijoittelulle $\mu$ tulee
päteä, että jos $x \in M$, $a \in N$ ja $\mu(x) = a$ niin $(\exists b
\in N: (a, b) \in P_x$ tai $(b, a) \in P_x)$ ja $(\exists y \in M: (x,
y) \in P_a$ tai $(y, x) \in P_a)$. Siis jos mies $x$ ja nainen $a$
ovat pari, ovat nainen ja mies esittäneet mieltymyksensä toisistaan.

\begin{thebibliography}{XXX99}

\bibitem[GaS62]{galeshapley62}
  D. Gale, L. S. Shapley.
  College Admissions and the Stability of Marriage.
  \emph{The American Mathematical Monthly, 69, 1 (1962)}

\bibitem[MII02]{manlove02}
  David F. Manlove, Robert W. Irving, Kazuo Iwama, Shuichi Miyazaki,
  Yasufumi Morita.
  Hard variants of stable marriage.
  \emph{Theoretical Computer Science, 276, (2002), sivut 261-279}

\bibitem[GuI89]{gusfield89}
  D. Gusfield, Robert W. Irving.
  The Stable Marriage Problem: Structure and Algorithms.
  \emph{MIT Press, Cambridge, MA, 1989}

\bibitem[Ito78]{itoga78}
  Stephen Y. Itoga.
  The Upper Bound for the Stable Marriage Problem.
  \emph{The Journal of the Operational Research Society, 29, 8 (1978)}

\bibitem[Che89]{cheng89}
  Cheng Ng.
  Lower Bounds for the Stable Marriage Problem and its Variants.
  \emph{30th Annual Symposium on Foundations of Computer Science,
    (1989), sivut 129-133}

\end{thebibliography}
\end{document}
