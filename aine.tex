\documentclass[gradu, twoside]{tktltiki}
\usepackage{amsmath}
\usepackage{url}
\begin{document}

\title{Vakaan avioliiton ongelman ja sen muunnelmien aikavaativuuksia}
\author{Jani Rahkola}
\date{\today}
\level{Aine}
\numberofpagesinformation{\numberofpages sivua}

\classification{A.1 [Introductory and Survey] \\ G.2.1 [Discrete
    Mathematics]: Combinatorics---Combinatorial algorithms \\ F.2.2
  [Analysis of Algorithms and Problems]: Nonnumerical Algorithms and
  Problems---Computations on Discrete Structures}

\keywords{vakaan avioliiton ongelma, aikavaativuus}

\maketitle

\doublespacing

\faculty{Matemaattis-luonnontieteellinen}
\department{Tietojenkäsittelytieteen laitos}
\subject{Tietojenkäsittelytiede}

\begin{abstract}
  Vakaan avioliiton ongelmassa miesten joukon $M$ ja naisten joukon
  $N$ välille pyritään löytämään relaatio $\mu \subset M \times N$
  noudattaen miesten ja naisten mieltymyksiä kuvaavia täydellisiä
  järjestysrelaatioita $P_M = \{P_m \subset N \times N | m \in M\}$ ja
  $P_N = \{P_n \subset M \times M | n \in N\}$. Gale ja Shapley
  antavat ongelman esittelyn yhteydessä algoritmin joka ratkaisee
  ongelman ajassa $O(n^2)$. Ng osoittaa, että Galen ja Shapleyn
  algoritmi on suoritusajaltaan optimaalinen.

  Käytännön kannalta on usein tarpeellista muokata ongelmanasettelua
  niin, että järjestysrelaatioiden ei tarvitse kattaa koko vastakkaisen
  sukupuolen joukkoa, tai niissä on luvallista olla tasatilanteita.
  Tässä aineessa käydään läpi vakaan avioliiton ongelman määrittely ja
  analyysi perusongelman ja mainittujen kahden muunnelman suoritusajalle.
\end{abstract}

\mytableofcontents

\section{Johdanto}

Gale ja Shapley määrittelevät vakaan avioliiton ongelman (stable
marriage problem) artikkelissaan College Admissions and the Stability
of Marriage \cite{galeshapley62} jonka motiivina on yliopistojen
valintaongelma. Vakaan avioliiton ongelma on yksinkertaistettu
tilanne, jossa joukosta naisia ja miehiä pyritään muodostamaan pareja
noudattaen sekä miesten että naisten mieltymyksiä. Hyväksyttävien
ratkaisujen joukoksi he määrittelevät ne ratkaisut jotka täyttävät
kaksi ehtoa. Ensinnäkin ratkaisun tulee olla vakaa (stable). Kenenkään
kahden parin kohdalla ei saa olla niin, että miesten (tai naisten)
vaihtaminen parien kesken tekisi ainakin toisesta uudesta parista sen
jäsenille mieluisemman. Toiseksi ratkaisun tulee olla optimaalinen.
Jokaisen miehen ja naisen tulee saada ratkaisussa vähintään yhtä hyvä
pari kuin missään muussa mahdollisessa ratkaisussa. Tämä
optimaalisuuden käsite vastaa heikon Pareto-optimaalisuuden käsitettä
\cite{gusfield89}.

Tämän yksinkertaistetun ongelmaan ratkaisemiseen Gale ja Shapley
antavat algoritmin jonka on myöhemmin osoitettu olevan
suoritusajaltaan optimaalinen \cite{gusfield89, cheng89}.
Päätösluvussaan Gale ja Shapley kuitenkin toteavat luomansa vakaan
avioliiton ongelman olevan hyvin teoreettinen ja kaukana
alkuperäisestä yliopistojen valintaongelmasta. Vakaan avioliiton
ongelma on kuitenkin erikoistapaus yliopistojen (myöhemmin
oppilaitosten) valintaongelmasta tilanteessa jossa jokaisella
oppilaitoksella on täytettävänä tasan yksi paikka \cite{manlove02}.
Niinpä vakaan avioliiton ongelmaa tutkimalla saadaan asetettua rajoja
myös oppilaitosten valintaongelmalle.

Vakaan avioliiton ongelmalle on myös suoria sovelluksia. Esimerkiksi
työpaikoilla tehtävä toimistojen jako on palautettavissa suoraan
vakaan avioliiton ongelmaan. Asemaan tai kokemukseen perustuva
ensisijaisuusjärjestelmä voidaan mallintaa toimistojen
``mieltymysten'' avulla. Tällaisessa järjestelmässä ei useinkaan ole
mielekästä vaatia työntekijöitä asettamaan kaikkia toimistoja tiukkaan
paremmuusjärjestykseen. Jos tarjolla on vain avokonttorin työpisteitä,
ei sijainnilla ole välttämättä väliä. Näin syntyy muunnelma vakaan
avioliiton ongelmasta, joka kuitenkin osoittautuu ratkeavaksi Galen ja
Shapleyn algoritmilla optimiajassa.

Kolmas sovellus vakaan avioliiton ongelmalle on elinluovutusten
järjestäminen. Luovuttajista ja vastaanottajista pyritään muodostamaan
mahdollisimman yhteensopivia pareja. Koska veriryhmä ja muut tekijät
sanelevat luovutuksen mahdollisuuden, on tarpeen muuntaa alkuperäistä
ongelmankuvausta niin, että mahdottomia pareja ei pääse syntymään.
Syntyy toinen muunnelma joka on ratkaistavissa pienellä muunnoksella
alkuperäiseen Galen ja Shapley algoritmiin.

Oppilaitosten valinnoissa usein yhdistyvät molemmat edellä esitellyt
muunnokset. Hakijoiden hakuvaihtoehtoja voidaan rajata, ja
oppilaitokset eivät välttämättä ole halukkaita muodostamaan tiukkaa
paremmuusjärjestystä kaikkien hakijoiden välille. Galen ja Shapleyn
alkuperäisen ongelman kannalta onkin merkittävää, että tämä kahden
muunoksen yhdistelmä on NP-täydellinen \cite{manlove02}.

Ensimmäisessä luvussa määritellään vakaan avioliiton ongelma ja
ratkaisun vakaus. Seuraavassa luvussa esitellään Galen ja Shapleyn
algoritmi ratkaisun löytämiseksi ja käydään läpi analyysi algoritmin
suoritusajasta. Luvuissa kolme, neljä ja viisi esitellään käytännön
kannalta kiinnostavia muunnelmia vakaan avioliiton ongelmasta.

\section{Vakaan avioliiton ongelma}

Määritellään Gale ja Shapley esittelemä \cite{galeshapley62} vakaan
avioliiton ongelma seuraavasti. Olkoon $M$ joukko miehiä ja $N$ joukko
naisia. Olkoon lisäksi $P_M = \{P_m \subset N \times N | m \in M\}$ ja
$P_N = \{P_n \subset M \times M | n \in N\}$ joukot täydellisiä
järjestysrelaatioita jotka kertovat mihin järjestykseen mies $m$
(vastaavasti nainen $n$) sijoittaa vastakkaisen sukupuolen edustajat.
Kutsutaan näitä relaatioita \emph{preferenssirelaatioiksi}. Jos siis
$m_1,m_2 \in M$ ja $m_1P_nm_2$ pitää nainen $n$ enemmän miehestä $m_1$
kuin $m_2$. Nyt nelikko $(M, N, P_M, P_N)$ määrittelee \emph{vakaan
  avioliiton ongelman }.

Yksi-yhteen relaatiota $\mu \subset M \times N$ kutsutaan
\emph{sijoitteluksi}. Vakaan avioliiton ongelman perusmuodossa
sijoittelun tulee olla bijektio. Siis jokaisella naisella ja miehellä
tulee olla pari. Merkitään miehen $m$ paria sijoittelussa $\mu$
$\mu(m)$ ja vastaavasti naisen $n$ paria $\mu^{-1}(n)$.

Vakaan avioliiton ongelman ratkaiseminen tarkoittaa sijoittelun $\mu$
määräämistä niin että se on vakaa. Sijoittelu on \emph{epävakaa}, jos
joillakin $m_1, m_2 \in M$ ja $n_1, n_2 \in N$ pätee, että $(m_1, n_1)
\in \mu$ ja $(m_2, n_2) \in \mu$ mutta $n_1P_{m_2}n_2$ ja
$m_2P_{n_1}m_1$. Siis mies $m_2$ pitää enemmän naisesta $n_1$ kuin
paristaan $n_2$, ja nainen $n_1$ pitää enemmän miehestä $m_2$ kuin
paristaan $m_1$. Sijoittelua voitaisiin siis parantaa $m_2$:n ja
$n_1$:n kannalta määräämällä $(m_2, n_1) \in \mu$.

\section{Ratkaisu ajassa $\boldsymbol{\Theta(n^2)}$}

Jokaiseen vakaan avioliiton ongelmaan on olemassa ratkaisu
\cite{galeshapley62}. Gale ja Shapley todistavat tämän antamalla
seuraavanlaisen algoritmin vakaan sijoittelun löytämiseen.
Ensimmäisellä kierroksella jokainen mies kosii naista josta hän pitää
eniten. Naiset valitsevat alustavasti mahdollisista kosijoistaan sen
josta he pitävät eniten. Seuraavilla kierroksilla valitsematta jääneet
miehet kosivat listallaan seuraavana olevia naisia, ja naiset
valitsevat kosijoistaan ja mahdollisesta alustavasti valitusta
miehestä jälleen sen josta he pitävät eniten. Näin jatketaan kunnes
jokaista naista on kosittu ainakin kerran, jolloin muodostuneet parit
voidaan vahvistaa.

Jos jokaista naista on kosittu ainakin kerran, ei yksikään mies ole
tullut enää torjutuksi. Jos joku mies olisi tullut torjutuksi, olisi
joku nainen saanut useamman kuin yhden kosinnan, jolloin joku naisista
olisi jäänyt ilman kosintaa. Siispä algoritmi on määrännyt parin
jokaiselle miehelle ja naiselle kun jokaista naista on kosittu ainakin
kerran.

Galen ja Shapleyn algoritmi ei vaadi, että naisten ja miesten joukot
ovat yhtämahtavat. Mikäli $|M| < |N|$ päättyy algoritmi kun $|M|$
kappaletta naisia on saanut kosinnan. Vastaavasti jos $|N| < |M|$,
päättyy algoritmi kun jokainen mies on tullut alustavasti valituksi
tai jokaisen naisen torjumaksi. \cite{galeshapley62}

\subsection{Sijoittelun vakaus ja Pareto-optimaalisuus}

Algoritmin antama sijoittelu on vakaa \cite{galeshapley62}. Oletetaan,
että saatu sijoittelu on epävakaa ja on siis olemassa $m \in M$ ja $n
\in N$ joilla $(m, n) \notin \mu$, mutta $nP_m\mu(m)$ ja
$mP_n\mu^{-1}(n)$. Koska $m$ pitää enemmän naisesta $n$ kuin
paristaan, on hänen täytynyt kosia naista $n$ ennen pariaan. Koska $m$
ja $n$ eivät kuitenkaan ole pari, on naisen $n$ täytynyt torjua
kosinta. Nainen $n$ pitää siis enemmän paristaan kuin miehestä $m$.
Tämä on kuitenkin ristiriita. Siis saatu sijoittelu on vakaa.

Galen ja Shapleyn algoritmin antama sijoittelu $\mu$ on miesten
kannalta \emph{heikosti Pareto-optimaalinen} \cite{gusfield89}. Ei
siis ole olemassa toista sijoittelua $\gamma$ jossa kaikilla $(m, n)
\in \gamma$ pätee, että $nP_m\mu(m)$. Vastaavasti voidaan määritellä,
että sijoittelu on heikosti Pareto-optimaalinen naisten kannalta.
Luonnollista on, että algoritmi antaa naisoptimaalisen sijoittelun,
kun kosijoiden rooli annetaan naisille.

\subsection{Yläraja $\boldsymbol{O(n^2)}$}

Gale ja Shapley kirjoittavat algoritmin päättyvän korkeintaan $n^2 -
2n + 2$ kierroksen jälkeen \cite{galeshapley62}. Asia esitetään
kuitenkin vain sivuhuomautuksena, eivätkä kirjoittajat ota kantaa
algoritmin varsinaiseen aikavaativuuteen. Kierrosten määrä ei anna
suoraan aihetta odottaa $O(n^2)$ aikavaativuutta, sillä yhden
kierroksen aikana voi tapahtua useita kosintoja. Ei siis ole selvää,
että yksi kierros on mahdollista suorittaa
vakioajassa. \enlargethispage{\baselineskip}

Gusfield ja Irving esittävät algoritmille muotoilun \cite{gusfield89}
jonka analysointi on alkuperäistä helpompaa. Siinä yhden kierroksen
aikana yksi mies kosii naista josta pitää eniten ja jota hän ei ole
vielä kosinut. Gusfiel ja Irving osoittavat että yksi kosinta voidaan
suorittaa vakioajassa \cite{gusfield89}. Nyt riittää tarkastella
kosintojen määrää joka tapauksessa $|M| = |N| = n$ on korkeintaan
$n^2$, sillä jokainen mies kosii jokaista naista korkeintaan kerran
\cite{gusfield89}. Tästä seuraa algoritmin suorituksen asymptoottinen
yläraja $O(n^2)$.

\subsection{Alaraja $\boldsymbol{\Omega(n^2)}$}

Koska kummankin sukupuolen preferenssirelaatioiden esittäminen vaatii
muistia $O(n^2)$ määrän, vie syötteen luku itsessään jo $O(n^2)$ ajan.
Niinpä alarajan tarkastelussa on mielekästä tutkia tilannetta jossa
algoritmin syöte on jo valmiiksi luettu muistiin \cite{cheng89}.

Ng osoittaa, että Galen ja Shapleyn algoritmi on asymptoottiselta
aikavaativuudeltaan optimaalinen \cite{cheng89}. Ngin päätuloksen
mukaan satunnaisesta mies-nainen -parista voidaan päätellä onko se
vakaa annetussa ongelmassa ajassa $\Omega(n^2)$. Tästä seuraa, että
vakaan avioliiton ongelman ratkaiseminen on asymptoottiselta
aikavaativuudeltaan $\Theta(n^2)$ \cite{cheng89}. Jos jokin algoritmi
löytäisi vakaan sijoittelun alle $\Theta(n^2)$ ajassa, päättelisi se
myös sijoitteluun kuuluvien parien vakauden alle $\Omega(n^2)$ ajassa
mikä olisi ristiriita. Toisin kuin Gusfieldin ja Irvinig analyysi
algoritmin ylärajasta, Ng olettaa, että naisten ja miesten joukot ovat
yhtämahtavat.

Ng analyysi tarkastelee sitä kuinka monta kertaa mikä tahansa vakaan
avioliiton ongelmaa ratkaiseva algorimi joutuu tarkastelemaan
syötteenään saatua ongelman kuvausta. Ng aloittaa kanonisesta
ongelmasta jossa tutkittava pari on vakaa. Sitten hän esittää
menetelmän jolla voidaan muodostaa tästä ongelmasta muunnelmia joissa
tutkittava pari ei ole vakaa. Nyt tunnistaakseen onko pari vakaa
syötteenä annetussa ongelmassa, tulee algoritmin kyetä erottamaan
kanoninen ongelma siitä luoduista muunnelmista. Ng osoittaa tähän
tarvittavan ainakin $\Omega(n^2)$ ongelman kuvauksen tarkastelua.
\cite{cheng89}

\section{Perusongelman muunnelmia}

Galen ja Shapleyn määrittelemässä vakaan avioliiton ongelmassa
molemmat sukupuolet määrittelevät kiinnostuksensa kaikista
vastakkaisen sukupuolen henkilöistä. Käytännön sovelluksissa tämä ei
kuitenkaan ole aina haluttua. Esimerkiksi oppilaitosten valinnoissa
hakijat eivät ole halukkaita opiskelemaan missä tahansa
oppilaitoksessa eivätkä oppilaitokset halua ottaa vastaan ketä tahansa
hakijaa. Näin syntyy muunnelma jossa toimijan ei tarvitse suostua
muodostamaan paria kenen tahansa vastakkaisen osapuolen edustajan
kanssa.

Monissa käytännön sovelluksissa ei ole aina haluttua määritellä
järjestystä kaikkien vastakkaisen osapuolen toimijoiden välille.
Esimerkiksi oppilaitosten valinnoissa jonkin rajan alle jääneistä
hakijoista valinta saatetaan tehdä esimerkiksi arpomalla. Jaettaessa
toimistoja työtekijöille voi työntekijällä olla vain muutama
vaihtoehto joihin hän erityisesti haluaa. Erona ensiksi esiteltyyn
muunnelmaan on kuitenkin se, että toimijat ovat valmiita muodostamaan
parin kenen tahansa vastapuolen toimijan kanssa.

Oppilaitosten valinnoissa on mielekästä tutkia myös kahden edellä
esitellyn muunnelman yhdistelmää. Esimerkiksi tietyn rajan
ylittäneet hakijat olisivat listan kärkipäässä, jonkin toisen rajan
alittaneet eivät voisi missään tapauksessa tulla valituiksi
oppilaitokseen, ja väliin jääneiden tapauksessa valinta kahden hakijan
välillä ratkaistaisiin aina arvalla.

\subsection{Preferenssirelaatiot osajoukoille}

Jos miesten ja naisten ei tarvitse suostua muodostamaan paria jokaisen
vastakkaisen sukupuolen edustajan kanssa, syntyy muunnelma vakaan
avioliiton ongelmasta jossa preferenssirelaatioiden ei tarvitse olla
määritelty koko vastakkaisen sukupuolen joukon yli. Merkitään $N_m$
(vastaavasti $M_n$) sitä joukkoa naisia, joiden kanssa mies $m$
(vastaavasti nainen $n$) on valmis muodostamaan parin. Nyt $P_m
\subset N_m \times N_m$ ja $P_n \subset M_n \times M_n$.

 Lisäksi sijoittelulle $\mu$ tulee päteä, että jos $m \in M$, $n \in
 N$ ja $(m, n) \in \mu$ niin $n \in N_m$ ja $m \in M_n$. Siis jos mies
 $m$ ja nainen $n$ ovat pari, ovat he olleet valmiita muodostamaan
 parin keskenään.

Tästä seuraa, että saatu sijoittelu voi olla osittainen sillä jokainen
mies ja nainen eivät välttämättä saa itselleen paria. Tästä syystä on
tarpeen muotoilla heikompi versio vakaudesta. Sijoittelu on
\emph{heikosti epävakaa}, jos on olemassa $m_1 \in M$ ja $n_1 \in N$
joilla seuraavat ehdot ovat voimassa \cite{gusfield89}:
\[
\text{1. }(m_1, n_1) \notin \mu \text{, mutta } m_1 \in M_{n_1} \text{
  ja } n_1 \in N_{m_1}
\]
\[
\text{2. jos }\exists n_2 \in N: (m, n_2) \in \mu \text{, niin
}n_1P_{m_1}n_2
\]
\[
\text{3. jos }\exists m_2 \in M: (m_2, n) \in \mu \text{, niin
}m_1P_{n_1}m_2
\]

Myös Pareto-optimaalisuus voidaan laajentaa koskemaan osajoukoille
määriteltyjä preferenssirelaatioita käyttäen saatua sijoittelua.
Jokainen mies pitää parempana sijoittelua jossa hänellä on pari, kuin
sijoittelua jossa hänellä ei ole paria. \cite{gusfield89}

Esitellyn muunnelman ratkaisemiseksi täytyy myös Galen ja Shapleyn
algoritmiin tehdä muunnos. Jokaisen miehen ja naisen
preferenssirelaatiot esikäsitellään seuraavasti. Kaikille $m \in M$ ja
$n \in N_m$ tarkistetaan onko $m \in M_{n}$. Jos näin ei ole,
poistetaan $n$ $N_m$:stä. \cite{gusfield89} Vastaava tehdään kaikille
$n \in N$. Lopuksi määritellään preferenssirelaatiot uudelleen. Siis
jokaisen miehen ja naisen preferenssirelaatiosta poistetaan ne parit
jotka koskevat sellaista vastakkaisen sukupuolen edustajaa, joka ei
halua muodostaa paria ko. miehen/naisen kanssa. Tämä esikäsittely
voidaan tehdä ajassa $O(n^2)$ kun henkilön kuuluminen toisen henkilön
preferenssirelaatioon voidaan tarkistaa vakioajassa. Tämä voidaan
taata valitsemalla preferenssirelaatioiden esitys sopivasti
\cite{gusfield89}. Nyt muunnettu ongelma voidaan ratkaista Galen ja
Shapleyn algoritmilla kunhan vain tehdään mahdolliseksi algoritmin
pysähtyminen vaikka kaikilla miehillä ja naisilla ei olisi paria.

\subsection{Osittaiset preferenssirelaatiot}

Jos henkilön ei tarvitse määrittää paremmuusjärjestystä kaikkien
vastakkaisen sukupuolen edustajien välille, saadaan muunnelma jossa
preferenssirelaatiot ovat osittaisia. Voi siis olla, että jos $m \in
M$, $n_1,n_2 \in N$ niin $(n_1, n_2) \notin P_m$ ja $(n_2, n_1) \notin
P_m$. Muunnelman mukaiset ongelmat voidaan kuitenkin ratkaista
alkuperäistä Galen ja Shapleyn algoritmia käyttäen, jos esitetyn
kaltaiset tasatilanteet ratkaistaan satunnaisesti \cite{gusfield89}.

Myös tässä muunnelmassa on tarpeen määritellä heikompi versio
vakaudesta. Esimerkiksi täydellisen väliinpitämättömyyden vallitessa
mikään sijoittelu ei olisi vakaa, sillä jokaisen kahden parin
tapauksessa henkilöille olisi mieluista muodostaa parit vaihtamalla
miehiä (tai naisia). Olkoon siis osittaisten preferenssirelaatioiden
tapauksessa sijoittelu $\mu$ \emph{heikosti epävakaa}, jos on olemassa
$m \in M$ ja $n \in N$ joilla $(m, n) \notin \mu$, mutta $(n, \mu(m))
\in P_m$ ja $(m, \mu^{-1}(n)) \in P_n$. Siis sijoittelu on heikosti
epävakaa, jos on olemassa yksi nainen ja mies, jotka pitävät aidosti
enemmän toisistaan kuin sijoittelun antamasta paristaan.
\cite{gusfield89}

Käytännön ongelmissa tasatilanteiden ratkaiseminen arvalla on usein
hyväksyttävää. Siitä seuraa kuitenkin, että algoritmi saattaa antaa
eri suorituskerroilla eri tuloksen. Kaikki algoritmin antamat
sijoittelut ovat kuitenkin edelleen heikosti Pareto-optimaalisia. Ei
siis ole olemassa sellaista sijoittelua jossa jokainen mies pitäisi
aidosti enemmän paristaan kuin missään algoritmin tuottamassa
sijoittelussa.

\subsection{Osittaiset preferenssirelaatiot osajoukoille}

Kun yhdistetään kaksi edellä esitettyä muunnelmaa, saadaan muunnelma
jossa preferenssirelaatiot ovat sekä osittaisia että määritelty vain
jollekin vastakkaisen sukupuolen joukon osajoukolle. Tässä
muunnelmassa ensimmäinen merkittävä huomio on, että kaikki mahdolliset
vakaat sijoittelut eivät välttämättä ole saman kokoisia. Tämä voidaan
todeta yksinkertaisen esimerkin avulla. Olkoon $m_1: n_1$, $m_2: n_1
n_2$, $n_1: (m_1 m_2)$ ja $n_2: m_2$. Tässä siis miellyttävin henkilö
on listassa vasemmanpuolimmaisena ja sulut kuvaavat tasatilannetta.
Nyt $\{(m_2, n_1)\}$ ja $\{(m_1, n_1), (m_2, n_2)\}$ ovat vakaita
sijoitteluja. \cite{manlove02}

Koska vakaat sijoittelut voivat olla eri kokoisia, on mielenkiintoista
pyrkiä löytämään suurimmat tai pienimmät vakaat sijoittelut. Molemmat
näistä ongelmista ovat kuitenkin NP-täydellisiä jopa erittäin
rajoitetussa tapauksessa jossa tasatilanteet ovat sallittuja vain
kahden vähiten miellyttävimmän henkilön välillä. Siis jokaisessa
preferenssirelaatiossa voi olla korkeintaan yksi tasatilanne. Manlove
ja kumppanit todistavat tämän tekemällä reduktion suurimman vakaan
sijoittelun löytämisestä exact maximal matching -ongelmaan ja
pienimmän vakaan sijoittelun löytämisestä minimum maximal matching
ongelmaan. \cite{manlove02}

\section{Yhteenveto}

Edellä on esitelty vakaan avioliiton ongelma ja sen ratkaisun
aikavaativuuksia Galen ja Shapleyn esittelemälle muodolle sekä
kolmelle käytännön kannalta merkittävälle muunnelmalle.
Ei-refleksiiviset ja ei-antisymmetriset preferenssirelaatiot
yhdistävän ongelman on osoitettu olevan NP-täydellinen. Tämä on
tarpeellista ottaa huomioon monissa vakaan avioliiton ongelman
yleistyksissä, kuten oppilaitosten valinnassa. Käytännön sovelluksissa
ainakin toisen osapuolen toimijoiden määrä on usein erittäin suuri,
esimerkiksi Suomen toisen asteen oppilaitosten yhteishaussa haki
vuoden 2011 keväällä yli 102 000 hakijaa \cite{OPH12}.

\begin{thebibliography}{XXX99}

\bibitem[GaS62]{galeshapley62}
  D. Gale, L. S. Shapley.
  College Admissions and the Stability of Marriage.
  \emph{The American Mathematical Monthly, 69, 1 (1962)}

\bibitem[MII02]{manlove02}
  David F. Manlove, Robert W. Irving, Kazuo Iwama, Shuichi Miyazaki,
  Yasufumi Morita.
  Hard variants of stable marriage.
  \emph{Theoretical Computer Science, 276, (2002), sivut 261-279}

\bibitem[GuI89]{gusfield89}
  D. Gusfield, Robert W. Irving.
  The Stable Marriage Problem: Structure and Algorithms.
  \emph{MIT Press, Cambridge, MA, 1989}

\bibitem[Che89]{cheng89}
  Cheng Ng.
  Lower Bounds for the Stable Marriage Problem and its Variants.
  \emph{30th Annual Symposium on Foundations of Computer Science,
    (1989), sivut 129-133}

\bibitem[OPH12]{OPH12}
  Opetushallitus.
  Aloituspaikat, hakijat, hyväksytyt ja paikan vastaanottaneet
  kevään 2011 ammatillisen ja lukiokoulutuksen yhteishaussa.
  \url{https://www.kouluta.fi/koulutadw/faces/app/startupDWReports.jspx?report_id=18}
      [3.3.2012]

\end{thebibliography}
\end{document}
