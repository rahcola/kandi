\documentclass[gradu, twoside]{tktltiki}
\usepackage{amsmath}
\begin{document}

\title{}
\author{Jani Rahkola}
\date{\today}
\level{}

\maketitle

\doublespacing

\faculty{Matemaattis-luonnontieteellinen}
\department{Tietojenkäsittelytieteen laitos}
\subject{Tietojenkäsittelytiede}

\mytableofcontents

\section{Vakaan avioliiton ongelma}

Määritellään vakaan avioliiton ongelma seuraavasti. Olkoon $A$
joukko naisia ja $B$ joukko miehiä. Olkoon lisäksi $P_{a \in A}
\subset B \times B$ ja $P_{b \in B} \subset A \times A$
järjestysrelaatioita jotka kertovat mihin järjestykseen mies $a$
(vastaavasti nainen $b$) sijoittaa vastakkaisen sukupuolen
edustajat. Jos siis $x,y \in A$ ja $xP_by$ pitää nainen $b$
enemmän miehestä $x$ kuin $y$.

Nyt injektiota $\mu : A \rightarrow B$ joka määrää jokaiselle
miehelle $a$ naisen $\mu(a)$ kutsutaan \emph{sijoitteluksi}.
Vakaan avioliiton ongelman ratkaiseminen tarkoittaa funktion $\mu$
määräämistä niin että se on vakaa. Sijoittelu on \emph{epävakaa}, jos
joillakin $x, y \in A$ ja $a, b \in B$ pätee, että $\mu(x) = a$
ja $\mu(y) = b$ mutta $aP_yb$ ja $yP_ax$. Siis mies $y$ pitää
enemmän naisesta $a$ kuin paristaan $b$, ja nainen $a$ pitää
enemmän miehestä $y$ kuin paristaan $x$. Sijoittelua voitaisiin
siis parantaa $y$:n ja $a$:n kannalta määräämällä $\mu(y) = a$.

\section{Ratkaisu ajassa $\boldsymbol{O(n^2)}$}

Jokaiseen ongelmaan on olemassa vakaa ratkaisu \cite{galeshapley62}.
Gale ja Shapley todistavat tämän antamalla seuraavanlaisen menetelmän
vakaan sijoittelun löytämiseen. Ensimmäisellä kierroksella jokainen
mies kosii naista josta hän pitää eniten. Naiset valitsevat
alustavasti mahdollisista kosijoistaan sen josta he pitävät eniten.
Seuraavilla kierroksilla valitsematta jääneet miehet kosivat
listallaan seuraavana olevia naisia, ja naiset valitsevat kosijoistaan
ja mahdollisesta alustavasti valitusta miehestä jälleen sen josta he
pitävät eniten. Näin jatketaan kunnes jokaista naista on kosittu
ainakin kerran, jolloin muodostuneet parit voidaan vahvistaa.

Jos jokaista naista on kosittu ainakin kerran, ei yksikään mies ole
tullut enää torjutuksi. Jos joku mies olisi tullut torjutuksi, olisi
joku nainen saanut useamman kuin yhden kosinnan, jolloin joku naisista
olisi jäänyt ilman kosintaa. Siispä menetelmä on määrännyt parin $(a,
\mu(a))$ kaikille $a \in A$ kun jokaista naista on kosittu ainakin
kerran.

Menetelmän antama sijoittelu on vakaa \cite{galeshapley62}. Oletetaan,
että saatu sijoittelu on epävakaa ja on siis olemassa $x \in A$ ja $a
\in B$ joilla $\mu(x) \neq a$, mutta $aP_x\mu(x)$ ja
$xP_a\mu^{-1}(a)$. Koska $x$ pitää enemmän naisesta $a$ kuin
paristaan, on hänen täytynyt kosia naista $a$ ennen pariaan. Koska $x$
ja $a$ eivät kuitenkaan ole pari, on naisen $a$ täytynyt torjua
kosinta. Nainen $a$ pitää siis enemmän paristaan kuin miehestä $x$.
Tämä on kuitenkin ristiriita. Siis saatu sijoittelu on vakaa.



\begin{thebibliography}{XXX99}

\bibitem[GaS62]{galeshapley62}
  D. Gale, L. S. Shapley.
  College Admissions and the Stability of Marriage.
  \emph{The American Mathematical Monthly, 69, 1 (1962)}

\bibitem[Ito78]{itoga78}
  Stephen Y. Itoga.
  The Upper Bound for the Stable Marriage Problem.
  \emph{The Journal of the Operational Research Society, 29, 8 (1978)}

\end{thebibliography}
\end{document}
