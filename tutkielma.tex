\documentclass[gradu, twoside]{tktltiki}
\usepackage{amsmath}
\usepackage{url}
\usepackage{mdwlist}
\begin{document}

\title{Kolme algoritmia opiskelijavalintaongelman ratkaisemiseen}
\author{Jani Rahkola}
\date{\today}
\level{LUK-tutkielma}
\numberofpagesinformation{}

\classification{A.1 [Introductory and Survey] \\ G.2.1 [Discrete
    Mathematics]: Combinatorics---Combinatorial algorithms \\ F.2.2
  [Analysis of Algorithms and Problems]: Nonnumerical Algorithms and
  Problems---Computations on Discrete Structures}

\keywords{vakaan avioliiton ongelma, opiskelijavalinta, aikavaativuus}

\maketitle

\doublespacing

\faculty{Matemaattis-luonnontieteellinen}
\department{Tietojenkäsittelytieteen laitos}
\subject{Tietojenkäsittelytiede}

\def\hyph{-\penalty0\hskip0pt\relax}
\hyphenation{}

\begin{abstract}
Opiskelijavalinta ongelma on tilanne, jossa useat oppilaitokset
kilpailevat keskenään hakijoista. Sekä hakijat että oppilaitokset
asettavat paremmuusjärjestykseen ne vastapuolen hakijat joista he ovat
kiinnostuneita. Opiskelijavalintaongelman ratkaisemiseksi algoritmin
tulee löytää sijoittelu joka määrää yksikäsitteisiä pareja hakijoista
ja oppilaitoksista halutuin ominaisuuksin. Tässä tutkielmassa
esitellään Galen ja Shapleyn algoritmi, Bostonin mekanismi ja top
trading cycles algoritmi, ja vertaillaan niitä vakauden,
optimaalisuuden, Pareto\hyph optimaalisuuden ja strategian kestävyyden
käsitteillä.
\end{abstract}

\mytableofcontents

\section{Johdanto}

Opiskelijavalintaongelma on tilanne, jossa useat oppilaitokset
kilpailevat keskenään hakijoista. Vastaavasti hakijat saattavat
kilpailla pääsystä haluttuihin oppilaitoksiin. Kilpailun synnyttää
oppilaitosten rajattu opiskelupaikkojen määrä. Hakijoiden
mahdollisuuksia puolestaan rajoittavat menestyminen valintakokeissa ja
mahdolliset valintajärjestelmän asettamat rajoitukset hakukohteiden
määrälle.

Sekä oppilaitokset että hakijat priorisoivat toiveensa asettamalla
toisen osapuolen toimijat paremmuusjärjestykseen. Oppilaitosten
tapauksessa järjestys kuvaa esimerkiksi valintakokeen tuloksia tai
aikaisempiin koulutodistuksiin perustuvaa pisteytystä. Hakijoilla
paremmuusjärjestys taas kuvaa suoraan heidän haluaan päästä
opiskelemaan järjestettyihin oppilaitoksiin.

Opiskelijavalintaongelma pyritään ratkaisemaan sijoittamalla hakijat
oppilaitosten avoimiin opiskelupaikkoihin noudattaen sekä oppilaitosten ja
hakijoiden mieltymyksiä, että oppilaitosten opiskelupaikkakiintiöitä.
Samalla pyritään optimoimaan osan tai kaikkien osapuolten lopputulosta
sijoittelussa paremmuusjärjestyksien avulla mitattuna. Luvussa ?
esitellään erilaisia tähän tarkoitukseen kehitettyjä ominaisuuksia.
Haluttuihin ominaisuuksiin vaikuttavat yleisimmin erilaiset
opiskelijavalintaa koskevat lait ja säädökset. Ne voivat esimerkiksi
määrittää tuleeko valinnan suosia hakijoita vai oppilaitoksia.

Kaikki esiteltävät algoritmit toimivat hakijoiden ja oppilaitosten
näkökulmasta samalla periaatteella. Kaikki toimijat ilmoittavat
jollekkin keskitetylle taholle mieltymyksensä. Samalla oppilaitokset
ilmoittavat avoimien paikkojensa määrän ja keitä hakijoista he eivät
ole missään tilanteessa valmiita ottamaan opiskelijoiksi. Hakijat
ilmoittavat vastaavasti mihin oppilaitoksiin he eivät missään
tilanteessa halua tulla sijoitetuksi. Algoritmi ajetaan kaikilla
edellä mainituilla syötteillä, ja tuloksena saadaan sijoittelu joka
määrittää kunkin hakijan opiskelupaikan.

\section{Opiskelijavalintaongelma ja ratkaisualgoritmit}

Määritellään johdannossa kuvattu opiskelijavalintaongelma formaalisti.
Olkoon $O$ oppilaitosten joukko ja $H$ hakijoiden joukko. Olkoon
lisäksi $H_o$ niiden hakijoiden joukko jotka oppilaitos $o$ on valmis
ottamaan vastaan ja vastaavasti $O_h$ niiden oppilaitosten joukko
joihin hakija $h$ on hakenut. Olkoon nyt $P_O = \{P_o \subset H_o
\times H_o | \forall o \in O\}$ ja $P_H = \{P_h \subset O_h \times O_h
| \forall h \in H\}$ joukot täydellisiä järjestysrelaatioita jotka
kertovat mihin järjestykseen oppilaitos $o$ (vastaavasti hakija $h$)
sijoittaa vastakkaisen puolen toimijat. Kutsutaan näitä relaatioita
\emph{preferenssirelaatioiksi}. Merkitään $a \prec_x b$ jos toimija
$x$ pitää enemmän $a$:sta kuin $b$:stä. Olkoon lisäksi jokaisella
oppilaitoksella $o$ avoimien paikkojen määrä $q_o$ ja $Q = \{q_o |
\forall o \in O\}$. Nyt monikko $(O, Q, H, P_O, P_H)$ määrittelee
\emph{opiskelijavalintaongelman}.

Opiskelijavalintaongelma pyritään ratkaisemaan määräämällä halutut
ominaisuudet toteuttava \emph{sijoittelu} $\mu \subset H \times O$
jossa $\mu(h) = o$ jos hakija $h$ on valittu oppilaitokseen $o$.
Merkitään $\mu(h) = \emptyset$ jos hakijaa $h$ ei ole sijoitettu
mihinkään oppilaitokseen. Sijoittelulle tulee päteä $|\mu^{-1}(o)|
\leq q_o$ kaikilla oppilaitoksilla $o$, eli jokaiseen oppilaitokseen
on sijoitettu korkeintaan niin monta hakijaa kuin sillä oli avoimia
paikkoja. Lisäksi jos $h \notin H_o$ tai $o \notin O_h$, niin $\mu(h)
\neq o$. Siis hakija $h$ ei saa tulla sijoitetuksi oppilaitokseen $o$,
jos hän ei niin halua tai oppilaitos ei halua ottaa hakijaa vastaan.

\subsection{Galen ja Shapleyn algoritmi}

Tutkiessaan yliopistojen opiskelijavalintaongelmaa artikkelissaan
College Admission and Stability of Marriage Gale ja Shapley kehittävät
vakauden ja optimaalisuuden käsitteen ja määrittelevät niiden avulla
vakaan avioliiton ongelman. Tämän jälkeen he yleistävät vakaan
avioliiton ongelman ratkaisevan algoritminsa opiskelijavalintaongelman
tapaukseen, ja osoittavat sen edelleen tuottavan vakaan ja
optimaalisen sijoittelun. Galen ja Shapleyn algoritmi tunnetaan myös
heidän itse antamalla nimellä viivytetyn valinnan makanismi (deferred
acceptance mechanism). \cite{galeshapley62}

Galen ja Shapleyn algoritmi etenee seuraavasti, kunnes jokainen hakija
on tullut alustavasti valituksi tai hylätyksi jokaisesta
hakukohteestaan:

\begin{enumerate}
\item Jotta hakija ei tulisi valituksi oppilaitokseen johon häntä ei
  voida valita, karsitaan jokaisen hakijan listalta ne oppilaitokset
  joihin hän ei voi tulla valituksi.

\item Jokainen hakija pyytää tulla sijoitetuksi hänelle mieluisimpaan
  oppilaitokseen.

\item Jokainen oppilaitos $o$ valitsee alustavasti $q_o$ mieluisinta
  hakijaa uusien pyyntöjen ja jo alustavasti valittujen joukosta.
  Loput hylätään.

\item Jokainen hylätty hakijat poistaa hakukohteiden listastaan
  oppilaitoksen josta hänet hylättiin, ja siirrytään kohtaan 2.
\end{enumerate}

Algoritmi voidaan kuvata myös seuraavasti. Jokainen hakija vuorollaan
pyytää tulla sijoitetuksi mieluisimpaan oppilaitokseen josta häntä ei
vielä ole hylätty. Mikäli oppilaitoksessa on tilaa, otetaan hakija
sisään, ja käsittely siirtyy seuraavaan vuoroaan odottavaan hakijaan.

Jos oppilaitoksen kaikki paikat on jo täytetty, mutta oppilaitos pitää
nyt pyynnön tehnyttä hakijaa mieluisampana kuin jotain jo paikan
täyttävää hakijaa, otetaan uusi hakija sisään ja hylätään vähemmän
mieluisa hakija. Jos vähemmän mieluisaa hakijaa ei löydy, hylätään
pyynnön tehnyt hakija. Käsittely siirtyy hylättyyn hakijaan, joka
pyytää tulla sijoitetuksi seuraavaksi mieluisimpaan oppilaitokseen.

\subsection{Bostonin mekanismi}

Bostonin kaupungin vuodesta 1999 käyttämä valinta-algoritmi on antanut
nimen opiskelijavalintaongelman ratkaisemiseksi käytettyille
algoritmeille joita usein kutsutaan \emph{Bostonin mekanismeiksi}.
Sitä käytetään tai on käytetty pienin muunnoksin Bostonin lisäksi
monissa muissa Yhdysvaltain koulupiireissä \cite{abdusön03}.
Seuraavaksi esitettävä algoritmi on Abdulkadiroğlun ja Sönmezin kuvaus
juuri Bostonin kaupungin käyttämästä algoritmista \cite{abdusön03}.
Myöhemmissä luvuissa esiteltävät ominaisuudet kuitenkin koskevat koko
sitä algoritmien joukkoa jonka Kojima ja kumppanit karakterisoivat
artikkelissaan The 'Boston' School Choice Mechanism \cite{kojima10}.

Algoritmi etenee kierroksittain ja jokaisella kierroksella tehdyt
sijoitukset ovat lopullisia. Algoritmin suoritus päättyy, kun kaikki
avoimet paikat on täytetty.

\begin{itemize*}
\item Ensimmäisellä kierroksella jokainen oppilaitos $o$ valitsee
  korkeintaan $q_o$ mieluisinta hakijaa niiden hakijoiden joukosta
  jotka pitävät oppilaitosta $o$ mieluisimpana hakukohteenaan. Valitut
  hakijat sijoitetaan oppilaitokseen.

\item Yleisesti kierroksella $k$ jokainen oppilaitos $o$ jolla on
  vielä avoimia paikkoja pyrkii täyttämään paikat mieltymystensä
  mukaisessa järjestyksessä niillä hakijoilla jotka pitävät
  oppilaitosta $o$ $k.$ mieluisimpana hakukohteenaan.
\end{itemize*}

Bostonin mekanismi eroaa Galen ja Shapleyn algoritmista sillä
merkittävällä tavalla, että kerran sisään otettujen hakijoiden
sijoitusta ei uudelleen mietitä vaikka uusi sisään pyrkivä hakija
olisi oppilaitoksen mielestä mieluisampi.

\subsection{Top trading cycles}

Artikkelissaan School Choice: A Mechanism Design Approach
Abdulkadiroğlu ja Sönmez esittelevät \emph{top trading cycles
  -algoritmin} \cite{abdusön03}. Se on yleistys Galen top trading
cycles -algoritmista \cite{shapley74} ja kuuluu Pápain
karakterisoimaan hierarkisten vaihtosäännöstöjen (hierarchical echange
rules) joukkoon \cite{papai00}.

Top trading cycles -algoritmi toimii seuraavasti. Suoritusta jatketaan
kunnes jokainen hakija on sijoitettu tai jokaisen oppilaitoksen paikat
on täytetty.
\begin{enumerate}

\item Jokainen hakija valitsee mieluisimman hakukohteensa ja jokainen
  oppilaitos mieluisimman hakijansa. Näin syntyy vähintään yksi
  kierros. Kierroksessa on vuoron perään hakija, hakijan mieluisin
  oppilaitos, tämän oppilaitoksen mieluisin hakija ja niin edelleen.
  Viimeisenä kierroksessa on aina oppilaitos joka pitää listan
  ensimmäistä hakijaa mieluisimpana.

\item Jokainen kierros käsitellään sijoittamalla hakija mieluisimpaan
  oppilaitokseensa.

\item Sijoitettut hakijat ja oppilaitokset joiden kaikki paikat on
  täytetty poistetaan käsiteltävien toimijoiden joukosta ja siirrytään
  kohtaan 1.
\end{enumerate}

\section{Sijoittelun ominaisuuksia}

\subsection{Vakaus}

Gale ja Shapley määrittelemä vakauden käsite \cite{galeshapley62} on
seuraava. Vakaassa sijoittelussa ei löydy sellaista oppilaitosta ja
hakijaa, että oppilaitoksessa on vapaa paikkaa tai se pitää hakijaa
mieluisampana kuin jotain siihen sijoitettua hakijaa, ja hakija ei ole
sijoitettu oppilaitokseen vaikka pitää sitä mieluisampana kuin
nykyistä sijoitustaan. Sijoittelu $\mu$ on siis \emph{epävakaa}, jos
löytyy hakija $h$ ja oppilaitos $o$ joilla seuraavat ehdot pätevät:

\begin{itemize*}
  \item $\mu(h) \neq o$, $h \in H_o$ ja $o \in O_h$,
  \item $\mu(h) = \emptyset$ tai $o \prec_h \mu(h)$, ja
  \item $|\mu^{-1}(o)| < q_o$ tai $\exists h' \in \mu^{-1}(o): h
    \prec_o h'$
\end{itemize*}

Gale ja Shapley pitävät vakautta tärkeänä sijoittelun ominaisuutena
\cite{galeshapley62}. Epävakaassa sijoittelussa jollain hakijalla ja
oppilaitoksella on syy haluta muuttaa sijoittelun tulosta sillä
epävakauden poistaminen parantaisi molempien tilannetta. Löyhästi
valvotuissa valinnoissa tämä saattaa johtaa oppilaitosten
omavaltaisiin päätöksiin. Laeilla säännellyissä valinnoissa epävakaus
voi olla ristiriidassa valintaa koskevien yleisten säännösten kanssa.
Abdulkadiroğlu ja Sönmez luonnehtivat vakautta kateuden kautta
\cite{abdusön03}. Epävakaassa sijoittelussa löytyy hakija joka on
oikeutetusti kateellinen jollekkin toiselle hakijalle hänen saamastaan
paikasta. Joissain tapauksissa vakaudesta on kuitenkin tarpeen luopua,
sillä tietyissä tilanteissa vakaus estää Pareto-optimaalisen ratkaisun
\cite{ergin02}.

Galen ja Shapleyn algoritmin antama sijoittelu on vakaa
\cite{galeshapley62, gusfield89}. Bostonin mekanismi ei kuitenkaan
takaa vakaata sijoittelua. Tarkastellaan esimerkkiä, jossa $O = \{o_1,
o_2, o_3\}$, $H = \{h_1, h_2, h_3\}$ ja toimijoiden mieltymykset
taulukon \ref{boston_vakaus} mukaiset.

\begin{table}[h]
  \begin{center}
    \begin{tabular}{ c c c | c c c }
    $o_1$ & $o_2$ & $o_3$ & $h_1$ & $h_2$ & $h_3$ \\
    \hline
    $h_2$ & $h_1$ & $h_1$ & $o_1$ & $o_2$ & $o_1$ \\
    $h_1$ & $h_3$ & $h_2$ & $o_3$ & $o_1$ & $o_2$ \\
    $h_3$ & $h_2$ & $h_3$ & $o_2$ & $o_3$ & $o_3$
    \end{tabular}
    \caption{Toimijoiden mieltymykset sarakkeittain, mieluisampi ylempänä.}
    \label{boston_vakaus}
  \end{center}
\end{table}

Bostonin mekanismi antaa sijoittelun $\{(h_1, o_1), (h_2, o_2), (h_3,
o_3)\}$ joka ei ole vakaa, sillä $h_3 \prec_{o_2} h_2$ ja $o_2
\prec_{h_3} o_3$.

Myöskään top trading cycles -algoritmi ei takaa vakaata sijoittelua.
Yllä olevassa esimerkissä syntyy kierros $h_1 \rightarrow o_1
\rightarrow h_2 \rightarrow o_2$ joten algoritmi määrää parit $(h_1,
o_1)$, $(h_2, o_2)$ ja $(h_3, o_3)$. Kuten edellä on todettu, tämä ei
ole vakaa sijoittelu.

\subsection{Optimaalisuus}

Toinen Galen ja Shapleyn määrittelemä ominaisuus on optimaalisuus
\cite{galeshapley62}. Sijoittelu on hakijaoptimaalinen, jos jokainen
hakija on siinä vähintään yhtä mieluisassa oppilaitoksessa kuin missä
tahansa muussa vakaassa sijoittelussa. Sijoittelu $\mu$ on
\emph{hakijaoptimaalinen}, jos ei ole olemassa toista vakaata
sijoittelu $\gamma$ jossa kaikilla $\gamma(h) = o$ pätee $o \prec_h
\mu(h)$. Vastaavasti voidaan määritellä oppilaitosoptimaalinen
sijoittelu.

Koska Bostonin mekanismi ja top trading cycles -algoritmi eivät
välttämättä anna vakaita sijoitteluja ei optimaalisuuden käsite ole
hyvin määritelty niiden tapauksessa. Galen ja Shapleyn algoritmin
antamat sijoittelut ovat kuitenkin hakijaoptimaalisia
\cite{galeshapley62}. Tämä voidaan todistaa induktiolla. Oletetaan,
että algoritmi on tilanteessa, jossa ketään hakijaa ei vielä ole
hylätty sellaisesta oppilaitoksesta johon hakija sijoitetaan jossain
vakaassa sijoittelussa. Oletetaan, että nyt oppilaitos $o$ hylkää
hakijan $h$, koska kaikki sen avoimet paikat on täytetty. Osoitetaan,
että ei ole olemassa sellaista vakaata sijoittelua jossa $h$
sijoitettaisiin oppilaitokseen $o$. Tehdään vastaoletus, että on
olemassa vakaa sijoittelu $\mu$ jossa $\mu(h) = o$. Mutta nyt on
olemassa hakija $h'$ jolla $\mu(h') \neq o$, mutta $h' \prec_o h$ ja
$o \prec_{h'} \mu(h')$, joten sijoittelu $\mu$ on epävakaa. Siispä
hakija hylätään oppilaitoksesta vain jos syntyvä sijoittelu olisi
epävakaa. Siispä saatu sijoittelu on optimaalinen.

Galen ja Shapleyn algoritmia voidaan lisäksi muuntaa antamaan
oppilaitosoptimaalisia sijoitteluja vaihtamalla hakijoiden ja
oppilaitosten roolit päittäin niin, että oppilaitokset pyytävät
mieluisimpia hakijoitaan opiskelemaan heille \cite{galeshapley62}.
Mikäli hakija- ja oppilaitosoptimaaliset sijoittelut ovat samat, on
vakaa sijoittelu yksikäsitteinen \cite{galeshapley62}.

Hakijaoptimaalinen sijoittelu on samalla
\emph{oppilaitospessimaalinen} vakaa sijoittelu \cite{gusfield89}.
Olkoo $\mu$ hakijaoptimaalinen sijoittelu ja $\gamma$ mikä tahansa muu
sijoittelu. Nyt jokainen oppilaitos pitää kaikkia sijoittelussa
$\gamma$ saamiaan hakijoita parempina kuin mitä tahansa hakijaa jonka
se sai sijoittelussa $\mu$ muttei $\gamma$. Siis kaikilla $o \in O$
pätee, että kaikilla $h \in \gamma^{-1}(o)$ ja $h' \in \mu^{-1}(o)
\setminus \gamma^{-1}(o)$: $h \prec_o h'$. Oppilaitospessimaalisessa
sijoittelussa ei siis ole niin, että jokainen oppilaitos saisi kaikki
huonoimmat hakijansa.

Sijoittelun vakaus vaikuttaa myös oppilaitosten saamiin hakijoihin.
Maalais sairaala -lause (rural hospitals theorem) toteaa, että
kaikille yhden opiskelijavalintaongelman ratkaiseville sijoitteluille
pätee seuraavaa \cite{gusfield89}:

\begin{itemize*}
  \item Jokainen oppilaitos saa saman määrän opiskelijoita kaikissa
    vakaissa sijoitteluissa,
  \item samat hakijat jäävät ilman opiskelupaikkaa kaikissa vakaissa
    sijoitteluissa, ja
  \item jokainen oppilaitos jonka kaikki avoimet paikat eivät täyty,
    saa jokaisessa vakaassa sijoittelussa täsmälleen samat
    opiskelijat.
\end{itemize*}

Maalais sairaala -lause siis takaa, että Galen ja Shapleyn algoritmin
vaihtaminen johonkin toiseen vakaan sijoittelun tuottavaan algoritmiin
ei muuta niiden oppilaitosten tilannetta jotka eivät saa täytettyä
kaikkia vapaita paikkojaan. Erityisesti siis oppilaitospessimaalisen
algoritmin vaihtaminen oppilaitosoptimaaliseen ei paranna vähälle
kiinnostukselle jääneiden oppilaitosten tilannetta.

\subsection{Pareto-optimaalisuus}

Sijoittelu $\mu$ on \emph{Pareto-optimaalinen} jos ei ole olemassa
toista sijoittelua $\gamma$ jossa kaikilla $h \in H$: $(\gamma(h) =
\mu(h)$ tai $\gamma(h) \prec_h \mu(h))$ ja on olemassa $h' \in H$
jolla $\gamma(h') \prec_{h'} \mu(h')$. Siis ei ole olemassa toista
sijoittelua jossa jokainen hakija pääsisi vähintään yhtä mieluisaan
oppilaitokseen ja ainakin yksi hakija pääsisi mieluisampaan
oppilaitoseen kuin sijoittelussa $\mu$.

Määritellyssä opiskelijavalintaongelmassa oppilaitosten on mahdollista
ottaa vastaan vain osan hakijoista, ja hakijoiden on mahdollista hakea
vain osaan oppilaitoksista. Mikäli nämä rajoitetteet käsitellään ennen
algoritmin suoritusta, ja molemmat osapuolet päivittävät
mieltymyksensä on yllä esitetty Pareto-optimaalisuus mahdollista
toteuttaa. Jos näin ei tehdä, ja esimerkiksi hakija voi listata
hakukohteidensa joukkoon oppilaitoksen johon hän ei voi päästä, täytyy
Pareto-optimaalisuutta rajoittaa \cite{kojima10}. Sijoittelu $\mu$ on
\emph{rajoitetusti Pareto-optimaalinen} jos ei ole olemassa toista
sijoittelua $\gamma$ jossa jokainen hakija pääsisi vähintään yhtä
mieluisaan oppilaitokseen, ainakin yksi hakija pääsisi mieluisampaan
oppilaitoseen kuin sijoittelussa $\mu$ ja ketään ei ole sijoitettu
oppilaitokseen johon hän ei ole hakenut tai joka ei pitänyt häntä
hyväksyttävänä hakijana.

Galen ja Shapleyn algoritmi ei takaa Pareto-optimaalista sijoittelua.
Roth osoittaa tämän antamalla esimerkin \ref{roth_optimaalisuus}
tilanteen jossa ainut vakaa sijoittelu ei ole Pareto-optimaalinen
\cite{roth82}. Artikkelissaan Efficient Resource Allocation on the
Basis of Priorities Ergin karakterisoi tilanteen, jossa vakaa
sijoittelu voi olla Pareto\hyph optimaalinen \cite{ergin02}.

\begin{table}[h]
  \begin{center}
    \begin{tabular}{ c c c | c c c }
      $o_1$ & $o_2$ & $o_3$ & $h_1$ & $h_2$ & $h_3$ \\
      \hline
      $h_1$ & $h_2$ & $h_2$ & $o_2$ & $o_1$ & $o_1$ \\
      $h_3$ & $h_1$ & $h_1$ & $o_1$ & $o_2$ & $o_2$ \\
      $h_2$ & $h_3$ & $h_3$ & $o_3$ & $o_3$ & $o_3$
    \end{tabular}
    \caption{Esimerkki tilanteesta, jossa Galen ja Shapleyn algoritmi ei
      anna Pareto-optimaalista sijoittelua.}
    \label{roth_optimaalisuus}
  \end{center}
\end{table}

Talukon \ref{roth_optimaalisuus} esimerkissä ainut vakaa sijoittelu on
$\{(h_1, o_1), (h_2, o_2), (h_3, o_3)\}$, mutta se ei ole
Pareto-optimaalinen. Sijoittelu $\mu = \{(h_1, o_2), (h_2, o_1), (h_3,
o_3)\}$ olisi Pareto-optimaalinen, sillä siinä jokainen hakija pääsee
mieluisimpaan hakukohteeseensa. Se ei kuitenkaan ole vakaa, sillä
$\mu(h_3) \neq o_1$, vaikka $h_3 \prec_{o_1} h_2$ ja $o_1 \prec_{h_3}
o_3$.

Galen ja Shapleyn algoritmin antama sijoittelu $\mu$ on kuitenkin
\emph{heikosti Pareto-optimaalinen} \cite{gusfield89}. Ei siis ole
olemassa toista sijoittelua $\gamma$ jossa jokainen hakija on päässyt
mieluisampaan oppilaitokseen kuin sijoitteluss $\mu$.

Bostonin mekanismi puolestaan takaa Pareto-optimaalisen sijoittelun.
Kojima ja kumppanit todistavat tämän osana karaterisointiaan
\cite{kojima10}. Myös top trading cycles -algoritmi takaa
Pareto-optimaalisen sijoittelun. Tämä on helppo osoittaa algoritmin
toimintaperiaatteen takia.

Ensimmäisellä kierroksella jokainen sijoitettu hakija pääsee
mieluisimpaan oppilaitokseensa. Siispä heidän kohdallaan sijoitusta ei
voi mitenkään parantaa. Seuraavilla kierroksilla jokainen sijoitettu
hakija pääsee mieluisimpaan vielä vapaita paikkoja sisältävään
oppilaitokseen. Heidän tilannettaa ei voida parantaa huonontamatta jo
sijoitettujen hakijoiden tilannetta. Jotta myöhemmillä kierroksilla
sijoitettu hakija voitaisiin siirtää hänelle mieluisampaan
oppilaitokseen, täytyy sieltä siirtää pois jokin toinen hakija. Jotta
hänen tilanteensa ei huononisi, täytyy hänet sijoittaa mieluisampaan
oppilaitokseen ja taas joudumme tekemään tilaa. Lopulta tulisi siirtää
hakija pois hänelle mieluisimmasta oppilaitoksesta, joka väistämättä
asettaisi hänet huonompaan tilanteeseen kuin missä hän aluksi oli.
Siispä top trading cycles -algoritmin antama sijoittelu on
Pareto-optimaalinen.

\subsection{Strategian kestävyys}

Algoritmit voidaan jakaa strategian kestäviin ja kestämättömiin.
Algoritmi on \emph{strategian kestävä}, jos hakija tai joukko
hakijoita ei voi päästä mieluisampaan oppilaitokseen ilmoittamalla
valheellisesti mieltymyksiään. Mieluisuus määritellään hakijoiden
todellisten, ilmoittamatta jääneiden, mieltymysten perusteella.
Yhteistyössä toimivasta ja mieltymyksensä valheellisesti ilmoittavasta
hakijajoukosta osa voi päästä mieluisampaan oppilaitokseen.

Galen ja Shapleyn algoritmi on strategian kestävä, riippumatta siitä
ovatko hakijat vai oppilaitokset pyynnön tekevänä osapuolena
\cite{dubins81}. Strategian kestävyys koskee kuitenkin molemmissa
tapauksissa vain pyynnön tekevää osapuolta. Jos hakijat toimivat
pyynnön tekevänä osapuolena, voivat oppilaitokset saada mieluisampia
hakijoita ilmoittamalla mieltymyksensä valheellisesti \cite{dubins81}.

Bostonin mekanismi ei ole strategian kestävä \cite{abdusön03}. Tämä on
helppo havaita esimerkistä jossa $O = \{o_1, o_2, o_3\}$, $H = \{h_1,
h_2, h_3\}$ ja toimijoiden mieltymykset taulukon
\ref{boston_strategia} mukaiset.

\begin{table}[h]
  \begin{center}
    \begin{tabular}{ c c c | c c c }
      $o_1$ & $o_2$ & $o_3$ & $h_1$ & $h_2$ & $h_3$ \\
      \hline
      $h_1$ & $h_1$ & $h_1$ & $o_1$ & $o_1$ & $o_1$ \\
      $h_2$ & $h_2$ & $h_2$ & $o_2$ & $o_2$ & $o_2$ \\
      $h_3$ & $h_3$ & $h_3$ & $o_3$ & $o_3$ & $o_3$
    \end{tabular}
    \caption{Toimijoiden mieltymykset sarakkeittain, mieluisampi ylempänä.}
    \label{boston_strategia}
  \end{center}
\end{table}

Jokaisella oppilaitoksella $o_n$ on yksi vapaa paikka. Bostonin
mekanismi antaa sijoittelun $\{(h_1, o_1), (h_2, o_2), (h_3, o_3)\}$.
Jos hakija $h_3$ olisi ilmoittanut mieltymyksensä järjestyksessä
$(o_2, o_1, o_3)$ olisi hän tullut sijoitetuksi oppilaitokseen $o_2$
ja näin parantanut saamaansa sijoitusta. Hakijan $h_3$ saama
mieluisampi opiskelupaikka tulee hakijan $h_2$ kustannuksella, sillä
hän tulee sijoitetuksi oppilaitokseen $o_3$. Pathak ja Sönmez
osoittavat, että sama pätee yleisessä tapauksessa. Jos mieltymyksensä
valheellisesti ilmoittanut hakija saa mieluisamman opiskelupaikan se
tapahtuu jonkun mieltymyksensä totuudenmukaisesti ilmoittaneen hakijan
kustannuksella \cite{pathak08}.

Top trading cycles on strategian kestävä \cite{abdusön03}.
Abdulkadiroğlu ja Sönmez antavat seuraavanlaisen intuition
todistukselleen. Ajatellaan, että hakija joka pohtii mieltymystensä
manipulointia sijoitetaan algoritmin askeleella $k$. Koska jokaisella
askeleella hakija on pyrkinyt mieluisimpaan jäljellä olevaan
oppilaitokseen, ovat kaikki oppilaitokset joihin hän olisi mieluummin
tullut sijoitetuksi täyttyneet ennen askelta $k$. Muokkaamalla
mieltymyksiään miten tahansa, hakija ei pysty vaikuttamaan ennen
askelta $k$ syntyneisiin kierroksiin, joten muokkauksesta voi olla
vain haittaa.

\section{Laskennan aikavaativuus}

Galen ja Shapleyn algoritmi on verrattain yksinkertainen yleistys
heidän vakaan avioliiton ongelman ratkaisevalle algoritmille
\cite{gusfield89}. On siis varsin todennäköistä, että myös
opiskelijavalintaongelman ratkaiseva algoritmi toimii vastaavassa
ajassa. Vakaan avioliiton ongelma ratkeaa ajassa $O(n^2)$, missä $n$
on osapuolten joukon mahtavuus \cite{gusfield89}. Näin ollen voi
olettaa opiskelijavalintaongelman ratkeavan ajassa $O(nm)$, missä $n$
on oppilaitosten ja $m$ hakijoiden joukon mahtavuus. Manlove ja
kumppanit antavat ymmärtää näin artikkelissaan Hard variants of stable
marriage \cite{manlove02} ja tekevät tämän oletuksen artikkelissaan A
constraint programming approach to the hospitals/residents problem
\cite{manlove07}.

\subsection{Tasatilanteet mieltymyksissä}

Opiskelijavalintaongelman määrittelyssä sekä oppilaitoksille että
hakijoille annettiin mahdollisuus olla hyväksymättä joitain
vastakkaisen puolen toimijoita. Tämä on käytännön kannalta erityisen
tärkeä myönnytys. Kaikki hakijat eivät ole valmiita esimerkiksi
muuttamaan pitkiä matkoja opiskelupaikan takia, ja täten kaikkien
valinnassa mukana olevien oppilaitosten listaaminen mahdollisiksi ei
olisi järkevää. Oppilaitoksilla on usein tarve karsia hakijoita heidän
ennakkotietojensa perusteella, sillä koulutuksen järjestäminen on
vaikeaa, jos opiskelijoiden lähtötiedot ovat hyvin eri tasolla.

Järjestettäessä hakijoita esimerkiksi valintakokeen pisteiden avulla
syntyy usein tasatilanteita hakijoiden välillä. Usein tasatilanteet
ratkaistaan arvalla, jolloin saadaan lopputulokseksi täydellinen
järjestys. Tämä ei kuitenkaan ole täysin ongelmatonta. Mikäli
tasatilanteiden purkaminen arvalla otetaan osaksi vakaan sijoittelun
takaavaa valinta-algoritmia, eivät kaikki saadut vakaat sijoittelut
enää ole saman kokoisia \cite{manlove02}. Isompi ongelma koskee
laskennan aikavaativuutta. Galen ja Shapleyn algoritmi toimii ajassa
$O(nm)$, mutta mikä tahansa algoritmi joka sallii tasatilanteet
mieltymyksissä on aikavaativuudeltaan NP-täydellinen \cite{manlove02}.

\section{Yhteenveto}

\newpage
\begin{thebibliography}{XXX99}

\bibitem[AbS03]{abdusön03}
  Atila Abdulkadiroğlu, Tayfun Sönmez
  School Choice: A Mechanism Design Approach.
  \emph{The American Economic Review, 93, 3 (2003), sivut 729-747}

\bibitem[DuF81]{dubins81}
  L. E. Dubins, D. A. Freedman
  Machiavelli and the Gale-Shapley Algorithm.
  \emph{The American Mathematical Monthly, 88, 7 (1981)}

\bibitem[Erg02]{ergin02}
  Haluk I. Ergin
  Efficient Resource Allocation on the Basis of Priorities.
  \emph{Econometrica, 70, 6, (2002), sivut 2489-2497}

\bibitem[GaS62]{galeshapley62}
  D. Gale, L. S. Shapley.
  College Admissions and the Stability of Marriage.
  \emph{The American Mathematical Monthly, 69, 1 (1962)}

\bibitem[GuI89]{gusfield89}
  D. Gusfield, Robert W. Irving.
  The Stable Marriage Problem: Structure and Algorithms.
  \emph{MIT Press, Cambridge, MA, 1989}

\bibitem[KoU10]{kojima10}
  Fuhito Kojima, M. Utku Ünver
  The 'Boston' School Choice Mechanism.
  \emph{Boston College Working Papers in Economics}

\bibitem[OPH12]{OPH12}
  Opetushallitus.
  Aloituspaikat, hakijat, hyväksytyt ja paikan vastaanottaneet
  kevään 2011 ammatillisen ja lukiokoulutuksen yhteishaussa.
  \url{https://www.kouluta.fi/koulutadw/faces/app/startupDWReports.jspx?report_id=18}
      [3.3.2012]

\bibitem[MII02]{manlove02}
  David F. Manlove, Robert W. Irving, Kazuo Iwama, Shuichi Miyazaki,
  Yasufumi Morita.
  Hard variants of stable marriage.
  \emph{Theoretical Computer Science, 276, (2002), sivut 261-279}

\bibitem[MMP07]{manlove07}
  D. F. Manlove, G. O'Malley, P. Prosser, C. Unsworth
  A constraint programming approach to the hospitals/residents
  problem.
  \emph{Lecture Notes in Computer Science 4510 (2007), sivut 155-170}

\bibitem[Che89]{cheng89}
  Cheng Ng.
  Lower Bounds for the Stable Marriage Problem and its Variants.
  \emph{30th Annual Symposium on Foundations of Computer Science,
    (1989), sivut 129-133}

\bibitem[PaS08]{pathak08}
  Parag A. Pathak, Tayfun Sönmez
  Leveling the Playing Field: Sincere and Sophisticated Players in the
  Boston Mechanism.
  \emph{The American Economic Review, 98, 4, (2008), sivut 1636-1652}

\bibitem[Pap00]{papai00}
  Szilvia Pápai
  Strategy-Proof Assignment by Hierarchical Exchange.
  \emph{Econometrica, 68, 6, (2000), sivut 1403-1433}

\bibitem[Rot82]{roth82}
  Alvin E. Roth
  The Economics of Matching: Stability and Incentives.
  \emph{Mathematics of Operations Research, 7, 4, (1982), sivut 617-628}

\bibitem[ShS74]{shapley74}
  Lloyd Shapley, Herbert Scarf
  On Cores and Indivisibility.
  \emph{Journal of Mathematical Economics, 1, 1, (1974), sivut 23-37}

\end{thebibliography}
\end{document}
