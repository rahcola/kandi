\documentclass[gradu, twoside]{tktltiki}
\usepackage{amsmath}
\usepackage{url}
\begin{document}

\title{Kolme algoritmia opiskelijavalintaongelman ratkaisemiseen}
\author{Jani Rahkola}
\date{\today}
\level{LUK-tutkielma}
\numberofpagesinformation{}

\classification{A.1 [Introductory and Survey] \\ G.2.1 [Discrete
    Mathematics]: Combinatorics---Combinatorial algorithms \\ F.2.2
  [Analysis of Algorithms and Problems]: Nonnumerical Algorithms and
  Problems---Computations on Discrete Structures}

\keywords{vakaan avioliiton ongelma, opiskelijavalinta, aikavaativuus}

\maketitle

\doublespacing

\faculty{Matemaattis-luonnontieteellinen}
\department{Tietojenkäsittelytieteen laitos}
\subject{Tietojenkäsittelytiede}

\begin{abstract}
Opiskelijavalinta ongelma on tilanne, jossa useat oppilaitokset
kilpailevat keskenään hakijoista. Sekä hakijat että oppilaitokset
asettavat paremmuusjärjestykseen ne vastapuolen hakijat joista he ovat
kiinnostuneita. Opiskelijavalintaongelman ratkaisemiseksi algoritmin
tulee löytää sijoittelu joka määrää yksikäsitteisiä pareja hakijoista
ja oppilaitoksista halutuin ominaisuuksin. Tässä tutkielmassa
esitellään kolme eri algoritmia ja ominaisuudet jotka algoritmit
toteuttavat.
\end{abstract}

\mytableofcontents

\section{Johdanto}

\section{Opiskelijavalintaongelma}

Opiskelijavalintaongelma on tilanne, jossa useat oppilaitokset
kilpailevat keskenään hakijoista. Hakijat ovat mahdollisesti
kiinnostuneita vain osasta oppilaitoksista, tai heidän
valinnanmahdollisuuksiaan on rajattu. Samoin oppilaitokset ovat
voineet karsia hakijoita esimerkiksi valintakokeilla ja
oppilaitoksilla on usein rajattu määrä avoimia paikkoja. Lisäksi
molemmat toimijat asettavat kiinnostuksen kohteensa
paremmuusjärjestykseen. Toiset oppilaitokset kiinnostavat hakijoita
toisia enemmän, ja oppilaitokset haluavat täyttää avoimet paikkansa
mahdollisimman hyvillä hakijoilla.

Määritellään yllä kuvattu ongelma formaalisti. Olkoon $O$
oppilaitosten joukko ja $H$ hakijoiden joukko. Merkitään $a \prec_x b$
jos toimija $x$ pitää enemmän $a$:sta kuin $b$:stä, ja $\emptyset
\prec_x a$ mikäli toimija $x$ ei halua tulla valituksi/valita toimijaa
$a$ missään tilanteessa. Olkoon lisäksi jokaisella oppilaitoksella $o
\in O$ avoimien paikkojen määrä $q_o$. Nyt nelikko $(O, H, \prec_{o \in O},
\prec_{h \in H})$ määrittelee \emph{opiskelijavalintaongelman}.

Opiskelijavalintaongelma pyritään ratkaisemaan määräämällä halutut
ominaisuudet toteuttava \emph{sijoittelu} $\mu : H \rightarrow O \cup
\{\emptyset\}$ jossa $\mu(h) = o$ jos hakija $h$ on valittu
oppilaitokseen $o$. Olkoon $\mu(h) = \emptyset$ jos hakijaa $h$ ei ole
sijoitettu mihinkään oppilaitokseen. Oppilaitokseen $o$ sijoitettujen
hakijoiden joukko on $\mu^{-1}(o) \subset H$.

Sijoittelulle tulee päteä $|\mu^{-1}(o)| \leq q_o$, eli jokaiseen
oppilaitokseen on valittu korkeintaan niin monta hakijaa kuin sillä
oli avoimia paikkoja. Lisäksi jos $\emptyset \prec_o h$ tai $\emptyset
\prec_h o$, niin $\mu(h) \neq o$. Siis hakija $h$ ei saa tulla
sijoitetuksi oppilaitokseen $o$, jos hän ei niin halua tai oppilaitos
ei halua ottaa hakijaa vastaan.

\section{Galen ja Shapleyn algoritmi}

Tutkiessaan yliopistojen opiskelijavalintaongelmaa artikkelissaan
College Admission and Stability of Marriage Gale ja Shapley kehittävät
vakauden ja optimaalisuuden käsitteen ja määrittelevät niiden avulla
vakaan avioliiton ongelman. Tämän jälkeen he yleistävät vakaan
avioliiton ongelman ratkaisevan algoritminsa opiskelijavalintaongelman
tapaukseen, ja osoittavat sen edelleen tuottavan vakaan ja
optimaalisen sijoittelun. Galen ja Shapleyn algoritmi tunnetaan myös
heidän itse antamalla nimellä viivytetyn valinnan makanismi (deferred
acceptance mechanism). \cite{galeshapley62}

Galen ja Shapleyn algoritmi etenee seuraavasti, kunnes jokainen hakija
on tullut alustavasti valituksi tai hylätyksi jokaisesta
hakukohteestaan:

\begin{enumerate}
\item Jotta hakija ei tulisi valituksi oppilaitokseen johon häntä ei
  voida valita, karsitaan jokaisen hakijan listalta ne oppilaitokset
  joihin hän ei voi tulla valituksi.

\item Jokainen hakija pyytää tulla sijoitetuksi hänelle mieluisimpaan
  oppilaitokseen.

\item Jokainen oppilaitos $o$ valitsee alustavasti $q_o$ mieluisinta
  hakijaa uusien pyyntöjen ja jo alustavasti valittujen joukosta.
  Loput hylätään.

\item Jokainen hylätty hakijat poistaa hakukohteiden listastaan
  oppilaitoksen josta hänet hylättiin, ja siirrytään kohtaan 2.
\end{enumerate}

\section{Bostonin mekanismi}

Bostonin kaupungin vuodesta 1999 käyttämä valinta-algoritmi on antanut
nimen opiskelijavalintaongelman ratkaisemiseksi käytettyille
algoritmeille joita usein kutsutaan \emph{Bostonin mekanismeiksi}.
Sitä käytetään tai on käytetty pienin muunnoksin Bostonin lisäksi
monissa muissa Yhdysvaltain koulupiireissä \cite{abdusön03}.
Seuraavaksi esitettävä algoritmi on Abdulkadiroğlun ja Sönmezin kuvaus
juuri Bostonin kaupungin käyttämästä algoritmista \cite{abdusön03}.
Myöhemmissä luvuissa esiteltävät ominaisuudet kuitenkin koskevat koko
sitä algoritmien joukkoa jonka Kojima ja kumppanit karakterisoivat
artikkelissaan The 'Boston' School Choice Mechanism \cite{kojima10}.

Algoritmi etenee kierroksittain ja jokaisella kierroksella tehdyt
sijoitukset ovat lopullisia. Algoritmin suoritus päättyy, kun kaikki
avoimet paikat on täytetty.

\begin{enumerate}
\item Jokainen oppilaitos $o$ valitsee korkeintaan $q_o$ mieluisinta
  hakijaa niiden hakijoiden joukosta jotka pitävät oppilaitosta $o$
  mieluisimpana hakukohteenaan. Valitut hakijat sijoitetaan
  oppilaitokseen.

\setcounter{enumi}{10}
\renewcommand{\labelenumi}{\alph{enumi}.}
\item Yleisesti kierroksella $k$ jokainen oppilaitos $o$ jolla on
  vielä avoimia paikkoja pyrkii täyttämään paikat mieltymystensä
  mukaisessa järjestyksessä niillä hakijoilla jotka pitävät
  oppilaitosta $o$ $k.$ mieluisimpana hakukohteena.
\end{enumerate}

\section{Top trading cycles}

Artikkelissaan School Choice: A Mechanism Design Approach
Abdulkadiroğlu ja Sönmez esittelevät \emph{top trading cycles
  -algoritmin} \cite{abdusön03}. Se on yleistys Galen top trading
cycles -algoritmista \cite{shapley74} ja kuuluu Pápain
karakterisoimaan hierarkisten vaihtosäännöstöjen (hierarchical echange
rules) joukkoon \cite{papai00}.

Top trading cycles -algoritmi toimii seuraavasti. Suoritusta jatketaan
kunnes jokainen hakija on sijoitettu tai jokaisen oppilaitoksen paikat
on täytetty.
\begin{enumerate}

\item Jokainen hakija valitsee mieluisimman hakukohteensa ja jokainen
  oppilaitos mieluisimman hakijansa. Näin syntyy vähintään yksi
  kierros. Kierroksessa on vuoron perään hakija, hakijan mieluisin
  oppilaitos, tämän oppilaitoksen mieluisin hakija ja niin edelleen.
  Viimeisenä kierroksessa on aina oppilaitos joka pitää listan
  ensimmäistä hakijaa mieluisimpana.

\item Jokainen kierros käsitellään sijoittamalla hakija mieluisimpaan
  oppilaitokseensa.

\item Sijoitettut hakijat ja oppilaitokset joiden kaikki paikat on
  täytetty poistetaan käsiteltävien toimijoiden joukosta ja siirrytään
  kohtaan 1.
\end{enumerate}

\section{Vakaus}

Gale ja Shapley määrittelemä vakauden käsite \cite{galeshapley62} on
seuraava. Vakaassa sijoittelussa ei löydy sellaista oppilaitosta ja
hakijaa, että oppilaitoksessa on vapaa paikkaa tai se pitää hakijaa
mieluisampana kuin jotain siihen sijoitettua hakijaa, ja hakija ei ole
sijoitettu oppilaitokseen vaikka pitää sitä mieluisampana kuin
nykyistä sijoitustaan. Sijoittelu $\mu$ on siis \emph{epävakaa}, jos
on olemassa $o \in O$ ja $h \in H$ joilla $\mu(h) \neq o$, mutta $o
\prec_h \mu(h)$ ja oppilaitoksessa $o$ on vapaa paikka, tai on
olemassa $h' \in H$ jolla $\mu(h') = o$ mutta $h \prec_o h'$.

Gale ja Shapley pitävät vakautta tärkeänä sijoittelun ominaisuutena
\cite{galeshapley62}. Epävakaassa sijoittelussa jollain hakijalla ja
oppilaitoksella on syy haluta muuttaa sijoittelun tulosta sillä
epävakauden poistaminen parantaisi molempien tilannetta. Löyhästi
valvotuissa valinnoissa tämä saattaa johtaa oppilaitosten
omavaltaisiin päätöksiin. Laeilla säännellyissä valinnoissa epävakaus
voi olla ristiriidassa valintaa koskevien yleisten säännösten kanssa.
Abdulkadiroğlu ja Sönmez luonnehtivat vakautta kateuden kautta
\cite{abdusön03}. Epävakaassa sijoittelussa löytyy hakija joka on
oikeutetusti kateellinen jollekkin toiselle hakijalle hänen saamastaan
paikasta. Joissain tapauksissa vakaudesta on kuitenkin tarpeen luopua,
sillä tietyissä tilanteissa vakaus estää Pareto-optimaalisen ratkaisun
\cite{ergin02}.

Galen ja Shapleyn algoritmin antama sijoittelu on vakaa
\cite{galeshapley62}.

Bostonin mekanismi ei takaa vakaata sijoittelua. Tarkastellaan
esimerkkiä, jossa $O = \{o_1, o_2, o_3\}$, $H = \{h_1, h_2, h_3\}$ ja
toimijoiden mieltymykset ovat seuraavat:

\[
  \begin{array}{c c c}
    o_1 & o_2 & o_3 \\
    \hline
    h_2 & h_1 & h_1 \\
    h_1 & h_3 & h_2 \\
    h_3 & h_2 & h_3
  \end{array}
  \begin{array}{|c c c}
    h_1 & h_2 & h_3 \\
    \hline
    o_1 & o_2 & o_1 \\
    o_3 & o_1 & o_2 \\
    o_2 & o_3 & o_3
  \end{array}
\]

Bostonin mekanismi antaa sijoittelun $\{(h_1, o_1), (h_2, o_2), (h_3,
o_3)\}$ joka ei ole vakaa, sillä $h_3 \prec_{o_2} h_2$ ja $o_2
\prec_{h_3} o_3$.

Myöskään top trading cycles -algoritmi ei takaa vakaata sijoittelua.
Yllä olevassa esimerkissä syntyy kierros $h_1 \rightarrow o_1
\rightarrow h_2 \rightarrow o_2$ joten algoritmi määrää parit $(h_1,
o_1)$, $(h_2, o_2)$ ja $(h_3, o_3)$. Kuten edellä on todettu, tämä ei
ole vakaa sijoittelu.

\section{Optimaalisuus}

Toinen Galen ja Shapleyn määrittelemä ominaisuus on optimaalisuus
\cite{galeshapley62}. Sijoittelu on hakijaoptimaalinen, jos jokainen
hakija on siinä vähintään yhtä mieluisassa oppilaitoksessa kuin missä
tahansa muussa vakaassa sijoittelussa. Sijoittelu $\mu$ on
\emph{hakijaoptimaalinen}, jos ei ole olemassa toista vakaata
sijoittelu $\gamma$ jossa kaikilla $\gamma(h) = o$ pätee $o \prec_h
\mu(h)$. Vastaavasti voidaan määritellä oppilaitosoptimaalinen
sijoittelu.

Koska Bostonin mekanismi ja top trading cycles -algoritmi eivät
välttämättä anna vakaita sijoitteluja, ei optimaalisuuden käsite ole
hyvin määritelty niiden tapauksessa. Galen ja Shapleyn algoritmin
antamat sijoittelut ovat kuitenkin hakijaoptimaalisia
\cite{galeshapley62}. Algoritmia voidaan lisäksi muuntaa antamaan
oppilaitosoptimaalisia sijoitteluja vaihtamalla hakijoiden ja
oppilaitosten roolit päittäin niin, että oppilaitokset pyytävät
mieluisimpia hakijoitaan opiskelemaan heille \cite{galeshapley62}.
Mikäli hakija- ja oppilaitosoptimaaliset sijoittelut ovat samat, on
vakaa sijoittelu yksikäsitteinen \cite{galeshapley62}.

Hakijaoptimaalinen sijoittelu on samalla oppilaitospessimaalinen vakaa
sijoittelu \cite{gusfield89}. Olkoo $\mu$ hakijaoptimaalinen
sijoittelu ja $\gamma$ mikä tahansa muu sijoittelu. Nyt jokainen
oppilaitos pitää kaikkia sijoittelussa $\gamma$ saamiaan hakijoita
parempina kuin mitä tahansa hakijaa jonka se sai sijoittelussa $\mu$
muttei $\gamma$. Siis kaikilla $o \in O$ pätee, että kaikilla $h \in
\gamma^{-1}(o)$ ja $h' \in \mu^{-1}(o) \setminus \gamma^{-1}(o)$: $h
\prec_o h'$.

\section{Pareto-optimaalisuus}

Sijoittelu $\mu$ on \emph{Pareto-optimaalinen} jos ei ole olemassa
toista sijoittelua $\gamma$ jossa kaikilla $h \in H$: $(\gamma(h) =
\mu(h)$ tai $\gamma(h) \prec_h \mu(h))$ ja on olemassa $h' \in H$
jolla $\gamma(h') \prec_{h'} \mu(h')$. Siis ei ole olemassa toista
sijoittelua jossa jokainen hakija pääsisi vähintään yhtä mieluisaan
oppilaitokseen ja ainakin yksi hakija pääsisi mieluisampaan
oppilaitoseen kuin sijoittelussa $\mu$.

Määritellyssä opiskelijavalintaongelmassa oppilaitosten on mahdollista
ottaa vastaan vain osan hakijoista, ja hakijoiden on mahdollista hakea
vain osaan oppilaitoksista. Mikäli nämä rajoitetteet käsitellään ennen
algoritmin suoritusta, ja molemmat osapuolet päivittävät
mieltymyksensä on yllä esitetty Pareto-optimaalisuus mahdollista
toteuttaa. Jos näin ei tehdä, ja esimerkiksi hakija voi listata
hakukohteidensa joukkoon oppilaitoksen johon hän ei voi päästä, täytyy
Pareto-optimaalisuutta rajoittaa \cite{kojima10}. Sijoittelu $\mu$ on
\emph{rajoitetusti Pareto-optimaalinen} jos ei ole olemassa toista
sijoittelua $\gamma$ jossa jokainen hakija pääsisi vähintään yhtä
mieluisaan oppilaitokseen, ainakin yksi hakija pääsisi mieluisampaan
oppilaitoseen kuin sijoittelussa $\mu$ ja ketään ei ole sijoitettu
oppilaitokseen johon hän ei ole hakenut tai joka ei pitänyt häntä
hyväksyttävänä hakijana.

Galen ja Shapleyn algoritmi ei takaa Pareto-optimaalista sijoittelua.
Roth antaa yksinkertaisen esimerkin tilanteesta jossa ainut vakaa
sijoittelu ei ole Pareto-optimaalinen \cite{roth82}. Algoritmin antama
sijoittelu $\mu$ on kuitenkin \emph{heikosti Pareto-optimaalinen}
\cite{gusfield89}. Ei siis ole olemassa toista sijoittelua $\gamma$
jossa jokainen hakija on päässyt mieluisampaan oppilaitokseen kuin
sijoitteluss $\mu$.

Bostonin mekanismi puolestaan takaa Pareto-optimaalisen sijoittelun.
Kojima ja kumppanit todistavat tämän osana karaterisointiaan
\cite{kojima10}. Samoin top trading cycles -algoritmi takaa
Pareto-optimaalisen sijoittelun. Tämä on helppo osoittaa algoritmin
toimintaperiaatteen takia. Ensimmäisellä kierroksella jokainen
sijoitettu hakija pääsee mieluisimpaan oppilaitokseensa. Siispä heidän
kohdallaan sijoitusta ei voi mitenkään parantaa.

Seuraavilla kierroksilla jokainen sijoitettu hakija pääsee
mieluisimpaan vielä vapaita paikkoja sisältävään oppilaitokseen.
Heidän tilannettaa ei voida parantaa huonontamatta jo sijoitettujen
hakijoiden tilannetta. Jotta myöhemmillä kierroksilla sijoitettu
hakija voitaisiin siirtää hänelle mieluisampaan oppilaitokseen, täytyy
sieltä siirtää pois jokin toinen hakija. Jotta hänen tilanteensa ei
huononisi, täytyy hänet sijoittaa mieluisampaan oppilaitokseen ja taas
joudumme tekemään tilaa. Lopulta tulisi siirtää hakija pois hänelle
mieluisimmasta oppilaitoksesta, joka väistämättä asettaisi hänet
huonompaan tilanteeseen kuin missä hän aluksi oli. Siispä top trading
cycles -algoritmin antama sijoittelu on Pareto-optimaalinen.

\section{Strategian kestävyys}

Algoritmit voidaan jakaa strategian kestäviin ja kestämättömiin.
Algoritmi on strategian kestävä, jos hakija tai joukko hakijoita ei
voi päästä mieluisampaan oppilaitokseen ilmoittamalla valheellisesti
mieltymyksiään.

Galen ja Shapleyn algoritmi on strategian kestävä \cite{dubins81}.

Bostonin mekanismi ei ole strategian kestävä.

Top trading cycles on strategian kestävä.

\section{Laskennan aikavaativuus}

Gusfield ja Irving esittävät analyysin vakaan avioliiton ongelman
ratkaisevalle Galen ja Shapleyn algoritmille, ja toteavat sen toimivan
$O(n^2)$ ajassa \cite{gusfield89}. Lisäksi on osoitettu, että
algoritmin aikavaativuus itse asiassa on $\Theta(n^2)$ \cite{cheng89}.
Gusfield ja Irving eivät analysoi algoritmin opiskelijavalintaongelman
ratkaisevaa versiota. Heidän antamasta muotoilusta on kuitenkin helppo
havaita, että algoritmin aikavaativuus on $O(nm)$ missä $n$ ja $m$
ovat hakijoiden ja oppilaitosten lukumäärät.

\section{Yhteenveto}

\newpage
\begin{thebibliography}{XXX99}

\bibitem[AbS03]{abdusön03}
  Atila Abdulkadiroğlu, Tayfun Sönmez
  School Choice: A Mechanism Design Approach
  \emph{The American Economic Review, 93, 3 (2003), sivut 729-747}

\bibitem[DuF81]{dubins81}
  L. E. Dubins, D. A. Freedman
  Machiavelli and the Gale-Shapley Algorithm
  \emph{The American Mathematical Monthly, 88, 7 (1981)}

\bibitem[Erg02]{ergin02}
  Haluk I. Ergin
  Efficient Resource Allocation on the Basis of Priorities
  \emph{Econometrica, 70, 6, (2002), sivut 2489-2497}

\bibitem[GaS62]{galeshapley62}
  D. Gale, L. S. Shapley.
  College Admissions and the Stability of Marriage.
  \emph{The American Mathematical Monthly, 69, 1 (1962)}

\bibitem[GuI89]{gusfield89}
  D. Gusfield, Robert W. Irving.
  The Stable Marriage Problem: Structure and Algorithms.
  \emph{MIT Press, Cambridge, MA, 1989}

\bibitem[KoU10]{kojima10}
  Fuhito Kojima, M. Utku Ünver
  The 'Boston' School Choice Mechanism
  \emph{Boston College Working Papers in Economics}

\bibitem[OPH12]{OPH12}
  Opetushallitus.
  Aloituspaikat, hakijat, hyväksytyt ja paikan vastaanottaneet
  kevään 2011 ammatillisen ja lukiokoulutuksen yhteishaussa.
  \url{https://www.kouluta.fi/koulutadw/faces/app/startupDWReports.jspx?report_id=18}
      [3.3.2012]

\bibitem[MII02]{manlove02}
  David F. Manlove, Robert W. Irving, Kazuo Iwama, Shuichi Miyazaki,
  Yasufumi Morita.
  Hard variants of stable marriage.
  \emph{Theoretical Computer Science, 276, (2002), sivut 261-279}

\bibitem[Che89]{cheng89}
  Cheng Ng.
  Lower Bounds for the Stable Marriage Problem and its Variants.
  \emph{30th Annual Symposium on Foundations of Computer Science,
    (1989), sivut 129-133}

\bibitem[Pap00]{papai00}
  Szilvia Pápai
  Strategy-Proof Assignment by Hierarchical Exchange
  \emph{Econometrica, 68, 6, (2000), sivut 1403-1433}

\bibitem[ShS74]{shapley74}
  Lloyd Shapley, Herbert Scarf
  On Cores and Indivisibility
  \emph{Journal of Mathematical Economics, 1, 1, (1974), sivut 23-37}

\end{thebibliography}
\end{document}
