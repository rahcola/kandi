% \iffalse meta-comment
%
% Copyright (C) 2002 Mikael Puolakka
% ----------------------------------
%
% Modifications in version 1.1 (c) 2007 Antti Leino
%
% This file may be distributed and/or modified under the
% conditions of the LaTeX Project Public License, either
% version 1.2 of this license or (at your option) any later
% version. The latest version of this license is in
%
%    http://www.latex-project.org/lppl.txt
%
% and version 1.2 or later is part of all distributions of
% LaTeX version 1999/12/01 or later.
%
% \fi
% \CheckSum{1706}
%% \CharacterTable
%%  {Upper-case    \A\B\C\D\E\F\G\H\I\J\K\L\M\N\O\P\Q\R\S\T\U\V\W\X\Y\Z
%%   Lower-case    \a\b\c\d\e\f\g\h\i\j\k\l\m\n\o\p\q\r\s\t\u\v\w\x\y\z
%%   Digits        \0\1\2\3\4\5\6\7\8\9
%%   Exclamation   \!     Double quote  \"     Hash (number) \#
%%   Dollar        \$     Percent       \%     Ampersand     \&
%%   Acute accent  \'     Left paren    \(     Right paren   \)
%%   Asterisk      \*     Plus          \+     Comma         \,
%%   Minus         \-     Point         \.     Solidus       \/
%%   Colon         \:     Semicolon     \;     Less than     \<
%%   Equals        \=     Greater than  \>     Question mark \?
%%   Commercial at \@     Left bracket  \[     Backslash     \\
%%   Right bracket \]     Circumflex    \^     Underscore    \_
%%   Grave accent  \`     Left brace    \{     Vertical bar  \|
%%   Right brace   \}     Tilde         \~}
%
% \iffalse
%
%<*driver>
\documentclass{ltxdoc}[1999/08/08 v2.0u Standard LaTeX documentation class]
\usepackage[utf8]{inputenc}[2000/07/01 v0.996 Input encoding file ]
\usepackage[T1]{fontenc}[2000/08/30 v1.91 Standard LaTeX package]
\usepackage[finnish]{babel}[2001/03/01 v3.7h The Babel package]
\EnableCrossrefs
\CodelineIndex
\RecordChanges
\begin{document}
  \DocInput{tktltiki.dtx}
\end{document}
%</driver>
% \fi
%
% \changes{v1.0}{2002/09/16}{Alkuperäinen versio}
% \changes{v1.1}{2007/09/12}{Tiivistelmälomaketta korjattu}
%
% \GetFileInfo{tktltiki.cls}
%
% \DoNotIndex{\',\.,\@M,\@@input,\@Alph,\@alph,\@addtoreset,\@arabic}
% \DoNotIndex{\@badmath,\@centercr,\@cite}
% \DoNotIndex{\@dotsep,\@empty,\@float,\@gobble,\@gobbletwo,\@ignoretrue}
% \DoNotIndex{\@input,\@ixpt,\@m,\@minus,\@mkboth}
% \DoNotIndex{\@ne,\@nil,\@nomath,\@plus,\roman,\@set@topoint}
% \DoNotIndex{\@tempboxa,\@tempcnta,\@tempdima,\@tempdimb}
% \DoNotIndex{\@tempswafalse,\@tempswatrue,\@viipt,\@viiipt,\@vipt}
% \DoNotIndex{\@vpt,\@warning,\@xiipt,\@xipt,\@xivpt,\@xpt,\@xviipt}
% \DoNotIndex{\@xxpt,\@xxvpt,\\,\ ,\addpenalty,\addtolength,\addvspace}
% \DoNotIndex{\advance,\ast,\begin,\begingroup,\bfseries,\bgroup,\box}
% \DoNotIndex{\bullet}
% \DoNotIndex{\cdot,\cite,\CodelineIndex,\cr,\day,\DeclareOption}
% \DoNotIndex{\def,\DisableCrossrefs,\divide,\DocInput,\documentclass}
% \DoNotIndex{\DoNotIndex,\egroup,\ifdim,\else,\fi,\em,\endtrivlist}
% \DoNotIndex{\EnableCrossrefs,\end,\end@dblfloat,\end@float,\endgroup}
% \DoNotIndex{\endlist,\everycr,\everypar,\ExecuteOptions,\expandafter}
% \DoNotIndex{\fbox}
% \DoNotIndex{\filedate,\filename,\fileversion,\fontsize,\framebox,\gdef}
% \DoNotIndex{\global,\halign,\hangindent,\hbox,\hfil,\hfill,\hrule}
% \DoNotIndex{\hsize,\hskip,\hspace,\hss,\if@tempswa,\ifcase,\or,\fi,\fi}
% \DoNotIndex{\ifhmode,\ifvmode,\ifnum,\iftrue,\ifx,\fi,\fi,\fi,\fi,\fi}
% \DoNotIndex{\input}
% \DoNotIndex{\jobname,\kern,\leavevmode,\let,\leftmark}
% \DoNotIndex{\list,\llap,\long,\m@ne,\m@th,\mark,\markboth,\markright}
% \DoNotIndex{\month,\newcommand,\newcounter,\newenvironment}
% \DoNotIndex{\NeedsTeXFormat,\newdimen}
% \DoNotIndex{\newlength,\newpage,\nobreak,\noindent,\null,\number}
% \DoNotIndex{\numberline,\OldMakeindex,\OnlyDescription,\p@}
% \DoNotIndex{\pagestyle,\par,\paragraph,\paragraphmark,\parfillskip}
% \DoNotIndex{\penalty,\PrintChanges,\PrintIndex,\ProcessOptions}
% \DoNotIndex{\protect,\ProvidesClass,\raggedbottom,\raggedright}
% \DoNotIndex{\refstepcounter,\relax,\renewcommand}
% \DoNotIndex{\rightmargin,\rightmark,\rightskip,\rlap,\rmfamily}
% \DoNotIndex{\secdef,\selectfont,\setbox,\setcounter,\setlength}
% \DoNotIndex{\settowidth,\sfcode,\skip,\sloppy,\slshape,\space}
% \DoNotIndex{\symbol,\the,\trivlist,\typeout,\tw@,\undefined,\uppercase}
% \DoNotIndex{\usecounter,\usefont,\usepackage,\vfil,\vfill,\viiipt}
% \DoNotIndex{\viipt,\vipt,\vskip,\vspace}
% \DoNotIndex{\wd,\xiipt,\year,\z@}
%
% \def\fileversion{v1.0}
% \def\filedate{2002/09/16}
% \newcommand{\luokka}{|tktltiki|}
% \newcommand{\bibtyyli}{|tktl|}
% \newenvironment{options}{%
%   \begin{list}{}{%
%   \renewcommand{\makelabel}[1]{\texttt{##1}\hfil}%
%   \setlength{\itemsep}{-.5\parsep}
%   \settowidth{\labelwidth}{\texttt{xxxxxxxxxxx\space}}%
%   \setlength{\leftmargin}{\labelwidth}%
%   \addtolength{\leftmargin}{\labelsep}}%
%   \raggedright}{
%   \end{list}
% }
%
% \title{Luokka \luokka\ \LaTeX{}-versiolle
%   2$_{\varepsilon}$\thanks{Tämän tiedoston versionumero on \fileversion,
%   muutettu viimeksi \filedate.}}
% \author{Mikael Puolakka \\ \texttt{Mikael.Puolakka@cs.Helsinki.FI}}
% \date{\filedate}
%
% \maketitle
%
% \begin{abstract}
% Dokumenttiluokka \luokka\ on tarkoitettu tukemaan Helsingin yliopiston
% Tietojenkäsittelytieteen laitoksen alempien opinnäytteiden ja harjoitusten
% laatimista. Luokka soveltuu siten käytettäväksi tieteellisen kirjoittamisen
% kurssille, ohjelmistotuotantoprojekteihin ja pro gradu -tutkielmiin. Luokka
% määrittelee sitä käyttävälle dokumentille tietyn ulkoasun ja tarjoaa
% käyttöliittymän dokumentin rakenteen hallitsemiseksi.
% \end{abstract}
%
% \tableofcontents
%
% \section{Johdanto}
%
% Tämä dokumentti on luokan \luokka\ toteutuskuvaus. Käytännön syistä tämä
% dokumentti ei noudata tyypillistä \LaTeX-tyylin dokumentaatiota, joka
% yhdistäisi sekä käyttöohjeen että toteutuskuvauksen. Sen sijaan luokan
% \luokka\ käyttöohje sisältyy tämän dokumentin koodiosioon, josta se on
% tuotettavissa itsenäiseksi \LaTeX-käsikirjoitustiedostokseen ajamalla
% luokan \luokka\ asennustiedosto |tktltiki.ins| \LaTeX-ohjelmalla.
%
% \StopEventually{\PrintChanges\PrintIndex}
%
% \section{Koodi}
% \label{koodi}
%
%    Identifioidaan itsemme:
%
%    \begin{macrocode}
%<*luokka>
\NeedsTeXFormat{LaTeX2e}[2000/06/01]
\ProvidesClass{tktltiki}
  [2007/09/12 v1.1 Dokumenttiluokka TiKi-kurssia varten]
%    \end{macrocode}
%
%    Määritellään ''kytkimiä'' optioita varten. Oletuksena kytkimet ovat
%    |false|-asennossa. Jos optio annetaan dokumentin
%    |\documentclass|-komennossa, vastaava kytkin käännetään päälle eli
%    |true|-asentoon.
%
%    \begin{macrocode}
\newif\if@finnish
\newif\if@swedish
\newif\if@english
\newif\if@gradu
\newif\if@emptyfirstpagenumber

\DeclareOption{finnish}{\@finnishtrue}
\DeclareOption{swedish}{\@swedishtrue}
\DeclareOption{english}{\@englishtrue}
\DeclareOption{gradu}{\@gradutrue}
\DeclareOption{emptyfirstpagenumber}{\@emptyfirstpagenumbertrue}
%    \end{macrocode}
%
%    Osa |article|-dokumenttiluokan optioista välitetään suoraan
%    |article|-luokalle, mutta \luokka-dokumenttiluokan oletusasetuksia
%    kumoavat optiot joudutaan välittämään sille välikäden kautta, jotta
%    saadaan tietää mitkä oletusasetukset on kumottu annetuilla optioilla.
%    Määritellään kytkimet sellaisia oletusasetuksia varten, jotka on
%    mahdollista kumota |\documentclass|-komennossa annettavilla optioilla,
%    ja asetetaan ne |true|-asentoon. Mikäli jokin annettu optio kumoaa
%    jonkin oletusasetuksen, vastaava kytkin asetetaan |false|-asentoon.
%    Lopuksi prosessoidaan annetut optiot.
%
%    \begin{macrocode}
\newif\if@defaulttypesize
\@defaulttypesizetrue
\newif\if@defaultpaper
\@defaultpapertrue
\newif\if@defaulttitlepage
\@defaulttitlepagetrue

\DeclareOption{10pt}{
  \@defaulttypesizefalse
  \PassOptionsToClass{10pt}{article}}
\DeclareOption{11pt}{
  \@defaulttypesizefalse
  \PassOptionsToClass{11pt}{article}}

\DeclareOption{a5paper}{
  \@defaultpaperfalse
  \PassOptionsToClass{a5paper}{article}}
\DeclareOption{b5paper}{
  \@defaultpaperfalse
  \PassOptionsToClass{b5paper}{article}}
\DeclareOption{letterpaper}{
  \@defaultpaperfalse
  \PassOptionsToClass{letterpaper}{article}}
\DeclareOption{legalpaper}{
  \@defaultpaperfalse
  \PassOptionsToClass{legalpaper}{article}}
\DeclareOption{executivepaper}{
  \@defaultpaperfalse
  \PassOptionsToClass{executivepaper}{article}}
  
\DeclareOption{notitlepage}{
  \@defaulttitlepagefalse
  \PassOptionsToClass{notitlepage}{article}}
  
\DeclareOption*{\PassOptionsToClass{\CurrentOption}{article}}
\ProcessOptions\relax  
%    \end{macrocode}
%
%    Dokumenttiluokka \luokka\ perustuu |article|-luokkaan, joka ladataan
%    oletusarvoisesti optioilla |12pt|, |a4paper| ja |titlepage| ellei
%    luokkaa \luokka\ ole kutsuttu vaihtoehtoisilla optioilla. Luokan
%    \luokka\ oletusoptioiden kanssa ristiriidattomat |article|-luokan
%    optiot välitettiin jo aiemmin suoraan |article|-luokalle.
%
%    \begin{macrocode}
\if@defaulttypesize
  \PassOptionsToClass{12pt}{article}
\fi
\if@defaultpaper
  \PassOptionsToClass{a4paper}{article}
\fi
\if@defaulttitlepage
  \PassOptionsToClass{titlepage}{article}
\fi
\ProcessOptions\relax
\LoadClass{article}[2000/05/19]
%    \end{macrocode}
%
%    Lisäksi oletusasetuksiin kuuluu skandimerkit ja vanhan \LaTeX-version
%    2.09 mukaiset symbolit.
%
%    \begin{macrocode}
\RequirePackage{latexsym}[1998/08/17 v2.2e Standard LaTeX package (lasy symbols)]
%    \end{macrocode}
%    Käytetään ääkkösiä turhan kikkailun sijasta. ISO~Latin~9 -merkistö
%    sisältää muitakin hyödyllisiä merkkejä.
%
%    \begin{macrocode}
\RequirePackage[latin9]{inputenc}[2000/07/01 v0.996 Input encoding file ]
%    \end{macrocode}
%    Käytetään fonttia, jossa skandimerkkejä kohdellaan ''oikeina''
%    kirjaimina.
%
%    \begin{macrocode}
\RequirePackage[T1]{fontenc}[2000/08/30 v1.91 Standard LaTeX package]
%    \end{macrocode}
%    Käytetään dokumentin kielenä joko suomea, ruotsia tai englantia.
%    Oletuskielenä on suomi, ellei optiona ole annettu minkään muun tuetun
%    kielen nimeä. Jos optiona on annettu yksi tuetun kielen nimi,
%    käytetään sitä. Jos optiona on annettu useampia tuettujen kielien
%    nimiä, käytetään sitä joka tulee annetuista kielistä ensimmäisenä
%    vastaan alla olevassa ehtolauseessa.
%
%    \begin{macrocode}
\if@finnish
  \RequirePackage[finnish]{babel}[2001/03/01 v3.7h The Babel package]
\else
  \if@english
    \RequirePackage[english]{babel}[2001/03/01 v3.7h The Babel package]
  \else
    \if@swedish
      \RequirePackage[swedish]{babel}[2001/03/01 v3.7h The Babel package]
    \else
      \RequirePackage[finnish]{babel}[2001/03/01 v3.7h The Babel package]
    \fi
  \fi  
\fi
%    \end{macrocode}
%
% \begin{macro}{\defaultsettings}
%
%    Lisää oletusasetuksia: asetellaan marginaaleja. Marginaalien
%    oletusasetukset saa mahdollisten muutosten jälkeen käyttöön
%    |\defaultsettings|-komennolla. Yhdellä sivulla voi käyttää vain yksiä
%    asetuksia, joten jos asetuksia haluaa muuttaa väliaikaisesti vain
%    yhdelle sivulle, kannattaa käyttää |\geometry|-komentoa asetuksien
%    muuttamiseen ja |\newpage|-komentoa sivun lopussa, jonka jälkeen
%    voi käyttää käskyä |\defaultsettings| oletusasetuksien palauttamiseen.
%    {\bfseries Huom! |\defaultsettings|-komento ei toimi toivotulla tavalla
%    kaikissa \LaTeX-toteutuksissa, joten sen käyttöä on syytä välttää
%    mahdollisten yhteensopivuusongelmien välttämiseksi. Ongelmana on se,
%    ettei |\geometry|-komentoa saisi kutsua preamblen jälkeen.}
%
%    \begin{macrocode}
\RequirePackage{geometry}[1999/10/07 v2.2 Page Geometry]
\newcommand{\defaultsettings}{%
  \if@twoside
    \geometry{top=2.5cm, left=2.8cm, right=3.2cm,
              textwidth=15cm, textheight=23cm,
              headheight=0.5cm, headsep=0.5cm}%
  \else
    \geometry{top=2.5cm, left=3.5cm, right=2.5cm,
              textwidth=15cm, textheight=23cm,
              headheight=0.5cm, headsep=0.5cm}%
  \fi
}
\defaultsettings
%    \end{macrocode}
% \end{macro}
%
%    Kappaleiden alkuja ei sisennetä ja kappaleiden väliin tulee hieman
%    tyhjää.
%
%    \begin{macrocode}
\setlength{\parindent}{0pt}
\setlength{\parskip}{1ex}
%    \end{macrocode}
%
% \begin{macro}{\mytableofcontents}
%
% \begin{sloppypar}
%    Sisällysluettelosivuilla ja mahdollisilla muilla tavallisilla sivuilla
%    otsikko- ja tiivistelmäsivua lukuun ottamatta sivunumerot tulevat
%    sivun oikeaan yläkulmaan ja sivunumerointiin käytetään roomalaisia
%    numeroita (i,~ii,~iii,~iv\ldots). |\mytableofcontents|-komennon jälkeen
%    sivunumerointiin käytetään arabialaisia numeroita (1,~2,~3,~4\ldots).
%    Tämän voi halutessaan kumota määrittämällä itse sivunumeroinnin
%    |\mytableofcontents|-komennon jälkeen.
% \end{sloppypar}
%
%    \begin{macrocode}
\pagestyle{myheadings}
\markright{}
\pagenumbering{roman}
%    \end{macrocode}
% \end{macro}
%
% \begin{macro}{\onehalfspacing}
% \begin{macro}{\doublespacing}
% \begin{macro}{\singlespacing}
%
%    Seuraavassa kahdessa käskyssä käytetään seuraavia lukuja:
%
%    \begin{center}
%    \begin{tabular}{llll}
%    \multicolumn{1}{l}{} &
%    \multicolumn{3}{c}{kirjasinkoko} \\
%    väli & 10pt  & 11pt  & 12pt \\
%    1.5  & 1.25  & 1.213 & 1.241 \\
%    2.0  & 1.667 & 1.618 & 1.655 \\
%    \end{tabular}
%    \end{center}
%    Muuttaa käytetyn rivivälin 1.5-riviväliksi.
%
%    \begin{macrocode}
\newcommand{\onehalfspacing}{%
  \ifcase\@ptsize\relax % 10pt
    \renewcommand{\baselinestretch}{1.25}%
  \or % 11pt
    \renewcommand{\baselinestretch}{1.213}%
  \or % 12pt
    \renewcommand{\baselinestretch}{1.241}%
  \fi
  \normalsize
}
%    \end{macrocode}
%    Muuttaa käytetyn rivivälin 2.0-riviväliksi.
%
%    \begin{macrocode}
\newcommand{\doublespacing}{
  \ifcase\@ptsize\relax % 10pt
    \renewcommand{\baselinestretch}{1.667}
  \or % 11pt
    \renewcommand{\baselinestretch}{1.618}
  \or % 12pt
    \renewcommand{\baselinestretch}{1.655}
  \fi
  \normalsize
}
%    \end{macrocode}
%    Muuttaa käytetyn rivivälin takaisin 1.0-riviväliksi.
%
%    \begin{macrocode}
\newcommand{\singlespacing}{%
  \renewcommand{\baselinestretch}{1.0}%
  \normalsize
}
%    \end{macrocode}
%    Huom! Pakkauksessa |setspace| on vaihtoehtoisia komentoja
%    rivivälistyksien muuttamiseen.
% \end{macro}
% \end{macro}
% \end{macro}
%
% \begin{macro}{\mytableofcontents}
%
%    Muuten sama kuin |\tableofcontents|, mutta aloittaa uuden sivun
%    sisällysluettelon jälkeen ja muuttaa sisällysluetteloon saakka
%    sivunumerointiin käytetyt roomalaiset numerot arabialaisiksi
%    numeroiksi (1,~2,~3,~4\ldots).
%
%    |emptyfirstpagenumber|-option ollessa voimassa varsinaisen tekstin
%    ensimmäiselle sivulle ei tule sivunumeroa. Tämä siis kuitenkin
%    vaatii |\mytableofcontents|-komennon käyttöä.
%
%    \begin{macrocode}
\newcommand{\mytableofcontents}{%
  \tableofcontents
  \newpage
  \pagenumbering{arabic}
  \if@emptyfirstpagenumber
    \thispagestyle{empty}
  \fi
}
%    \end{macrocode}
% \end{macro}
%
% \begin{macro}{\theappendix}
%
%    Luodaan laskuri liitteille. Jokainen |\internalappendix|- ja
%    |\externalappendix|-komento kasvattaa laskuria yhdellä.
%    |\theappendix|-komennolla saadaan tietää |appendix|-laskurin nykyinen
%    arvo. Laskurin arvo ilmoittaa liitteiden nykyisen lukumäärän lisättynä
%    yhdellä, joten käytännössä se ilmoittaa seuraavan mahdollisen liitteen
%    järjestysnumeron ja sopii siten käytettäväksi |\internalappendix|- ja
%    |\externalappendix|-komentojen \meta{liitteen numero}-parametriksi.
%
%    \begin{macrocode}
\newcounter{appendix}
%    \end{macrocode}
% \end{macro}
%
% \begin{macro}{\appendices}
%
%    Aloittaa dokumentin \emph{Liitteet}-osion. Liitteiden sivunumeroja ei
%    ilmesty sisällysluetteloon. Alustaa |appendix|-laskurin, joka
%    ilmoittaa aina seuraavaksi tulevan liitteen järjestysnumeron eli kun
%    liitteitä on 0 kpl, laskurin lukema on 1, ja kun liitteitä on 1 kpl,
%    laskurin lukema on 2, jne. Lopuksi nollaa sivunumerolaskurin |page|,
%    jotta liitesivujen lukumäärän laskeminen olisi helpompaa.
%
%    \begin{macrocode}
\newcommand{\appendices}{%
  \setcounter{appendix}{1}%
  \newpage
%    \end{macrocode}
%    Jos et halua liitteisiin sivunumeroita, poista kommenttimerkki
%    seuraavalta riviltä ja poista sitä seuraavat kaksi riviä.
%    \begin{macrocode}
  %\pagestyle{empty}
  \pagestyle{myheadings}
  \markright{}
  \appendix
  \addtocontents{toc}{\protect \contentsline {section}{\enclname}{}}
  \setcounter{page}{0}
}
%    \end{macrocode}
% \end{macro}
%
% \begin{macro}{\internalappendix}
%
%    Tuottaa liitteen otsikon siihen kohtaan, jossa tätä kutsutaan, ja
%    liittää sisällysluetteloon liitteen nimen. Liiteiden numerointi
%    tapahtuu manuaalisesti. Aloittaa aina uuden sivun. Kunkin liitteen
%    sivut numeroidaan erikseen. Käyttö:
%
%    \begin{quote}
%    |\internalappendix|\marg{liitteen numero}\marg{liitteen nimi}
%    \end{quote}
%
%    \begin{macrocode}
\newcommand{\internalappendix}[2]{%
%    \end{macrocode}
%    |appendixpage|-laskurin lukemaan lisätään aina edellisen liitteen
%    sivujen lukumäärä. Viimeisen liitteen sivujen laskenta suoritetaan
%    |\AtEndDocument|-komennossa avulla hieman myöhempänä.
%
%    \begin{macrocode}
  \addtocounter{appendixpage}{\value{page}}%
  \newpage
  \setcounter{page}{1}%
%    \end{macrocode}
%    Tämä on se otsikko, joka näkyy liitteen otsikkona.
%
%    \begin{macrocode}
  \section*{\appendixname\ #1. #2}
%    \end{macrocode}
%    Lisätään seuraavaksi liitteen nimi sisällysluetteloon.
%
%    \begin{macrocode}
  \addtocontents{toc}{
    \protect \contentsline {section}{\hspace{0.5cm}#1 #2}{}}
  \addtocounter{appendix}{+1}%
}
%    \end{macrocode}
% \end{macro}
%
% \begin{macro}{\externalappendix}
%
%    Ei tee mitään muuta kuin liittää liitteen nimen sisällysluetteloon.
%    Tätä kannattaa kutsua silloin, kun haluaa liittää osaksi dokumenttia
%    dokumentin ulkopuolisen liitteen, kuten ohjelmalistauksen, levykkeen,
%    tms. Käyttö:
%
%    \begin{quote}
%    |\externalappendix|\marg{liitteen numero}\marg{liitteen nimi}
%    \end{quote}
%
%    \begin{macrocode}
\newcommand{\externalappendix}[2]{%
%    \end{macrocode}
%    Lisätään liitteen nimi sisällysluetteloon.
%
%    \begin{macrocode}
  \addtocontents{toc}{
    \protect \contentsline {section}{\hspace{0.5cm}#1 #2}{}}
  \addtocounter{appendix}{+1}%
}
%    \end{macrocode}
% \end{macro}
%
%    Määritellään uusi laskuri, jolla lasketaan sekä tavallisten liitteiden
%    että dokumentin ulkopuolisten liitteiden sivujen lukumäärää.
%
%    \begin{macrocode}
\newcounter{appendixpage}
%    \end{macrocode}
%
% \begin{macro}{\numberofappendixpages}
%
%    |\numberofappendixpages|-komento kertoo kaikkien liitteiden sivumäärän. Sillä on
%    yksi valinnainen parametri: ulkoliitteiden sivumäärä. Jos se puuttuu
%    komentokutsusta, ulkoliitteiden sivumääräksi tulee 0.
%
%    Kaikkien muiden kuin viimeisen liitteen sivumäärät lisättiin laskuriin
%    |\internalappendix|-komennossa. Viimeisen liitteen sivumäärä lisätään laskuriin
%    |\AtEndDocument|in avulla ja laskurin arvo kirjoitetaan
%    |aux|-tiedostoon labelin |@lastappendixpage| ''osanumeroksi''\footnote{%
%    ''Osa'' on esimerkiksi |equation|, |section|, |figure|, tms.}, jolloin
%    siihen voi viitata |\ref|-komennolla.
%
%    \begin{macrocode}
\newcommand{\numberofappendixpages}[1][0]{%
  \addtocounter{appendixpage}{#1}%
  \ref{@lastappendixpage}%
}

\AtEndDocument{%
  \addtocounter{appendixpage}{\value{page}}%
  \immediate\write\@auxout{\string
    \newlabel{@lastappendixpage}{{\theappendixpage}{\thepage}}}%
}
%    \end{macrocode}
% \end{macro}
%
% \begin{macro}{\lastpage}
% \begin{macro}{\numberofpages}
%
%    
%
%    \begin{macrocode}
\newcommand{\lastpage}{%
  \label{lastpage}%
}

\newcommand{\numberofpages}{%
  \@ifundefined{r@lastpage}{0}{\pageref{lastpage}}%
}
%    \end{macrocode}
% \end{macro}
% \end{macro}
%
% \begin{macro}{\and}
%
%    |\author|-komento ei tunnu toimivan |\and|-komennon kanssa.
%    Määritellään |\and| uudelleen siten, että |\author| toimii ja että
%    sekä |\maketitle|n tekemä etusivu että |abstract|-ympäristön tekemä
%    tiivistelmäsivu näyttävät hyviltä.
%
%    \begin{macrocode}
\renewcommand{\and}{%
  \\ % Rivinvaihto.
  \hspace{1em}%
}
%    \end{macrocode}
% \end{macro}
%
%    Seuraavassa on vanhojen |tktltiki.sty|- ja
%    |tktlgradu.sty|-tyylitiedostojen sisältöjä muutettuna siten, että ne
%    kelpaavat tähän dokumenttiluokkatiedostoon.
%
%    Optio |openbib| on nykyään erillinen pakkaus.
%
%    \begin{macrocode}
\RequirePackage{openbib}
%    \end{macrocode}
%
%    Määritellään lähdeluettelon suomenkieliseksi otsikoksi \emph{Lähteet}
%    eikä \emph{Viitteet}.
%
%    \begin{macrocode}
\addto\captionsfinnish{\def\refname{Lähteet}}
%    \end{macrocode}
%
%    Määritellään \emph{Liitteet}-osion englanninkieliseksi nimeksi
%    \emph{Appendices} ja ruotsinkieliseksi nimeksi \emph{Bilagor}.
%
%    \begin{macrocode}
\addto\captionsenglish{\def\enclname{Appendices}}
\addto\captionsswedish{\def\enclname{Bilagor}}
%    \end{macrocode}
%
% \begin{environment}{thebibliography}
%
%    Määritellään |thebibliography|-ympäristö uudelleen niin, että
%    viitteiden tunnisteille jää tarpeeksi tilaa.
%
%    \begin{macrocode}
\renewenvironment{thebibliography}[1]
     {\section*{\refname
        \@mkboth{\uppercase{\refname}}{\uppercase{\refname}}
%    \end{macrocode}
%    Määritellään, että lähdeluettelon sivusta tulee merkintä
%    sisällysluetteloon.
%
%    \begin{macrocode}
        \addcontentsline{toc}{section}{\refname}}%
      \list{\@biblabel{\arabic{enumiv}}}%
%    \end{macrocode}
%    Alkuperäisessä määritelmässä rivi alla on
%
%    \begin{verbatim}
%          {\settowidth\labelwidth{\@biblabel{#1}}%
%    \end{verbatim}
%
%    \begin{macrocode}
           {\settowidth\labelwidth{\makebox[5em]{}}%
            \leftmargin\labelwidth
            \advance\leftmargin\labelsep
%    \end{macrocode}
%    Jätetty pois:
%
%    \begin{verbatim}
%            \if@openbib
%              \advance\leftmargin\bibindent
%              \itemindent -\bibindent
%              \listparindent \itemindent
%              \parsep \z@
%            \fi
%    \end{verbatim}
%
%    \begin{macrocode}
            \usecounter{enumiv}%
            \let\p@enumiv\@empty
            \renewcommand\theenumiv{\arabic{enumiv}}}%
%    \end{macrocode}
%    Jätetty pois:
%
%    \begin{verbatim}
%      \if@openbib
%        \renewcommand\newblock{\par}
%      \else
%    \end{verbatim}
%
%    \begin{macrocode}
        \renewcommand\newblock{\hskip .11em \@plus.33em \@minus.07em}%
%    \end{macrocode}
%    Jätetty pois:
%
%    \begin{verbatim}
%      \fi
%    \end{verbatim}
%
%    \begin{macrocode}
      \sloppy\clubpenalty4000\widowpenalty4000%
      \sfcode`\.=\@m}
     {\def\@noitemerr
       {\@latex@warning{Empty `thebibliography' environment}}%
      \endlist}
%    \end{macrocode}
% \end{environment}
%
%    Määritellään viitteen tunnisteen tekevä komento uudelleen niin,
%    ettei hakasulkeita tule ympärille. Alkuperäisessä määritelmässä rivi
%    on
%
%    \begin{verbatim}
%   \def\@biblabel#1{[#1]}
%    \end{verbatim}
%
%    \begin{macrocode}
\def\@biblabel#1{#1}
%    \end{macrocode}
%
%    Määritellään päivämäärä uudestaan 10.2.1998-muotoiseksi
%
%    \begin{macrocode}
\def\datefinnish{%
  \def\today{\number\day.\number\month.\number\year}}
%    \end{macrocode}
%
% \begin{macro}{\level}
% \begin{macro}{\maketitle}
%
%    Määritellään kansilehden formaatti kokonaan uudestaan.
%    Uudella komennolla |\level| voidaan kertoa kirjoitelman
%    laji, esim \emph{Laudaturtutkielma}. Määritellään myös kansilehden
%    kielikohtaiset termit. Jos dokumenttiluokka ladataan optiolla |gradu|,
%    otsikkosivun oikeaan yläkulmaan ei ilmesty työn arvosteluun liittyviä
%    kenttiä.
%
%    \begin{macrocode}
\newcommand{\level}[1]{\gdef\@level{#1}}
\level{}

\addto\captionsfinnish{%
  \def\dateofacceptance{hyväksymispäivä}%
  \def\grade{arvosana}%
  \def\instructor{arvostelija}%
  \def\uh{HELSINGIN YLIOPISTO}%
  \def\helsinki{Helsinki}%
  \def\ccs{ACM Computing Classification System (CCS):}%
}
\addto\captionsenglish{%
  \def\dateofacceptance{Date of acceptance}%
  \def\grade{Grade}%
  \def\instructor{Instructor}%
  \def\uh{UNIVERSITY OF HELSINKI}%
  \def\helsinki{Helsinki}%
  \def\ccs{ACM Computing Classification System (CCS):}%
}
\addto\captionsswedish{%
  \def\dateofacceptance{godk.datum}%
  \def\grade{vitsord}%
  \def\instructor{bedömare}%
  \def\uh{HELSINGFORS UNIVERSITET}%
  \def\helsinki{Helsingfors}%
  \def\ccs{ACM Computing Classification System (CCS):}%
}

\renewcommand{\maketitle}{\begin{titlepage}%
  \if@gradu
    \texttt{}
  \else
    \hspace*{85mm} \dateofacceptance \hspace{10mm} \grade\
  \fi

  \vspace{10mm}

  \if@gradu
    \texttt{}
  \else
    \hspace*{85mm} \instructor
  \fi

  \vspace*{70mm}

  {\large\bf \@title}

  \vspace{5mm}

  \@author

  \vfill

  \helsinki\ \@date

  \@level

  \uh\\
  \@department

  \end{titlepage}%
}
%    \end{macrocode}
% \end{macro}
% \end{macro}
%
% \begin{environment}{abstract}
% \begin{macro}{\numberofpagesinformation}
% \begin{macro}{\classification}
% \begin{macro}{\keywords}
% \begin{macro}{\faculty}
% \begin{macro}{\department}
% \begin{macro}{\subject}
% \begin{macro}{\depositeplace}
% \begin{macro}{\additionalinformation}
%
%    Määritellään abstraktin formaatti uudestaan. Komennoilla |\numberofpagesinformation|,
%    |\classification|, |\keywords|, |\faculty|, |\department|, |\subject|,
%    |\depositeplace| ja |\additionalinformation| voidaan kertoa vastaavat
%    tiedot tiivistelmälehteä varten. Abstraktisivulle ei tule ollenkaan
%    sivunumeroa.
%
%    \begin{macrocode}
\def\division{\char'057}

\newcommand{\faculty}[1]{\gdef\@faculty{#1}}
\newcommand{\department}[1]{\gdef\@department{#1}}
\newcommand{\subject}[1]{\gdef\@subject{#1}}
\newcommand{\depositeplace}[1]{\gdef\@depositeplace{#1}}
\newcommand{\additionalinformation}[1]{\gdef\@additionalinformation{#1}}
\newcommand{\numberofpagesinformation}[1]{\gdef\@numberofpagesinformation{#1}}
\newcommand{\classification}[1]{\gdef\@classification{#1}}
\newcommand{\keywords}[1]{\gdef\@keywords{#1}}

\faculty{}
\department{}
\subject{}
\depositeplace{}
\additionalinformation{}
\numberofpagesinformation{}
\classification{}
\keywords{}

\addto\captionsfinnish{%
  \faculty{Matemaattis-luonnontieteellinen tiedekunta}
  \department{Tietojenkäsittelytieteen laitos}%
  \subject{Tietojenkäsittelytiede}
}
\addto\captionsenglish{%
  \faculty{Faculty of Science}
  \department{Department of Computer Science}%
  \subject{Computer Science}
}
\addto\captionsswedish{%
  \faculty{Matematisk-naturvetenskapliga fakulteten}
  \department{Institutionen för datavetenskap}%
  \subject{Datavetenskap}
}
%    \end{macrocode}
%
%    Määritellään |@summary|-ympäristö, jota käytetään hyväksi hieman
%    jäljempänä |abstract|-ympäristön uudelleenmäärittelyssä. Tämä
%    ympäristö toimii siten, että |abstract|-ympäristössä
%    |\begin{@summary}|- ja |\end{@summary}|-komentojen väliin
%    sijoitetaan varsinainen tiivistelmäteksti, joka annetaan
%    \LaTeX-tiedostossa |\begin{abstract}|- ja |\end{abstract}|-komentojen
%    välissä. Tiivistelmätekstin perään liitetään |\classification|-komennon
%    parametrina annettujen aiheluokkien mukainen CR-luokitusteksti, mikäli
%    CCS-luokat on annettu |\classification|-komennolla.
%
%    \begin{macrocode}
\newsavebox{\@abstract}
\newenvironment{@summary}{
  \begin{lrbox}{\@abstract}
    \begin{minipage}[t]{5.95in}
      \setlength{\parskip}{2ex}
}{

      \if \@classification 
      \else
        \ccs\ \@classification
      \fi
    \end{minipage}
  \end{lrbox}
  \put(58, 650){\mbox{\usebox{\@abstract}}}
}
%    \end{macrocode}
%
%    Ja sitten lopultakin määritellään |abstract|-ympäristö uudelleen.
%
%    \begin{macrocode}

\def\abst@small{\fontsize{10}{12}\selectfont}
\def\abst@tiny{\fontsize{6}{7}\selectfont}

\renewenvironment{abstract}{%
\if@twoside
\begin{picture}(580,845)(71,-64)%
\else
\begin{picture}(580,845)(74,-64)%
\fi
\put(58,  784){\makebox(100, 8)[l]{\abst@small\@faculty}}
\put(289, 784){\makebox(100, 8)[l]{\abst@small\@department}}
\put(58,  761){\makebox(100, 8)[l]{\abst@small\@author}}
\put(58,  732){\parbox[l]{450pt}{\renewcommand{\baselinestretch}{.9}\abst@small\@title}}
\put(58,  704){\makebox(100, 8)[l]{\abst@small\@subject}}
\put(58,  681){\makebox(100, 8)[l]{\abst@small\@level}}
\put(212, 681){\makebox(100, 8)[l]{\abst@small\@date}}
\put(366, 681){\makebox(100, 8)[l]{\abst@small\@numberofpagesinformation}}
\put(58,  94) {\makebox(100, 8)[l]{\abst@small\@keywords}}
\put(58,  72) {\makebox(100, 8)[l]{\abst@small\@depositeplace}}
\put(58,  35) {\makebox(100, 8)[l]{\abst@small\@additionalinformation}}
\begin{@summary}\abst@small}
{\end{@summary}
%    \end{macrocode}
%    Iso kehys.
%
%    \begin{macrocode}
\put(53,30){\framebox(462,786){}}
%    \end{macrocode}
%    Vaakaviivat
%
%    \begin{macrocode}
\put(53,779){\line(1,0){462}}
\put(53,756.24){\line(1,0){462}}
\put(53,722.1){\line(1,0){462}}
\put(53,699.34){\line(1,0){462}}
\put(53,676.58){\line(1,0){462}}
\put(53,67){\line(1,0){462}}
\put(53,89.76){\line(1,0){462}}
\put(53,112.52){\line(1,0){462}}
%    \end{macrocode}
%    Pystyviivat. (Jostain syystä koodissa pitää olla tyhjä rivi
%    |picture|-ympäristön jälkeen. Muuten abstraktisivu menee pieleen.)
%
%    \begin{macrocode}
\put(284,779){\line(0,1){37}}
\put(207,676.58){\line(0,1){22.76}}
\put(361,676.58){\line(0,1){22.76}}
\put(58,809){\makebox(150,6)[l]{
\tiny Tiedekunta --- Fakultet --- Faculty}}
\put(289,809){\makebox(100,6)[l]{\abst@tiny Laitos --- Institution --- Department}}
\put(58,773){\makebox(100,5)[l]{\abst@tiny Tekij\"a --- F\"orfattare --- Author}}
\put(58,750){\makebox(100,5)[l]{\abst@tiny Ty\"on nimi --- Arbetets titel --- Title}}
\put(58,716){\makebox(100,5)[l]{\abst@tiny Oppiaine --- L\"aro\"amne --- Subject}}
\put(58,693){\makebox(100,5)[l]{\abst@tiny Ty\"on laji --- Arbetets art --- Level}}
\put(212,693){\makebox(100,5)[l]{\abst@tiny Aika --- Datum --- Month and year }}
\put(366,693){\makebox(100,5)[l]{\abst@tiny Sivum\"a\"ar\"a --- Sidoantal --- 
    Number of pages}}
\put(58,670){\makebox(100,5)[l]{\abst@tiny Tiivistelm\"a --- Referat --- Abstract}}
\put(58,106){\makebox(100,5)[l]{\abst@tiny Avainsanat --- Nyckelord --- Keywords}}
\put(58,83){\makebox(100,5)[l]{\abst@tiny S\"ailytyspaikka --- F\"orvaringsst\"alle
--- Where deposited}}
\put(58,61){\makebox(100,5)[l]{\abst@tiny Muita tietoja --- \"ovriga uppgifter
--- Additional information}}
\put(53,821){\makebox(100,8)[l]{\abst@small HELSINGIN YLIOPISTO --- HELSINGFORS
UNIVERSITET --- UNIVERSITY OF HELSINKI}}
\end{picture}

%    \end{macrocode}
%
%    Kommentoidut layoutasetuskomennot eivät toimi kaikissa
%    \LaTeX-toteutuksissa oikein. Siksi ne on kommentoitu pois ja sama
%    asia hoidetaan nykyisin antamalla |picture|-ympäristölle
%    \mbox{x- ja y-koordinaatit}. Sivunumeroa ei näytetä.
%    Mikäli abstraktisivua ei haluta laskea mukaan sivunumeroihin
%    (roomalaiset numerot), voi alla olevan kommenttimerkin poistaa
%    |\setcounter|-komennon edestä.
%
%    \begin{macrocode}
\pagestyle{empty}
%\if@twoside
%  \geometry{top=0.2cm, left=0.25cm, right=0.7cm, noheadfoot}
%\else
%  \geometry{top=0.2cm, left=0.9cm, noheadfoot}
%\fi
\newpage
%\setcounter{page}{1}
%\defaultsettings
}
%    \end{macrocode}
% \end{macro}
% \end{macro}
% \end{macro}
% \end{macro}
% \end{macro}
% \end{macro}
% \end{macro}
% \end{macro}
% \end{environment}
%
%    (Jostain syystä koodissa pitää olla tyhjä rivi ennen seuraavia
%    kommenttirivejä. Muuten abstraktisivu menee pieleen.)
%
%    \begin{macrocode}

% Yhdelle riville 75 merkkiä välilyönnit mukaan lukien:
% 3456789012345678901234567890123456789012345678901234567890123456789012345
%</luokka>
%    \end{macrocode}
%
% \section{Käyttöohjeet}
%
%
%    \begin{macrocode}
%<*ohjeet>
%%%%%%%%%%%%%%%%%%%%%%%%%%%%%%%%%%%%%%%%%%%%%%%%%%%%%%%%%%%%%%%%%%%%%%%%%%%%
%%                                                                         %
%% LaTeX-dokumenttiluokan tktltiki käyttöohje                              %
%% Mikael Puolakka 16.9.2002                                                %
%%                                                                         %
%%%%%%%%%%%%%%%%%%%%%%%%%%%%%%%%%%%%%%%%%%%%%%%%%%%%%%%%%%%%%%%%%%%%%%%%%%%%

\documentclass[gradu,emptyfirstpagenumber]{tktltiki}
\usepackage{url}

\newcommand{\filedate}{16.9.2002}

\newcommand{\koodi}[1]{\texttt{#1}}

\newcommand{\luokka}{\koodi{tktl\-tiki}}

\newcommand{\bibtyyli}{\koodi{tktl}}

\newcommand{\DescribeEnv}[1]{\marginpar{\texttt{#1}}}
\newcommand{\DescribeMacro}[1]{\marginpar{\texttt{#1}}}

\newcommand{\meta}[1]{#1}
\newcommand{\oarg}[1]{[<#1>]}
\newcommand{\marg}[1]{\{<#1>\}}

%%\newcommand{\koodikoko}{\small}
\newcommand{\koodikoko}{\footnotesize}

%%\newcommand{\BibTeX}{BibTeX}
\def\BibTeX{{\rmfamily B\kern-.05em%
  \textsc{i\kern-.025em b}\kern-.08em%
  T\kern-.1667em\lower.7ex\hbox{E}\kern-.125emX}}

\hyphenation{kä-si-kir-joi-tus-tie-dos-to
             kä-si-kir-joi-tus-tie-dos-to-jen
             kä-si-kir-joi-tus-tie-dos-ton}

\begin{document}

\title{Dokumenttiluokan \luokka\ käyttöohje}
\author{Mikael Puolakka}
\date{\filedate}
\faculty{Matemaattis-luonnontieteellinen}
\department{Tietojenkäsittelytieteen laitos}
\level{Käyttöohje}
\additionalinformation{Tämä on dokumenttiluokan \luokka\ käyttöohje.}
\numberofpagesinformation{\numberofpages\ sivua}
\classification{A.1, I.7.m}
\keywords{käyttöohje, \LaTeX-dokumenttiluokka, \BibTeX-tyyli}

\maketitle

\begin{abstract}
Dokumenttiluokka \luokka\ on tarkoitettu tukemaan Helsingin yliopiston
Tietojenkäsittelytieteen laitoksen alempien opinnäytteiden ja harjoitusten
laatimista. Luokka soveltuu siten käytettäväksi tieteellisen kirjoittamisen
kurssille, ohjelmistotuotantoprojekteihin ja pro gradu -tutkielmiin. Luokka
määrittelee sitä käyttävälle dokumentille tietyn ulkoasun ja tarjoaa
käyttöliittymän dokumentin rakenteen hallitsemiseksi.
\end{abstract}

\mytableofcontents

\section{Johdanto}
Tämä dokumentti on \LaTeX -dokumenttiluokan \luokka\ käyttöohje.
Dokumentti pyrkii antamaan kuvan siitä, miten
\LaTeX-käsikirjoitustiedostoja\footnote{Käsikirjoitustiedosto on tiedosto,
joka sisältää ohjeet dokumentin valmistamiseksi. Käsikirjoitustiedosto siis
määrittää dokumentin.} rakennetaan käyttäen luokkaa \luokka\ ja millä
tavalla luokan \luokka\ käyttö vaikuttaa sen dokumentin ulkoasuun, joka
saadaan \LaTeX-käsikirjoitustiedoston tulosteena.

Tämä dokumentti kuvailee dokumenttiluokan \luokka\ käyttöliittymän eli ne
komennot ja ympäristöt, jotka luokka \luokka\ tarjoaa sitä käyttävien
\LaTeX-käsi\-kir\-joi\-tus\-tie\-dos\-to\-jen rakennuspalasiksi. Noista
rakennuspalasista muodostuu laadittavan dokumentin looginen rakenne.
Tämä dokumentti kertoo myös, miten käytetyt komennot ja ympäristöt
vaikuttavat laadittavan dokumentin ulkoasuun ja miten ylipäänsä luokan
\luokka\ käyttö vaikuttaa laadittavan dokumentin ulkoasuun ja rakenteeseen.
Jotta tämän käyttöohjeen lukijalle syntyisi visuaalinen mielikuva luokan
\luokka\ määrittämästä dokumentin tyypistä, on tämä dokumentti tuotettu
käyttämällä kyseessä olevaa luokkaa.

Dokumentin seuraavat kaksi lukua käsittelevät dokumenttiluokan \luokka\
käyttöä. Luku~\ref{kaytto} on tarkoitettu pikaohjeeksi, josta saa yleiskuvan
luokan \luokka\ kaytosta. Luku~\ref{pitkakaytto} on taas tarkoitettu
yksityiskohtaiseksi ohjeeksi siitä, miten luokkaa \luokka\ käyttäen
rakennetaan \LaTeX-käsikirjoitustiedosto. Luku~\ref{pitkakaytto} on jaettu
siten alilukuihin, että kukin aliluku kuvaa yhden osan rakennettavasta
\LaTeX-käsikirjoitustiedostosta. Lisäksi aliluvut etenevät siinä
järjestyksessä kuin niiden kuvailemat osat esiintyvät useimmissa
dokumenttiluokkaa \luokka\ käyttävissä käsikirjoitustiedostoissa.

Mainittakoon selvyyden vuoksi, että jatkossa \emph{dokumentti} tarkoittaa
sitä tulostus- ja selailukelpoista dokumenttia, joka saadaan tulosteena kun
käsikirjoitustiedosto syötetään \LaTeX-ohjelmalle.

\section{Lyhyt versio dokumenttiluokan \luokka\ käytöstä}
\label{kaytto}

Kuvassa~\ref{runko} esitetty \LaTeX-käsikirjoitustiedoston runko toimii
lyhyenä versiona
dokumenttiluokan \luokka\ käyttöohjeesta. Se esittelee luokan \luokka\
keskeiset komennot ja ympäristöt sekä niiden käyttötarkoitukset. Se ei
kuitenkaan ole täydellinen malli käsikirjoitustiedostosta, sillä se ei
esimerkiksi sisällä ollenkaan \verb|\cite|-komennolla tehtyjä
lähdeviittauksia eikä \verb|\label|- ja \verb|\ref|-komennoilla tehtyjä
ristiviittauksia.

Tavallisesti \LaTeX-käsikirjoitustiedostot ovat \koodi{TEX}-päätteisiä,
joten kuvan~\ref{runko} tiedostorungon voisi tallentaa esimerkiksi nimellä
\koodi{esim.tex}.

\begin{figure}
\centering
{\koodikoko
\begin{verbatim}
    \documentclass{tktltiki}
    \usepackage{url}

    \begin{document}

    \title{Otsikko}
    \author{Tekijä}
    \date{\today}
    \level{Pro gradu -tutkielma}

    \maketitle

    \doublespacing

    \faculty{Matemaattis-luonnontieteellinen}
    \department{Tietojenkäsittelytieteen laitos}
    \subject{Tietojenkäsittelytiede}
    \depositeplace{Tietojenkäsittelytieteen laitoksen kirjasto,
                   sarjanumero C-2004-X}
    \additionalinformation{}
    \numberofpagesinformation{\numberofpages\ sivua +
                              \numberofappendixpages[100] liitesivua}
    \classification{A.1 [Introductory and Survey] \\
                    I.7.m [Document and Text Processing]: Miscellaneous}
    \keywords{avainsana}

    \begin{abstract}
    Tiivistelmä
    \end{abstract}

    \mytableofcontents

    \section{Eka luku}

    Tämä yksi luku on oikeastaan kaksi lukua: ensimmäinen ja viimeinen.

    \subsection{Ekan luvun eka aliluku}

    Viimeiset sanat.

    \bibliographystyle{tktl}
    \bibliography{munBibTeXtiedosto}

    \lastpage
    \appendices

    \internalappendix{\theappendix}{Eka liite}

    Liitetekstiä.

    \internalappendix{2}{Toka liite}

    Liitetekstiä. Ehkä kuviakin.

    \externalappendix{\theappendix}{Ohjelmalistaus joka ei sisälly
                                    \LaTeX-tiedostoon}
    \end{document}
\end{verbatim}
}
\caption{\LaTeX-käsikirjoitustiedoston runko} \label{runko}
\end{figure}

\section{Pitkä versio dokumenttiluokan \luokka\ käytöstä}
\label{pitkakaytto}

Tämä luku on pitkä versio dokumenttiluokan \luokka\ käyttöohjeesta. Tämän
luvun aliluvuissa käsitellään tarkemmin luvussa~\ref{kaytto}
esitellyt luokan \luokka\ komennot sekä monia muita hyödyllisiä
\LaTeX-komentoja.

\subsection{Aloitus}
\label{aloitus}

Kun käsikirjoitustiedosto syötetään \LaTeX-ohjelmalle, tulosteena saadaan
valmis dokumentti. Tämän valmiin dokumentin yleisluonne määritellään
valitsemalla käsikirjoitustiedostossa dokumenttiluokaksi luokka \luokka.
Toisin sanoen luokka \luokka\ määrittää valmiin dokumentin tyypin.

Luokka \luokka\ määrää dokumentin eri osien ulkoasun ja sen, mitä kaikkia
tietoja dokumentin eri osiin tulee. Esimerkiksi otsikkosivu ja
tiivistelmäsivu ovat luokkaa \luokka\ käyttävässä dokumentissa aivan
erinäköisiä kuin esimerkiksi luokkaa \koodi{report} käyttävässä
dokumentissa.

Dokumentin yleispiirteitä voidaan edelleen täsmentää optioilla. Luokassa
\luokka\ määritellyt optiot on lueteltu alla olevassa listassa.
Vaihtoehdot, joista enintään yksi pitäisi valita, on eroteltu symbolilla
\koodi{|}.

\begin{description}

\item[\koodi{gradu}] \texttt{} \\
Saa aikaan sen, ettei valmiin dokumentin otsikkosivun oikeaan yläkulmaan
ilmesty työn arvosteluun liittyviä sanoja \emph{\dateofacceptance},
\emph{\grade} ja \emph{\instructor}.

\item[\koodi{emptyfirstpagenumber}] \texttt{} \\
Jättää sivunumeron pois dokumentin varsinaisen tekstin ensimmäiseltä
sivulta. Varsinainen teksti alkaa tavallisesti dokumentissa
sisällysluettelon jälkeen ja sen sivunumerointiin käytetään arabialaisia
numeroja.

\item[\koodi{finnish} | \koodi{swedish} | \koodi{english}] \texttt{} \\
Valitsee dokumentin kielen. Optio \koodi{finnish} valitsee dokumentin
kieleksi suomen, optio \koodi{swedish} ruotsin ja optio \koodi{english}
englannin. Oletuksena on optio \koodi{finnish}. Dokumentin kieli vaikuttaa
siten dokumenttiin, että kaikki automaattisesti tuotetut otsikot ja kentät
tuotetaan valitulla kielellä. Lisäksi dokumentin tavutuksessa käytetään
valitun kielen mukaista tavutusta.

Mikäli dokumenttiin sisältyy vieraskielisiä sanoja tai lauseita,
vieraskielisen osuuden tavutussäännöt voidaan muuttaa esimerkiksi
\koodi{Babel}-pakkauksen komennolla \verb|\foreignlanguage|. Komennon
ensimmäisenä parametrina annetaan kielen nimi, jonka sääntöjen mukaisesti
toisena parametrina annettu teksti ladotaan. Vieraan kielen käyttäminen
edellyttää kuitenkin sitä, että \TeX\ lataa kielen tavutussäännöt.
Paikallisen \TeX-järjestelmän lataamat kielien tavutussäännöt pitäisi saada
selville \TeX in tuottamasta lokitiedostosta, jossa voi lukea esim.\
\foreignlanguage{english}{\texttt{Babel <v3.7h> and hyphenation patterns
for american, french, german, ngerman, finnish, italian, swedish,
nohyphenation, loaded.}} Tarkemmat ohjeet monikielisistä dokumenteista
löytyvät \koodi{Babel}-pakkauksen dokumentaatiosta.

\end{description}

Yllä mainittujen optioiden lisäksi luokalle \luokka\ kelpaavat kaikki ne
optiot, jotka kelpaavat standardiluokalle \verb|article|. Tämä johtuu
siitä, että \luokka\ perustuu \koodi{article}-luokkaan. Alla olevassa
listassa on lueteltu \koodi{article}-luokan tunnistamat optiot.

\begin{description}
\item[\koodi{10pt} | \koodi{11pt} | \koodi{12pt}] \texttt{} \\
Valitsee dokumentin peruskirjasimen koon. Luokan \luokka\ oletuksena on
\koodi{12pt}, joka valitsee 12 pisteen kirjasinkoon.

\item[\koodi{letterpaper} | \koodi{legalpaper} | \koodi{executivepaper} |
      \koodi{a4paper} | \koodi{a5paper} | \koodi{b5paper}] \texttt{} \\
Määrittää dokumentin paperikoon. Luokan \luokka\ oletuksena on
\koodi{a4paper}.

\item[\koodi{landscape}] \texttt{} \\
Saa aikaan sen, että dokumentti muokataan sopivaksi vaakatasotulostusta
varten valitulle paperikoolle.

\item[\koodi{final} | \koodi{draft}] \texttt{} \\
Optio \koodi{draft} saa aikaan sen, että dokumentissa oikean marginaalin
yli menevät ylipitkät rivit merkataan mustilla laatikoilla. Optio
\koodi{final}, joka ei merkitse ylipitkiä rivejä, on oletuksena.

\item[\koodi{oneside} | \koodi{twoside}] \texttt{} \\
Muokkaa dokumentista sopivan joko yksipuoliseen tai kaksipuoliseen
tulostukseen. Oletuksena on yksipuolinen tulostus (\koodi{oneside}).

\item[\koodi{onecolumn} | \koodi{twocolumn}] \texttt{} \\
Määrittää sen, ladotaanko teksti dokumentissa yhdelle palstalle
(\koodi{onecolumn}) vai kahdelle (\koodi{twocolumn}). Oletuksena on
\koodi{onecolumn}.

\item[\koodi{notitlepage} | \koodi{titlepage}] \texttt{} \\
Optio \koodi{titlepage} saa aikaan sen, että \verb|\maketitle|-komento
tekee erillisen otsikkosivun ja \koodi{abstract}-ympäristö laittaa
tiivistelmän erilliselle sivulle. Luokan \luokka\ oletuksena on
\koodi{titlepage}.

\item[\koodi{leqno}] \texttt{} \\
Asettaa matemaattisten kaavojen numeroinnnin vasempaan reunaan
\koodi{equation}- ja \koodi{eqnarray}-ympäristöissä.

\item[\koodi{fleqn}] \texttt{} \\
Latoo matemaattiset kaavat vasempaan reunaan tasattuina.

\end{description}

Dokumentin luokka ja optiot ilmoitetaan dokumentin määrittävän
käsikirjoitustiedoston ensimmäisellä rivillä komennolla
\begin{verbatim}
    \documentclass[<optiot>]{tktltiki}
\end{verbatim}
jossa optiot siis toimivat lisämääreinä dokumenttiluokalle \luokka\ ja
erotetaan toisistaan pilkulla. Tavallisesti \LaTeX-käsikirjoitustiedostojen
tiedostopääte on \koodi{TEX}, joten muokattavan käsikirjoitustiedoston
nimi voisi olla esimerkiksi \koodi{malli.tex}.

Kuten jo aiemmin mainittiin, luokka \luokka\ määrittää dokumentin tyypin.
Luokan \luokka\ mukaisen dokumentin ominaisuuksiin kuuluu alla olevassa
listassa mainitut ominaisuudet.

\begin{itemize}
\item
Helsingin yliopiston Tietojenkäsittelytieteen laitoksen raporttityylin
mukaisesti muotoillut otsikkosivu eli kansilehti, tiivistelmäsivu ja
lähdeluettelo.
\item
Sopivan kokoiset marginaalit.
\item
Ei sisennyksiä kappaleiden aluissa.
\item
Hieman tyhjää tilaa kappaleiden välissä.
\item
Sivunumerointi roomalaisilla numeroilla sisällysluettelosivuilla ja
mahdollisilla muilla tavallisilla sivuilla ennen sisällysluetteloa, minkä
jälkeen sivunumerointi arabialaisilla numeroilla, mikäli sisällysluettelon
luontiin käytetty komentoa \verb|\mytableofcontents|.
\item
Sivunumerointi sivun oikeassa yläkulmassa.
\end{itemize}

Ylimääräisiä komentoja ja dokumentin ulkoasuun vaikuttavia
tyylimäärittelyjä voidaan ottaa käyttöön pakkausten avulla. Pakkaus otetaan
käyttöön kirjoittamalla komento
\begin{verbatim}
    \usepackage[<optiot>]{<pakkaus>}
\end{verbatim}
Optiot ovat tässä niitä optioita, jotka komennolla käyttöön otettu pakkaus
tunnistaa. Kunkin pakkauksen tunnistamat optiot on mainittu pakkauksen
dokumentaatiossa. Jos optioita ei käytetä, komento on yksinkertaisesti
\begin{verbatim}
    \usepackage{<pakkaus>}
\end{verbatim}
Komento \verb|\usepackage| voi esiintyä käsikirjoitustiedostossa vasta
\verb|\documentclass|-komennon jälkeen.

Dokumenttiluokan \luokka\ kanssa suositellaan käytettävän seuraavia
pakkauksia:
\begin{description}
\item[\koodi{psfig}] \texttt{} \\
Pakkaus kuvien liittämistä varten.
\item[\koodi{graphicx}] \texttt{} \\
Edellistä helppokäyttöisempi pakkaus kuvien liittämistä varten.
\item[\koodi{subfigure}] \texttt{} \\
Pakkaus vierekkäisten kuvien liittämistä varten.
\item[\koodi{url}] \texttt{} \\
WWW-osoitteiden ladontapakkaus.
\end{description}

Taulukossa~\ref{pakkaukset} on lueteltu ne pakkaukset optioineen, jotka
dokumenttiluokka \luokka\ ottaa käyttöön automaattisesti. Näitä pakkauksia
ei siis tarvitse ottaa erikseen käyttöön \verb|\usepackage|-komennolla,
mikäli niiden tarjoamia komentoja ja dokumentin ulkoasuun vaikuttavia
tyylimäärittelyjä tarvitaan dokumentin laadinnassa.

\begin{table}[h]
\begin{center}
\begin{tabular}{l|l}
\textbf{pakkaus} & \textbf{optiot} \\ \hline
\koodi{babel}    & \koodi{finnish} tai \koodi{english} tai
                   \koodi{swedish} \\
\koodi{fontenc}  & \koodi{T1} \\
\koodi{geometry} & - \\
\koodi{inputenc} & \koodi{latin9} \\
\koodi{latexsym} & - \\
\koodi{openbib}  & - \\
\end{tabular}
\caption{Dokumenttiluokan \luokka\ käyttöönottamat pakkaukset}
\label{pakkaukset}
\end{center}
\end{table}

Kun käsikirjoitustiedostossa on määritelty dokumentin tyyppi, otettu
käyttöön halutut pakkaukset ja määritelty mahdolliset omat komennot ja
ympäristöt\footnote{Katso lisätietoja esimerkiksi \emph{Pitkänpuoleinen
johdanto \LaTeXe:n käyttöön} -oppaasta, joka löytyy WWW-osoitteesta
\url{ftp://ftp.funet.fi/
pub/TeX/CTAN/info/
lshort/finnish/lyhyt2e.pdf}}, varsinainen dokumentin sisältö aloitetaan
kirjoittamalla käsikirjoitustiedostoon seuraava
komento:\footnote{Vastaavasti dokumentti päätetään komennolla
$\mathtt{\backslash}$\koodi{end\{document\}}}
\begin{verbatim}
    \begin{document}
\end{verbatim}

\subsection{Otsikkosivu}

%%\DescribeMacro{\maketitle}
Otsikkosivu eli kansilehti tuotetaan dokumenttiin kirjoittamalla
käsikirjoitustiedostoon komento \verb|\maketitle|.
%%\DescribeMacro{\title}
%%\DescribeMacro{\author}
%%\DescribeMacro{\date}
%%\DescribeMacro{\level}
Otsikkosivun ja osa tiivistelmäsivun muuttuvista tiedoista ilmoitetaan
käsikirjoitustiedostossa komennoilla \verb|\title| (työn nimi),
\verb|\author| (tekijä), \verb|\date| (aika, ei pakollinen) ja
\verb|\level| (työn laji) ennen otsikkosivun tuottamista. Dokumentissa
otsikkosivu tulee automaattisesti omalle sivulleen ilman sivunumerointia.

\subsection{Tiivistelmäsivu}

%%\DescribeEnv{abstract}
Kun otsikkosivun tuottava \verb|\maketitle|-komento on kirjoitettu
käsikirjoitustiedostoon, dokumenttiin voidaan tehdä tiivistelmäsivu
\verb|abstract|-ympäristön avulla. Osa tiivistelmäsivun muuttuvista
tiedoista annettiin aiemmin mainituilla komennoilla ennen
\verb|\maketitle|-komentoa.
%%\DescribeMacro{\faculty}
%%\DescribeMacro{\department}
%%\DescribeMacro{\subject}
%%\DescribeMacro{\depositeplace}
%%\DescribeMacro{\additionalinformation}
%%\DescribeMacro{\classification}
%%\DescribeMacro{\keywords}
Loput tiivistelmäsivun muuttuvista tiedoista annetaan komennoilla
\verb|\faculty| (tiedekunta/osasto), \verb|\department| (laitos),
\verb|\subject| (oppiaine), \verb|\depositeplace| (säilytyspaikka),
\verb|\additionalinformation| (muita tietoja), \verb|\classification|
(aiheluokat) ja \verb|\keywords| (avainsanat).

%%\DescribeMacro{\numberofpagesinformation}
Komennon \verb|\numberofpagesinformation| parametrina annetaan varsinaisen
dokumentin rungon\footnote{Tavallisesti sisällysluettelon ja mahdollisten
liitteiden välinen osa dokumentista} sivumäärä ja liitteiden sivumäärä.

%%\DescribeMacro{\lastpage}
%%\DescribeMacro{\numberofpages}
%%\DescribeMacro{\numberofappendixpages}
Sivumäärien laskemiseen on olemassa komennot \verb|\lastpage|,
\verb|\numberofpages| ja \verb|\numberofappendixpages|. Komennolla
\verb|\lastpage| ilmoitetaan, että käsikirjoitustiedostossa päättyy
varsinaisen dokumentin rungon käsittely. On siis mielekästä kirjoittaa
\verb|\lastpage|-komento käsikirjoitustiedostoon juuri ennen
\verb|\appendices|-ko\-men\-toa tai --- jos liitteitä ei ole --- ennen
\verb|\end{document}|-komentoa.

Komento \verb|\numberofpages| tulostaa kutsukohtaansa sen sivun
sivunumeron, jolla komentoa \verb|\lastpage| on kutsuttu, tai numeron 0,
jos komentoa \verb|\lastpage| ei ole kutsuttu missään dokumentin kohdassa.
Käytännössä komennon \verb|\numberofpages| ilmoittama sivunumero on samalla
varsinaisen dokumentin sivumäärä, mikäli komentoa \verb|\lastpage| on
kutsuttu oikeassa kohdassa ja sivumäärään halutaan sisällyttää ainoastaan
sisällysluettelon ja mahdollisten liitteiden väliset sivut. Komento
\verb|\numberofappendixpages| tulostaa dokumentin liitteiden sivumäärän,
mikäli liitteisiin liittyvissä toimenpiteissä on käytetty luokan \luokka\
tarjoamia liitekomentoja \verb|\appendices| ja \verb|\internalappendix|.
Komennon \verb|\numberofappendixpages| valinnaisella parametrilla voidaan
ilmoittaa dokumentin ulkopuolisten liitteiden yhteenlaskettu sivumäärä.

\begin{sloppypar}
Nyt dokumentin sivumäärä voidaan ilmoittaa
\verb|\numberofpagesinformation|-komennolle seuraavasti, kun dokumentin
viimeisellä varsinaisella sivulla kutsutaan \verb|\lastpage|-komentoa:
\verb|\numberofpagesinformation{\numberofpages\|\verb|sivua +|
\verb|\numberofappendixpages[100] liitesivua}|. Komennon
\verb|\numberofappendixpages| valinnaisella parametrilla (tässä~100)
ilmoitetaan dokumentin ulkopuolisten liitteiden yhteenlaskettu sivumäärä.
Jos dokumentin ulkopuolisia liitteitä ei ole, riittää kirjoittaa
\verb|\numberofappendixpages\ |, jossa \verb*|\ | sijoittaa välilyönnin
sivumäärän perään.
\end{sloppypar}

Kun kaikki tiivistelmäsivulle tulevista muuttuvista tiedoista on
kirjoitettu, tiivistelmäsivu luodaan dokumenttiin kirjoittamalla
käsikirjoitustiedostoon
\begin{verbatim}
    \begin{abstract}
    Tiivistelmäteksti, joka tulee Tiivistelmä --- Referat --- Abstract
    -otsikon alle tiivistelmäsivulle.
    \end{abstract}
    \end{verbatim}
Tämä luo omalla, sivunumeroimattomalla sivullaan olevan tiivistelmäsivun
annetuilla tiedoilla.

\subsection{Sisällysluettelo}

%%\DescribeMacro{\mytableofcontents}
\begin{sloppypar}
Sisällysluettelo luodaan kirjoittamalla käsikirjoitustiedostoon komento
\verb|\mytableofcontents|. Tämä tuottaa dokumenttiin sisällysluettelon,
jonka sivunumerointiin käytetään roomalaisia numeroita ja joka aloittaa
loppuessaan uuden sivun, mistä lähtien sivunumerointiin käytetään
arabialaisia numeroita.
\end{sloppypar}

Jos sisällysluettelon jälkeen on vielä tarkoitus tulla sivuja, joilla
käytetään roomalaista sivunumerointia, sisällysluettelo on syytä luoda
\verb|\tableofcontents|-komennolla. Tämä ainoastaan tuottaa dokumenttiin
sisällysluettelon eikä siis vaikuta sivunumerointiin tai aloita loppuessaan
uutta sivua. Sivunumeroinnin voi vaihtaa sitten haluamassaan kohdassa
arabialaiseksi kirjoittamalla käsikirjoitustiedostoon
\begin{verbatim}
    \newpage
    \pagenumbering{arabic}
\end{verbatim}
mikä aloittaa uuden sivun sivunumerolla 1 ja arabialaisella
sivunumeroinnilla.

\subsection{Varsinainen teksti}
\label{teksti}

Otsikkosivun, tiivistelmäsivun ja sisällysluettelon jälkeen dokumentissa
alkaa tavallisesti varsinainen tekstiosa. Kun käsikirjoitustiedostossa
käytetään sisällysluettelon luontiin \verb|\mytableofcontents|-komentoa,
dokumentissa alkaa automaattisesti uusi sivu sisällysluettelon jälkeen
ilman lisäkomentojen kirjoittamista.

%%\DescribeMacro{\section}
%%\DescribeMacro{\subsection}
%%\DescribeMacro{\subsubsection}
Varsinaisen tekstiosan kirjoittamiseen ei välttämättä tarvita muita
komentoja kuin \verb|\section|, \verb|\subsection| ja
\verb|\subsubsection|. Komento \verb|\section| aloittaa pääluvun ja sen
parametrina annetaan pääluvun otsikko. Komennot \verb|\subsection| ja
\verb|\subsubsection| ovat alilukujen aloittamista varten siten, että
\verb|\subsection| aloittaa dokumentissa ensimmäisen tason aliluvun ja
\verb|\subsubsection| toisen tason aliluvun. Näille komennoille annetaan
niin ikään parametrina aloitettavan luvun otsikko.

%%\DescribeMacro{\label}
%%\DescribeMacro{\ref}
%%\DescribeMacro{\pageref}
%%\DescribeMacro{\cite}
Erittäin hyödyllisiä komentoja käsikirjoitustiedostoa kirjoitettaessa ovat
\verb|\label|, \verb|\ref| ja \verb|\pageref|, joita käytetään
ristiviittauksissa, sekä \verb|\cite|, jota käytetään viitattaessa
lähteisiin.

Kirjallisissa töissä on usein ristiviittauksia kuviin, taulukoihin,
tiettyihin tekstin osiin, jne. \LaTeX\ tarjoaa ristiviittaamiseen
komennot \verb|\label{<tunniste>}|,\linebreak \verb|\ref{<tunniste>}| ja
\verb|\pageref{<tunniste>}|, joissa tunniste on käyttäjän valitsema nimi
viitattavalle kohteelle. Valmiissa dokumentissa \LaTeX\ korvaa
\verb|\ref|-komen\-non sen otsikon, alaotsikon, kuvan, taulukon tai teoreeman
numerolla, jonka perässä käsikirjoitustiedostossa on vastaavanniminen
\verb|\label|-komento. Komento \verb|\pageref| tulostaa sen sivun numeron,
jossa vastaava \verb|\label|-komento on. Juuri ristiviittausten saamiseksi
ajan tasalle käsikirjoitustiedosto on ajettava \LaTeX:n läpi ainakin
kahdesti.\footnote{Ohjelmien käyttöä käsitellään luvussa~\ref{ohjelmat}}

Esimerkiksi teksti \emph{Viittaus tähän alaotsikkoon \label{sec:this}
näyttää tältä: ''katso osiota~\ref{sec:this} sivulla~\pageref{sec:this}.''}
saatiin aikaan tähän käyttöohjeeseen kirjoittamalla
käsikirjoitustiedostoon rivit:
\begin{verbatim}
    Viittaus tähän alaotsikkoon \label{sec:this}
    näyttää tältä: ''katso osiota~\ref{sec:this}
    sivulla~\pageref{sec:this}.''
\end{verbatim}
Luonnollisesti soveliaampi paikka komennolle \verb|\label{sec:this}| olisi
ollut heti tämän aliluvun otsikon alla.

%%\DescribeEnv{thebibliography}
Bibliografia eli lähdeluettelo voidaan tehdä käsin käsikirjoitustiedostoon
sijoitettavan \koodi{thebibliography}-ympäristön avulla tai
\BibTeX-ohjelmalla, joka kirjoittaa \koodi{thebibliography}-ympäristön
sisältöineen \koodi{BBL}-päätteiseen tiedostoon. Tästä aiheesta kerrotaan
enemmän luvussa~\ref{lahdeluettelo}. Tässä vaiheessa on tärkeintä tietää,
kuinka kirjoihin, artikkeleihin tai muihin lähteisiin viitataan
varsinaisessa tekstissä.

%%\DescribeMacro{\bibitem}
Jokainen lähde lähdeluettelon tuottavassa
\koodi{thebibliography}-ympäristössä alkaa komennolla
\verb|\bibitem{<tunniste>}|.
%%\DescribeMacro{\cite}
Viittaus tunnisteen määrittämään lähteeseen tapahtuu komennolla
\verb|\cite{<tunniste>}|. Kun viitataan lähteeseen, joka alkaa\linebreak
\koodi{thebibliography}-ympäristössä komennolla \verb|\bibitem{mjpuolak}|,
kirjoitetaan esimerkiksi
\begin{verbatim}
    WWW-lähteitä ja niihin viittaamista käsitellään erillisessä
    ohjeessa~\cite{mjpuolak}.
\end{verbatim}

%%\DescribeMacro{\singlespacing}
%%\DescribeMacro{\onehalfspacing}
%%\DescribeMacro{\doublespacing}
\begin{sloppypar}
Rivivälityksen muuttamiseen on olemassa komennot \verb|\singlespacing|,
\verb|\onehalfspacing| ja \verb|\doublespacing|.
\end{sloppypar}

\onehalfspacing
Komennolla \verb|\onehalfspacing| rivivälitys muutetaan 1.5-riviväliksi. \\
(Tämän ja edellisen rivin välissä on käytetty 1.5-riviväliä.)

\doublespacing
Komennolla \verb|\doublespacing| rivivälitys muutetaan puolestaan
2.0-riviväliksi. \\
(Tämän ja edellisen rivin välissä on käytetty 2.0-riviväliä.)

\singlespacing
Komennolla \verb|\singlespacing| onnistuu paluu 1.0-riviväliin. \\
(Tämän ja edellisen rivin välissä on käytetty 1.0-riviväliä.)

\subsubsection{Suuret projektit}

%%\DescribeMacro{\input}
%%\DescribeMacro{\include}
Ryhmätöitä aloitettaessa kannattaa koota pääkäsikirjoitustiedoston rungoksi
vain välttämättömät asettelut ja tuoda sisältö mukaan \verb|\input|- tai
\verb|\include|-komen\-noil\-la. Näin käsikirjoitus voidaan jakaa useampaan
tiedostoon isoja dokumentteja tehtäessä.

Komento \verb|\input{<tiedosto>}| ottaa yksinkertaisesti mukaan tiedoston
sisällön siihen paikkaan, jossa sitä kutsutaan käsikirjoitustiedostossa.
Tiedoston nimi voi olla täydellinen nimi päätteineen tai vain nimen
ensimmäinen osa, jolloin \LaTeX\ käyttää oletuspäätteenä päätettä
\koodi{.tex}. Komentoa \verb|\input{<tiedosto>}| voidaan kutsua missä
tahansa käsikirjoitustiedoston kohdassa.

Komentoa \verb|\include{<tiedosto>}| voidaan käyttää ainoastaan itse
tekstiosassa eli \verb|\begin{document}|- ja
\verb|\end{document}|-komentojen välissä lisäämään toisen tiedoston
sisältö. Komentoa ei voi käyttää tiedostossa, joka luetaan jonkun muun
\verb|\include|-komennon toimesta. \LaTeX\ aloittaa uuden sivun, ennen kuin
se alkaa käsitellä \verb|\include|-komennolla lisätyn tiedoston sisältöä.

%%\DescribeMacro{\includeonly}
Komennolla \verb|\includeonly{tiedosto1,tiedosto2,...}| \LaTeX\ ohjataan
lukemaan ainoastaan komennon parametrina annetut tiedostot. Komentoa voi
käyttää ainoastaan käsikirjoitustiedoston esittelyosassa.\footnote{Se osa
käsikirjoitustiedostosta, joka edeltää
$\mathtt{\backslash}$\koodi{begin\{document\}}-komentoa.} Kun tämä komento
on kirjoitettu esittelyosaan, suoritetaan vain
\verb|\includeonly|-komennon listassa mainittujen tiedostojen
\verb|\include|-komennot.

\begin{sloppypar}
Kuvassa~\ref{paakasis} on yksinkertainen esimerkki komentoja \verb|\input|,
\verb|\include| ja \verb|\includeonly| hyödyntävän
pääkäsikirjoitustiedoston sisällöstä.
\end{sloppypar}
\begin{figure}
\centering
{\koodikoko
\begin{verbatim}
    \documentclass{tktltiki}

    \includeonly{ekaeka,ekaliite}

    \begin{document}

    \title{Otsikko}
    \author{Tekijä}
    \date{\today}

    \maketitle

    \begin{abstract}
    \input{tiivistelma}
    \end{abstract}

    \mytableofcontents

    \section{Eka luku}

    \input{eka}

    \subsection{Ekan luvun eka aliluku}

    \include{ekaeka}

    \subsection{Ekan luvun toinen aliluku}

    \include{ekatoka}

    \bibliographystyle{tktl}
    \bibliography{munBibTeXtiedosto}

    \lastpage
    \appendices

    \internalappendix{\theappendix}{Eka liite}

    \include{ekaliite}

    \internalappendix{2}{Toka liite}

    \include{tokaliite}

    \end{document}
\end{verbatim}
}
\caption{Esimerkki pääkäsikirjoitustiedostosta} \label{paakasis}
\end{figure}

Kuvan~\ref{paakasis} esimerkissä komento \verb|\includeonly{ekaeka,ekaliite}|
merkitsee sitä,
että ainoastaan tiedostot \koodi{ekaeka} ja \koodi{ekaliite} luetaan
mukaan niistä, jotka liitetään \verb|\include|-komennolla. Tiedostojen
\koodi{ekatoka} ja \koodi{tokaliite} sisältöjä ei siis oteta mukaan
valmiiseen dokumenttiin eikä niitä koskaan luetakaan. Sen sijaan tiedostot
\koodi{tiivistelma} ja \koodi{eka} liitetään osaksi dokumenttia.

Kuvan~\ref{paakasis} käsikirjoitustiedostoa voitaisiin kutsua
pää- tai kantatiedostoksi.
Tämä on se tiedosto, joka käsitellään \LaTeX-ohjelmalla kun tuotetaan valmis
dokumentti luvussa~\ref{ohjelmat} mainittujen ohjeiden mukaisesti.

\pagebreak
\subsubsection{Kuvien lisääminen}

Kuvat ovat objekteja, jotka eivät ole osa normaalia tekstiä. Tavallisesti
kuvat vaeltavat sopivaan paikkaan dokumentissa, kuten sivun ylälaitaan.
Yhtä kuvaa ei pilkota kahdelle sivulle, vaan se sijoitetaan kokonaisena
yhdelle sivulle.

Kuvien ja muiden normaaliin tekstiin kuulumattomien objektien
sisällyttämiseen käytetään ympäristöä \koodi{figure}, jonka käyttö voisi
pääpiirteissään näyttää seuraavalta:
\begin{verbatim}
    \begin{figure}[sijoitus]

      kuvan runko

    \caption{kuvan otsikko} \label{tunniste}
    \end{figure}
\end{verbatim}
Käsikirjoitustiedostoon kirjoitettu komento \verb|\begin{figure}| aloittaa
kuvan muodostaman itsenäisen moduulin ja komento \verb|\end{figure}| päättää
sen. Näiden komentojen väliin kirjoitetaan kuvan runko, joka muodostuu
mistä tahansa tekstistä, \LaTeX-komennoista, jne. Kuva otsikoidaan
\verb|\caption|-komennolla. Lisäksi jokainen kuva kannattaa merkitä
\verb|\label|-komennolla ristiviittausten mahdollistamiseksi.
Ympäristön \koodi{figure} (sekä ympäristön \koodi{table}) kohdalla komennon
\verb|\label| paikka on tarkkaan määrätty: se täytyy sijoittaa
käsikirjoitustiedostoon \verb|\caption|-komennon perään tai sen parametriin.

Koska kuva on tekstiin kuulumaton objekti, se saattaa vaeltaa tiettyjen
sijoittelusääntöjen puitteissa paikasta toiseen. Kuvan sijoitteluun voidaan
vaikuttaa valinnaisella parametrilla \koodi{sijoitus}, joka sisältää yhdestä
kirjaimesta neljään kirjaimeen koostuvan merkkijonon, jossa kukin kirjain
tarkoittaa jotain seuraavista paikoista:
\begin{description}
\item[\koodi{h}]
\emph{Here}: paikkaan tekstissä, missä \koodi{figure}-ympäristö esiintyy.
\item[\koodi{t}]
\emph{Top}: tekstisivun ylälaitaan.
\item[\koodi{b}]
\emph{Bottom}: tekstisivun alalaitaan.
\item[\koodi{p}]
\emph{Page of floats}: erilliselle sivulle, joka ei sisällä tekstiä vaan
pelkkiä kuvia ja taulukkoja (\koodi{table}).
\end{description} 
Parametrin \koodi{sijoitus} puuttuessa oletuksena on sijoitus \koodi{tbp}.
Käytännössä valinnaisella parametrilla \koodi{sijoitus} ilmoitetaan, minne
kuva on lupa sijoittaa, joten on syytä antaa tarpeeksi vaihtoehtoja.

Kuvan runko voi sisältää tekstiä, \LaTeX-komentoja, jne. Yleinen ja helppo
tapa lisätä kuvia dokumenttiin on käyttää valmiita kuvia, jotka on tehty
jollain grafiikan tekoon erikoistuneella ohjelmalla.\footnote{Tällaisia
ohjelmia ovat mm.\ XFig, CorelDraw!, Freehand, Gnuplot ja XPaint.} Kuvien
käsittelyyn suositeltuja LaTeX-pakkauksia ovat \koodi{graphicx},
\koodi{psfig} ja \koodi{subfigure}, joista on ollut puhetta
luvussa~\ref{aloitus}.

Kuvien \koodi{kuva1.eps}, \koodi{kuva2.eps} ja \koodi{kuva3.eps} liittäminen
dokumenttiin voisi tapahtua seuraavilla käsikirjoitustiedostoon
kirjoitettavilla komennoilla, kun käytetään \koodi{graphicx}-pakkausta:
{\koodikoko
\begin{verbatim}
    ...
    \usepackage{graphicx} % Pakkaus kuvien liittämistä varten.
    ...
    \begin{figure}[htb]
    \includegraphics{kuva1.eps}
    \caption{Hieno kuva} \label{ekakuva}
    \end{figure}
    ...
    \begin{figure}[htb]
    \centering \includegraphics{kuva2.eps}
    \caption{Hienompi kuva} \label{tokakuva}
    \end{figure}
    ...
    \begin{figure}[htb]
    \centering \includegraphics{kuva3.eps}
    \caption{Hienoin kuva} \label{kolmaskuva}
    \end{figure}
    ...
\end{verbatim}
}
Kahden jälkimmäisen kuvan lisäämisessä käytetty komento \verb|\centering|
keskittää kuvan sivulle vaakasuunnassa.

\subsection{Lähdeluettelon tuottaminen}
\label{lahdeluettelo}

\begin{sloppypar}
Varsinaista tekstiä seuraa dokumentissa usein lähdeluettelo, jonka
tuottamiseen on kaksi erilaista tapaa: \BibTeX\ ja käsikirjoitustiedostoon
käsin lisättävä \verb|thebibliography|-ympäristö. Seuraavissa aliluvuissa
kuvaillaan lähdeluettelon luonti molemmilla eri tavoilla.
\end{sloppypar}

\subsubsection{\BibTeX -lähdetietokantaa käyttäen}

\BibTeX\ on erillinen ohjelma, joka tuottaa lähdeluettelon dokumenttiin
hankkien tiedot \koodi{BIB}-päätteisistä tietokantatiedostoista eli
\BibTeX-tietokannoista. Kaikki viitteet sijoitetaan siis
\BibTeX-tietokantoihin, joissa jokaista viitettä vastaa yksi tietue ja
tietueen kentät vastaavat viitteen tiettyä ominaisuutta, kuten
kirjoittajaa, nimeä, julkaisijaa, jne. Kun \BibTeX-ohjelmaa käyttää, se
poimii tekstissä \verb|\cite|-komennolla viitatut lähteet tietokannasta ja
kirjoittaa \koodi{BBL}-päätteiseen tiedostoon
\verb|thebibliography|-ympäristön sisältöineen, jonka perusteella \LaTeX\
sitten osaa luoda lähdeluettelon dokumenttiin.

\BibTeX-ohjelman käytön etuna on se, että samaa \BibTeX-tietokantaa voi
käyttää useissa dokumenteissa ja että viittaamatta jäävät tiedot voivat
vapaasti jäädä ko.\ tiedostoon. Samoin on mahdollista tuottaa lista
tiedostossa jo olevista viitteistä kommenttimerkintöineen --- tästä on
hyötyä tiedonhakuvaiheessa muistiinpanojen koordinaattorina. Lisäksi
\BibTeX-ohjelman tuottamien lähdeluettelojen ulkoasu perustuu erityisiin
\BibTeX-tyylitiedostoihin, mikä antaa mahdollisuuden tehdä erityyppisiä
lähde- ja kirjallisuusluetteloita.

%%\DescribeMacro{\bibliography}
\BibTeX-ohjelmaa käytettäessä lähdeluettelo kootaan tietokantatiedostojen
tiedoista. Jos siis lähdetiedot sijaitsevat esimerkiksi tiedostoissa
\koodi{insect.bib} ja \koodi{animal.bib}, \LaTeX-käsikirjoitustiedostoon
kirjoitetaan komento
\begin{verbatim}
    \bibliography{insect,animal}
\end{verbatim}
siihen kohtaan, johon valmiissa dokumentissa halutaan sijoittaa
\BibTeX-ohjelman luoma lähdeluettelo.

%%\DescribeMacro{\cite}
Tietokantatiedostojen \koodi{insect.bib} ja \koodi{animal.bib} sisältämiin
lähdetietoihin viitataan tekstissä \verb|\cite|-komennolla, jolla
määritetyt lähteet \BibTeX\ poimii mukaan luomaansa lähdeluetteloon.
Lähdeluetteloon voidaan sisällyttää myös sellaisia lähteitä, joihin ei
viitata tekstissä \verb|\cite|-komennolla. Tämä tapahtuu
\verb|\nocite|-komen\-nol\-la, jonka parametrina annetaan pilkulla erotettuina
lähteiden tunnisteet. Kun lähdeluetteloon halutaan sisällyttää kaikki
tietokantatiedostojen tietueet, tämä onnistuu kirjoittamalla
\verb|\nocite{*}| ennen komentoa \verb|\bibliography{insect,animal}|.

%%\DescribeMacro{\bibliographystyle}
\BibTeX-ohjelman käyttöä varten \LaTeX-käsikirjoitustiedostossa täytyy olla
jossain kohtaa \verb|\begin{document}|-komennon jälkeen komento
\verb|\bibliographystyle|, jonka parametrina annetaan lähdeluettelon
luomisessa käytetty \BibTeX-tyyli. \BibTeX-tyylitiedostot ovat
\koodi{BST}-päätteisiä tiedostoja, esim.\ \koodi{alpha.bst}, mutta\linebreak
\verb|\bibliographystyle|-komennon parametriksi kirjoitetaan vain
tyylitiedoston nimi ilman päätettä eli tässä tapauksessa \koodi{alpha}.

\LaTeX-pakkauksen standardityylit lähdeluetteloille ovat \koodi{plain},
\koodi{unsrt}, \koodi{alpha} ja\linebreak \koodi{abbrv}. Lisäksi on olemassa lukuisia
muita \koodi{BST}-päätteisiä tyylitiedostoja, jotka perustuvat enemmän tai
vähemmän standardityyleihin. Eräs suositeltava \BibTeX-tyyli on Helsingin
yliopiston Tietojenkäsittelytieteen laitoksella kehitetty \bibtyyli,
joka on modifioitu standardityylistä \verb|alpha|.

Tyyli \bibtyyli\ käyttää kielisensitiivisissä kohdissa joko suomen, ruotsin
tai englannin kieltä sen mukaan, mikä näistä kielistä on \LaTeX-pakkauksen
\verb|Babel| käytössä lähdeluettelon luontiajankohtana. Tämä tietenkin
määritetään käsikirjoitustiedostossa. Tyylin \bibtyyli\ käytön ehtona on
siis se, että paketti \verb|Babel| on käytössä ja joko suomen, ruotsin tai
englannin kieli on aktiivisena.

Dokumenttiluokan \luokka\ kanssa tyylin \bibtyyli\ käyttäminen on helppoa,
sillä \luokka\ ottaa \verb|Babel|-pakkauksen automaattisesti käyttöönsä ja
valitsee aktiivikseksi kieleksi joko suomen, ruotsin tai englannin kielen.

Kun siis tietokantatiedostot \koodi{insect.bib} ja \koodi{animal.bib}
sisältävät niiden lähteiden tiedot, joihin käsikirjoitustiedostoon
kirjoitetussa tekstissä viitataan \verb|\cite|-komennolla,
käsikirjoitustiedostoon riittää kirjoittaa esimerkiksi
\begin{verbatim}
    \bibliographystyle{alpha}
    \bibliography{insect,animal}
\end{verbatim}
siihen kohtaan, johon lähdeluettelon halutaan ilmestyvän valmiissa
dokumentissa. Näin \BibTeX\ saa tietoonsa, mistä tietokannoista se poimii
viitteiden tiedot ja mitä tyyliä se käyttää lähdeluettelon muokkaamiseen.
Kun käsikirjoitustiedosto on käsitelty luvussa~\ref{ohjelmat} mainitulla
tavalla, lähdeluettelon valmiiseen dokumenttiin tuottanut
\koodi{thebibliography}-ympäristö eli lähdeluettelon ''käsikirjoitus''
löytyy \koodi{BBL}-päätteisestä tiedostosta, jonka nimen alkuosa on sama
kuin \LaTeX-käsi\-kir\-joi\-tus\-tie\-dos\-ton.

Lisätietoja \BibTeX-ohjelman käytöstä ja \BibTeX-tietokantojen teosta
löytyy dokumentista \emph{\BibTeX-tyylin \koodi{tktl} käyttöohje}, joka
on samalla \BibTeX-tyylin \koodi{tktl} käyttöohje.

\subsubsection{Dokumentin sisäistä viiteympäristöä käyttäen}

%%\DescribeEnv{thebibliography}
Ympäristön \koodi{thebibliography} avulla lähdeluettelon tuottaminen käsin
onnistuu kirjoittamalla \LaTeX-käsikirjoitustiedostoon esimerkiksi
\begin{verbatim}
    \begin{thebibliography}{XXX88}
    \bibitem[Gri87]{grimm87}
    Grimm, S. S.,
    {\em How to write computer documentation for users.}
    Van Nostrand Reinhold Co.,
    New York, 1987.
       .
       .
       .
    \end{thebibliography}
\end{verbatim}

Kun lähdeluettelon tuottaa käsin, \koodi{thebibliography}-ympäristö
kirjoitetaan käsikirjoitustiedostoon.
%%\DescribeMacro{\cite}
Käsikirjoitustiedoston tekstiosassa lähteeseen viitataan
\verb|\cite|-komennolla, esim.\ \verb|\cite{grimm87}|.

\subsection{Viimeinen sivu ennen liitteitä}
Useimmissa tapauksissa lähdeluettelo päättää varsinaisen dokumentin.
%%\DescribeMacro{\lastpage}
Sivumäärän laskemiseksi lähdeluettelokomentojen jälkeen kannattaa
kirjoittaa käsikirjoitustiedostoon komento \verb|\lastpage|, jolla
ilmoitetaan että kyseessä on viimeinen varsinainen sivu ennen
liitteiden määrittämistä tai käsikirjoitustiedoston loppumista.

%%\DescribeMacro{\numberofpagesinformation}
%%\DescribeMacro{\numberofpages}
%%\DescribeMacro{\numberofappendixpages}
Jos käsikirjoitustiedoston alussa kirjoitettiin
\verb|\numberofpagesinformation|\linebreak \verb|{\numberofpages\| \verb|sivua +|
\verb|\numberofappendixpages[100]| \verb|liitesivua}|, tulostuu valmiin
dokumentin tiivistelmäsivulle \emph{Sivumäärä}-kohtaan esim.\
\emph{10~sivua +\linebreak 105~liitesivua}. Tämä tarkoittaa sitä, että dokumentin
varsinainen tekstiosuus sisältää 10~sivua, dokumentti sisältää
5~käsikirjoitustiedostossa määritettyä liitesivua ja dokumentin
käsikirjoitustiedostoon kuulumattomia sivuja on~100~kpl.

\subsection{Liitteet}
Usein lähdeluetteloa seuraavat dokumentissa liitteet.
%%\DescribeMacro{\appendices}
Luokkaa \luokka\ käyttävässä käsikirjoitustiedossa liiteosan aloittaminen
ilmoitetaan \verb|\appendices|-komennolla.

Liitteitä ei tarvitse määrittää dokumentin käsikirjoitustiedostossa.
Esimerkiksi ohjelmalistauksen sisällyttäminen
\LaTeX-käsikirjoitustiedostoon ei olisi mielekästä. Sen sijaan liitteen
nimen sisällyttäminen dokumentin sisällysluetteloon on mielekästä.

%%\DescribeMacro{\internalappendix}
%%\DescribeMacro{\externalappendix}
\begin{sloppypar}
Dokumentin käsikirjoitustiedostoon sisältyvät liitteet lisätään
\verb|\internalappendix|-komennolla ja käsikirjoitustiedoston ulkopuoliset
liitteet \verb|\externalappendix|-\linebreak komennolla. Molemmille komennoille
annetaan parametreina liitteen nimi ja juokseva numero, esim.\
\verb|\internalappendix{1}{Malli ABC}| tai
\verb|\externalappendix{3}{Koodi}|.
\end{sloppypar}

Komento \verb|\internalappendix| tuottaa liitteen otsikon valmiin
dokumentin vastaavaan kohtaan, jossa sitä käsikirjoitustiedostossa
kutsutaan, ja liittää dokumentin sisällysluetteloon liitteen nimen sekä
aloittaa aina uuden sivun. Kunkin\linebreak \verb|\internalappendix|-komentoa
seuraavan liitteen sivut numeroidaan automaattisesti erikseen valmiissa
dokumentissa. Käsikirjoitustiedostoon kirjoitettu komento
\verb|\externalappendix| ainoastaan liittää valmiin dokumentin
sisällysluetteloon liitteen nimen.

Tarpeen vaatiessa liitteiden numerointi voidaan hoitaa
\koodi{appendix}-laskurilla, jonka arvo kasvaa jokaisen
\verb|\internalappendix|- ja \verb|\externalappendix|-komennon
kutsukerralla.
%%\DescribeMacro{\theappendix}
Laskurin \koodi{appendix} senhetkisen arvon saa ilmestymään dokumentiin
kirjoittamalla käsikirjoitustiedostoon \verb|\theappendix|. Laskurin arvo
on olemassa olevien liitteiden määrä lisättynä yhdellä.

\subsection{Luokan käyttäjän omat modifikaatiot}

Dokumenttiluokan \luokka\ käyttäjä voi halutessaan kumota luokan
määrittelyjä ja oletusasetuksia. Luokan toteutuskuvaus toiminee tässä
apuna. Lisäksi muiden luokkien ja pakkausten toteutusdokumentaatiot sekä
erilaiset \LaTeX-oppaat auttavat käyttäjää tutustumaan
\LaTeX-ladontajärjestelmään pintaa syvemmältä.

\subsection{Ohjelmien käyttö}
\label{ohjelmat}

Seuraavassa on lyhyt johdatus \LaTeX-ladontajärjestelmään liittyvien
ohjelmien käyttöön. Huomaa, että ohjeet pätevät useimpiin ajantasalla
oleviin \TeX-toteutuksiin
mutta eivät välttämättä kaikkiin. Eroavaisuuksia saattaa löytyä ja
parhaiten niistä osannee informoida \TeX-järjestelmäsi ylläpito.

\LaTeX-käsikirjoitustiedostoista saa
PostScript-tiedostoja kirjoittamalla seuraavat komennot
komentoriville:
\begin{verbatim}
    latex <tiedosto>.tex
    latex <tiedosto>.tex
    dvips -o <tiedosto>.ps <tiedosto>.dvi
\end{verbatim}

Mikäli dokumentin lähdeluettelon tuottamiseen käytetään \BibTeX-ohjelmaa,
edelliset ohjeet muuttuvat seuraavanlaisiksi:
\begin{verbatim}
    latex <tiedosto>.tex
    bibtex <tiedosto>
    latex <tiedosto>.tex
    latex <tiedosto>.tex
    dvips -o <tiedosto>.ps <tiedosto>.dvi
\end{verbatim}

Yllä olevien ohjeiden mukaisesti \LaTeX-käsikirjoitustiedostoista saa myös
PDF-tie\-dos\-to\-ja, kun komento \koodi{dvips -o <tiedosto>.ps <tiedosto>.dvi}
korvataan komennolla \koodi{dvipdf <tiedosto>.dvi}

\LaTeX-käsikirjoitustiedostoista saa PDF-tiedostoja suoraan kirjoittamalla
seuraavan komennon komentoriville:
\begin{verbatim}
    pdflatex <tiedosto>
\end{verbatim}

Tässä \koodi{<tiedosto>} on \LaTeX-käsikirjoitustiedostosi nimi ilman
mitään tiedostopäätteitä, kuten \koodi{.tex} tai \koodi{.aux}. Enemmän
tietoa ohjelmien käytöstä saa optiolla \verb|-help| tai \verb|--help|, eli
esimerkiksi komento \koodi{latex -help} tuottaa \LaTeX-ohjelman ohjeet.
Unix-koneissa ohjelmista saattaa löytyä man-sivu, jonka pääsee lukemaan
komennolla \koodi{man <ohjelma>}. Esimerkiksi \LaTeX-ohjelman man-sivu
aukeaa komennolla \koodi{man latex}.

%%\section{Lisätietoja}
%%
%%Helsingin yliopiston Tietojenkäsittelytieteen laitoksen \LaTeX-sivun
%%osoite on \url{http://www.cs.helsinki.fi/...}. Sivulta löytyy viitteitä
%%sekä laitoksen omaan LaTeX-materiaaliin että muihin hyödyllisiin
%%\LaTeX-lähteisiin. Myös tämä dokumentti on saatavilla kyseisellä
%%WWW-sivulla.
%%
\lastpage
\appendices

\end{document}

%% Yhdelle riville 75 merkkiä välilyönnit mukaan lukien:
%% 3456789012345678901234567890123456789012345678901234567890123456789012345
%</ohjeet>
%    \end{macrocode}
%
% \Finale
\endinput
